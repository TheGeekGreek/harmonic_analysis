\section{The Real Method}
In this last section we are concerned with one interpolation theorem which uses real variables techniques for its proof. This stands in contrast with the complex variables techniques used for the previous two theorems. Proofs by real variables techniques are often more straight forward and do not make use of advanced theorems. Therefore the proofs are likely longer and less natural.

\subsection{The Marcinkiewicz Interpolation Theorem}
This theorem applies to sublinear operators (aswell as for quasilinear operators by a slight change of the constant) which is in comparison to the linearity assumed by the other interpolation theorems less restrictive on the nature of the operator in question. Unfortunately, this theorem will not provide an exact answer to the question raised in the introduction since the constant will differ as well as the continuity property of the estimate. Fix a measure space $(X,\mu)$. Recall, that for $0 < p < \infty$ the space \emph{weak} $L^p(X)$ is defined as the set of all $\mu$-measurable functions $f$ such that

\begin{equation*}
	\norm[0]{f}_{L^{p,\infty}} := \sup\cbr[0]{\gamma d_f(\gamma)^{1/p}: \gamma > 0} < \infty.
\end{equation*}

The space \emph{weak} $L^{\infty}(X)$ is by definition $L^{\infty}(X)$ (see \cite[5]{grafakos:fourier:2014}).

\vspace{2mm}

\begin{mdframed}
	\begin{theorem}\emph{(The Marcinkiewicz interpolation theorem)}
		Let $(X,\mu)$, $(Y,\nu)$ be two $\sigma$-finite measure spaces and $0 < p_0 < p_1 \leq \infty$. Further let $T$ be a sublinear operator defined on
		
		\begin{equation*}
			L^{p_0} + L^{p_1} := \cbr[0]{ f_0 + f_1 : f_0 \in L^{p_0}(X), f_1 \in L^{p_1}(X)}
		\end{equation*}
		
		\noindent and taking values in the space of measurable functions on $Y$. Assume that there exist $0 \leq A_0,A_1 < \infty$ such that

		\begin{equation}
			\norm[0]{T(f)}_{L^{p_0,\infty}} \leq A_0 \norm[0]{f}_{L^{p_0}} \qquad \text{and} \qquad \norm[0]{T(f)}_{L^{p_1,\infty}} \leq A_1 \norm[0]{f}_{L^{p_1}}\label{hyp:fp_0fp_1}
		\end{equation}

		\noindent for all $f \in L^{p_0}$, $f \in L^{p_1}$ respectively. Then for all $p_0 < p < p_1$ and for all $f \in L^p$ we have the estimate

		\begin{equation}
			\norm[0]{T(f)}_{L^p} \leq A \norm[0]{f}_{L^p}
		\end{equation}

		\noindent where

		\begin{equation}
			A := 2\del[3]{ \frac{p}{p - p_0} + \frac{p}{p_1 - p} }^{1/p}A_0^{\frac{\frac{1}{p} - \frac{1}{p_1}}{\frac{1}{p_0}-\frac{1}{p_1}}}A_1^{\frac{\frac{1}{p_0}-\frac{1}{p}}{\frac{1}{p_0}-\frac{1}{p_1}}}.
			\label{eq:constant}
		\end{equation}
	\end{theorem}
\end{mdframed}

\begin{proof}
Let us first consider the case \underline{$p_1 < \infty$}. Fix $f \in L^p(X)$, $\alpha > 0$ and $\delta > 0$ ($\delta$ will be determined later). We split $f$ using so-called \emph{cut-off} functions, by stipulating $f = f_0(\cdot;\alpha,\delta) + f_1(\cdot;\alpha,\delta)$, where $f_0(\cdot;\alpha,\delta)$ is the \emph{unbounded part of $f$} and $f_1(\cdot;\alpha,\delta)$ is the \emph{bounded part of $f$}, defined by

\begin{gather}
	\begin{aligned}
		f_0(x;\alpha,\delta) &:= \begin{cases}
		f(x), & \abs[0]{ f(x)} > \delta \alpha,\\
		0, & \abs[0]{ f(x)} \leq \delta \alpha.
		\end{cases}\\
		f_1(x;\alpha,\delta) &:= \begin{cases}
			f(x), & \abs[0]{ f(x) } \leq \delta \alpha,\\
		0, & \abs[0]{ f(x)} > \delta \alpha.
		\end{cases}
	\end{aligned}
	\label{eq:cut_off}
\end{gather}

\noindent for $x \in X$. To facilitate reading we will omit the dependency of $f_0(\cdot;\alpha,\delta)$ and $f_1(\cdot;\alpha,\delta)$ upon the parameters $\alpha$ and $\delta$ in what follows and we simply write $f_0$, $f_1$ respectively. 
	
\begin{lemma}
	The functions $f_0$ and $f_1$ defined in \textup{(\ref{eq:cut_off})} satisfy $f_0 \in L^{p_0}(X)$ and $f_1 \in L^{p_1}(X)$.
	\label{lem:f0f1}
\end{lemma}
	
\begin{proof}
Since $p_0 < p$ we have 

\begin{equation*}
	\begin{aligned}
		\norm[0]{f_0}^{p_0}_{L^{p_0}} &= \int_{X} \abs[0]{ f_0}^{p_0} \d\mu\\
		&=\int_{X} \abs[0]{ f }^{p_0} \chi_{\cbr[0]{ \abs[0]{f} > \delta\alpha}} \d\mu\\
		&= \int_{\cbr[0]{\abs[0]{ f } > \delta\alpha }} \abs[0]{ f}^{p_0}\d\mu\\ 
		&= \int_{\cbr[0]{\abs[0]{ f} > \delta\alpha }} \abs[0]{ f }^p \abs[0]{ f }^{p_0 - p} \d\mu\\
		&= \int_{\cbr[0]{\abs[0]{ f} > \delta\alpha }} \frac{\abs[0]{ f}^p}{\abs[0]{ f}^{p - p_0}} \d\mu\\
		&\leq \frac{1}{(\delta\alpha)^{p - p_0}} \int_{\cbr[0]{\abs[0]{ f} > \delta\alpha }} \abs[0]{ f}^p \d\mu\\
		&= (\delta\alpha)^{p_0 - p} \int_{X} \abs[0]{ f }^p \chi_{\cbr[0]{\abs[0]{ f} > \delta\alpha }} \d\mu\\
		& \leq (\delta\alpha)^{p_0 - p} \int_{X} \abs[0]{ f }^p d\mu\\
		&= (\delta\alpha)^{p_0 - p} \norm[0]{f}^p_{L^p}
	\end{aligned}
\end{equation*}

\noindent and so $f_0 \in L^{p_0}(X)$. In the same manner it can be checked that $f_1 \in L^{p_1}(X)$ by the estimate $\norm[0]{f_1}^{p_1}_{L^{p_1}} \leq (\delta\alpha)^{p_1 - p}\norm{f}_{L^p}^p$. The set $\cbr[0]{\abs[0]{f} > \delta\alpha}$ is clearly measurable since $\abs[0]{f}^2 = \del[0]{\Re f}^2 + \del[0]{\Im f}^2$ and so by the measurability of $f$ its real and imaginary part is measurable. Furthermoremore the sum and product of measurable functions is again a measurable function and so we have that $\abs[0]{f}^2$ is measurable and therefore also $\abs[0]{f}$.
\end{proof}

By lemma \ref{lem:f0f1} we therefore have $f = f_0 + f_1 \in L^{p_0} + L^{p_1}$. 

\begin{lemma}
	For fixed $\alpha > 0$, the distribution function $d_{T(f)}(\alpha)$ obeys an upper bound of the form

	\begin{equation*}
		d_{T(f)}(\alpha) \leq  \del{\frac{A_0}{\alpha/2}}^{p_0} \norm[0]{f_0}^{p_0}_{L^{p_0}} + \del{\frac{A_1}{\alpha/2}}^{p_1} \norm[0]{f_1}^{p_1}_{L^{p_1}}
	\end{equation*}
\end{lemma}

\begin{proof}
	Since $T$ is a sublinear operator we have $\abs[0]{ T(f)} = \abs[0]{ T(f_0 + f_1)} \leq \abs[0]{ T(f_0)} + \abs[0]{ T(f_1)}$. Thus for any $y \in Y$ with $\abs[0]{ T(f)(y)} > \alpha$ we therefore have either $\abs[0]{ T(f_0)(y)} > \alpha/2$ or $\abs[0]{ T(f_1)(y) } > \alpha/2$ 
	\footnote{Without loss of generality assume $\abs[0]{ T(f_0)(y) } \leq \abs[0]{ T(f_1)(y)} $. Then we have $\alpha < \abs[0]{ T(f)(y)} \leq \abs[0]{ T(f_0)(y) } + \abs[0]{ T(f_1)(y)} \leq 2\abs[0]{ T(f_1)(y)}$.}
		. Hence

\begin{equation*}
	\cbr[0]{\abs[0]{ T(f)} > \alpha } \subseteq \cbr[0]{\abs[0]{ T(f_0) } > \alpha/2 } \cup \cbr[0]{\abs[0]{ T(f_1)} > \alpha/2 }
\end{equation*}

\noindent and so by the monotonicity and subadditivity property of the measure $\mu$ we have

\begin{equation*}
	\begin{aligned}
		d_{T(f)}(\alpha) &= \mu(\cbr[0]{\abs[0]{ T(f)} > \alpha})\\
		&\leq \mu(\cbr[0]{\abs[0]{ T(f_0)} > \alpha/2 } \cup \cbr[0]{\abs[0]{ T(f_1)} > \alpha/2 })\\
		&\leq \mu(\cbr[0]{\abs[0]{ T(f_0) } > \alpha/2 }) + \mu(\cbr[0]{\abs[0]{ T(f_1)} > \alpha/2 })\\
		&= d_{T(f_0)}(\alpha/2) + d_{T(f_1)}(\alpha/2)
		\label{est:T}
	\end{aligned}
\end{equation*}

Now by first one of the hypotheses (\ref{hyp:fp_0fp_1}) we can estimate $d_{T(f_0)}(\alpha/2)$ as follows:

\begin{equation*}
	\begin{aligned}
		d_{T(f_0)}(\alpha/2) &= \del{\frac{\alpha/2}{\alpha/2}}^{p_0} d_{T(f_0)}(\alpha/2)\\
		&\leq \del{\frac{1}{\alpha/2}}^{p_0} \del{\sup\cbr[0]{ \gamma d_{T(f_0)}(\gamma)^{1/p_0}: \gamma > 0}}^{p_0}\\
		& = \del{\frac{1}{\alpha/2}}^{p_0} \norm[0]{T(f_0)}^{p_0}_{L^{p_0,\infty}}\\
		& \leq \del{\frac{A_0}{\alpha/2}}^{p_0} \norm[0]{f_0}^{p_0}_{L^{p_0}}
	\label{est:p_0}
	\end{aligned}
\end{equation*}

Analogously, we get  

\begin{equation*}
	d_{T(f_1)}(\alpha/2) \leq \del{\frac{A_1}{\alpha/2}}^{p_1} \norm[0]{f_1}^{p_1}_{L^{p_1}}
	\label{est:p_1}
\end{equation*}

\noindent by the second one of the hypotheses (\ref{hyp:fp_0fp_1}).
\end{proof}

By

\begin{equation*}
	\begin{aligned}
		\int_0^{\frac{1}{\delta}\abs[0]{ f}}\alpha^{p-p_0-1} \d\alpha = 
			\begin{cases}
				\frac{1}{p-p_0}\frac{1}{\delta^{p-p_0}}\abs[0]{ f }^{p - p_0}, & p \geq p_0 + 1\\
				\lim_{\omega \searrow 0} \int_\omega^{\frac{1}{\delta}\abs[0]{ f}}\alpha^{p-p_0-1} \d\alpha\\
				= \lim_{\omega \searrow 0}\sbr{\frac{1}{p-p_0}\alpha^{p - p_0}}_\omega^{\frac{1}{\delta}\abs[0]{ f}}\\
				= \frac{1}{p-p_0}\sbr{\frac{1}{\delta^{p-p_0}}\abs[0]{ f }^{p - p_0} - \lim_{\omega \searrow 0} \omega^{p-p_0}}\\
				= \frac{1}{p-p_0}\frac{1}{\delta^{p-p_0}} \abs[0]{ f}^{p - p_0}, & p_0 < p < p_0 + 1
			\end{cases}
	\end{aligned}
\end{equation*}

\noindent and

\begin{equation*}
	\begin{aligned}
		\int_{\frac{1}{\delta}\abs[0]{ f}}^{\infty}\alpha^{p-p_1-1} \d\alpha &= \lim_{\omega \nearrow \infty} \sbr{ \frac{1}{p - p_1} \alpha^{p - p_1}}^\omega_{\frac{1}{\delta}\abs[0]{ f}}\\
		&= \frac{1}{p - p_1} \sbr{ \lim_{\omega \nearrow \infty} \omega^{p - p_1} - \frac{1}{\delta^{p - p_1}} \abs[0]{ f}^{p - p_1}}\\
		&= \frac{1}{p_1 - p}\frac{1}{\delta^{p-p_1}} \abs[0]{ f }^{p - p_1}
	\end{aligned}
\end{equation*}

\noindent and the representation (see \cite[5]{grafakos:fourier:2014}) 

\begin{equation*}
	\norm[0]{f}^p_{L^p} = p \int_0^{\infty} \alpha^{p-1}d_f(\alpha) \d\alpha
\end{equation*}

	\noindent for $ 0 < p < \infty$ we get

\begin{equation}
	\begin{aligned}
		\norm[0]{T(f)}^p_{L^p} = &~p\int_0^{\infty}\alpha^{p-1}d_{T(f)} \d\alpha\\
		\leq &~p(2A_0)^{p_0}\int_0^{\infty}\alpha^{p-p_0-1} \int_{\cbr[0]{\abs[0]{f} > \delta \alpha}} \abs[0]{ f}^{p_0}\d\mu \d\alpha\\
		&+ p(2A_1)^{p_1}\int_0^{\infty}\alpha^{p-p_1-1} \int_{\cbr[0]{\abs[0]{ f } \leq \delta \alpha}} \abs[0]{ f}^{p_1}\d\mu \d\alpha\\
		= &~p(2A_0)^{p_0}\int_{\cbr[0]{\abs[0]{ f } > 0}} \abs[0]{ f }^{p_0} \int_0^{\frac{1}{\delta}\abs[0]{ f}}\alpha^{p-p_0-1} \d\alpha \d\mu\\
		&+ p(2A_0)^{p_0}\int_{\cbr[0]{\abs[0]{f} = 0}} \abs[0]{ f }^{p_0} \int_0^{\frac{1}{\delta}\abs[0]{ f}}\alpha^{p-p_0-1} \d\alpha \d\mu\\
		&+ p(2A_1)^{p_1}\int_X \abs[0]{ f}^{p_1} \int_{\frac{1}{\delta}\abs[0]{ f}}^{\infty}\alpha^{p - p_1 - 1} \d\alpha \d\mu\\
		= &~p(2A_0)^{p_0}\int_X \abs[0]{ f }^{p_0} \int_0^{\frac{1}{\delta}\abs[0]{ f}}\alpha^{p-p_0-1} \d\alpha \d\mu\\
		&+ p(2A_1)^{p_1}\int_X \abs[0]{ f}^{p_1} \int_{\frac{1}{\delta}\abs[0]{ f}}^{\infty}\alpha^{p - p_1 - 1} \d\alpha \d\mu\\
		=&~\frac{p(2A_0)^{p_0}}{p-p_0}\frac{1}{\delta^{p-p_0}}\int_X \abs[0]{f}^{p_0}\abs[0]{ f }^{p-p_0}  \d\mu\\
		&+ \frac{p(2A_1)^{p_1}}{p_1-p}\frac{1}{\delta^{p-p_1}}\int_X \abs[0]{ f}^{p_1} \abs[0]{ f}^{p-p_1}\d\mu\\
		= &~p\del{ \frac{(2A_0)^{p_0}}{p - p_0}\frac{1}{\delta^{p - p_0}} + \frac{(2A_1)^{p_1}}{p_1 - p}\delta^{p_1 - p}}\norm[0]{f}_{L^p}^p
	\end{aligned}
	\label{est:Tfp}
\end{equation}

		Now we pick $\delta > 0$ such that $(2A_0)^{p_0}\delta^{p_0 - p} = (2A_1)^{p_1}\delta^{p_1 - p}$. Solving for $\delta$ yields 

		\begin{equation}
			\delta = \frac{1}{2} \del{ \frac{A_0}{A_1}}^{p_1/(p_1 - p_0)}
			\label{eq:delta}
		\end{equation}

		Substituting (\ref{eq:delta}) in estimate (\ref{est:Tfp}) leads to

		\begin{equation*}
			\begin{aligned}
				\norm[0]{T(f)}_{L^p}^p &\leq p\del{ \frac{(2A_0)^{p_0}}{p - p_0}\frac{2^{p - p_0}A_1^\frac{p_1(p-p_0)}{p_1-p_0}}{A_0^\frac{p_0(p-p_0)}{p_1 - p_0}} + \frac{(2A_1)^{p_1}}{p_1 - p} \frac{A_0^\frac{p_0(p_1 - p)}{p_1 - p_0}}{2^{p_1 - p}A_1^\frac{p_1(p_1 - p)}{p_1 - p_0}} }\norm[0]{f}_{L^p}^p\\
				&=  2^pp\del{ \frac{A_0^\frac{p_0(p_1 - p)}{p_1 - p_0}A_1^\frac{p_1(p-p_0)}{p_1-p_0}}{p - p_0} + \frac{A_0^\frac{p_0(p_1 - p)}{p_1 - p_0}A_1^\frac{p_1(p - p_0)}{p_1 - p_0}}{p_1- p} }\norm[0]{f}_{L^p}^p\\
			\end{aligned}
		\end{equation*}

		\noindent and taking the $p$-th power finally yields

		\begin{equation*}
			\begin{aligned}
				\norm[0]{T(f)}_{L^p} &\leq 2\del{ \frac{p}{p - p_0} + \frac{p}{p_1- p} }^{1/p} A_0^\frac{p_0(p_1 - p)}{p(p_1 - p_0)}A_1^\frac{p_1(p - p_0)}{p(p_1 - p_0)}\norm[0]{f}_{L^p}\\
				&= 2\del{ \frac{p}{p - p_0} + \frac{p}{p_1- p} }^{1/p} A_0^{\frac{p_0(p_1 - p)}{p(p_1 - p_0)}\frac{p_1}{p_1}}A_1^{\frac{p_1(p - p_0)}{p(p_1 - p_0)}\frac{p_0}{p_0}}\norm[0]{f}_{L^p}\\
				&= 2\del{ \frac{p}{p - p_0} + \frac{p}{p_1- p} }^{1/p} A_0^{\frac{\frac{p_1 - p}{pp_1}}{\frac{p_1 - p_0}{p_0p_1}}}A_1^{\frac{\frac{p - p_0}{p_0p}}{\frac{p_1 - p_0}{p_0p_1}}}\norm[0]{f}_{L^p}\\
				&= 2\del{ \frac{p}{p - p_0} + \frac{p}{p_1- p} }^{1/p} A_0^{\frac{\frac{1}{p} - \frac{1}{p_1}}{\frac{1}{p_0} - \frac{1}{p_1}}}A_1^{\frac{\frac{1}{p_0} - \frac{1}{p}}{\frac{1}{p_0} - \frac{1}{p_1}}}\norm[0]{f}_{L^p}
			\end{aligned}
		\end{equation*}

		Assume \underline{$p_1 = \infty$}. We again use the cut-off functions defined in (\ref{eq:cut_off}) to decompose $f$.  Since $\cbr[0]{\abs[0]{ f_1} > \delta\alpha } = \varnothing$, we have 

\begin{equation*}
	\norm[0]{T(f_1)}_{L^\infty} \leq A_1 \norm[0]{f_1}_{L^\infty} = A_1 \inf \cbr[0]{B > 0: \mu(\cbr[0]{\abs[0]{ f_1 } > B}) = 0 } \leq A_1\delta\alpha = \alpha/2
\end{equation*}

\noindent provided we define $\delta := 1/(2A_1)$. Therefore the set $\cbr[0]{\abs[0]{ T(f_1)} > \alpha/2}$ has measure zero (this is immediate since $\norm[0]{T(f_1)}_{L^\infty} =  \inf \cbr[0]{B > 0: \mu(\cbr[0]{\abs[0]{ T(f_1)} > B}) = 0 } \leq \alpha/2 $ Thus similar to the previous case we get 

\begin{equation*}
	d_{T(f)}(\alpha) \leq d_{T(f_0)}(\alpha/2)
\end{equation*}

	\noindent and again the first of the hypotheses (\ref{hyp:fp_0fp_1}) yields 
	
\begin{equation*}
	d_{T(f_0)}(\alpha/2) \leq \del{\frac{A_0}{\alpha/2}}^{p_0} \int_{\cbr[0]{2A_1\abs[0]{ f } > \alpha}} \abs[0]{ f }^{p_0}\d\mu
\end{equation*}

	Thus by

	\begin{equation}
		\begin{aligned}
			\norm[0]{T(f)}_{L^p}^p &= p \int_0^{\infty} \alpha^{p-1}d_{T(f)} \d\alpha\\
			&\leq p (2A_0)^{p_0} \int_0^{\infty} \alpha^{p-p_0-1} \int_{\cbr[0]{2A_1\abs[0]{ f } > \alpha}} \abs[0]{ f }^{p_0}\d\mu \d\alpha\\
			&= p(2A_0)^{p_0} \int_X \abs[0]{ f}^{p_0} \int_0^{2A_1\abs[0]{f }} \alpha^{p - p_0 - 1}\d\alpha\d\mu\\
			&= \frac{2^ppA_0^{p_0}A_1^{p - p_0}}{p - p_0} \int_X \abs{ f}^{p} \d\mu\\
			&= \frac{2^ppA_0^{p_0}A_1^{p - p_0}}{p - p_0} \norm[0]{f}_{L^p}^p
			\label{est:p_1_infty}
		\end{aligned}
	\end{equation}

	That the constant $2^ppA_0^{p_0}A_1^{p - p_0}/(p - p_0)$ found in (\ref{est:p_1_infty}) is the $p$-th power of the one stated in the theorem can be easily seen by passing the constant (\ref{eq:constant}) to the limit $p_1 \rightarrow \infty$.
\end{proof}
