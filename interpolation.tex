%%%%%%%%%%%%%%%%%%%%%%%%%%%%%%%%%%%%%%%%%%%%%%%%%%%%%%%%%%%%%%%%%%%%%%%%%%
%Author:																 %
%-------																 %
%Yannis Baehni at University of Zurich									 %
%baehni.yannis@uzh.ch													 %
%																		 %
%Version log:															 %
%------------															 %
%06/02/16 . Basic structure												 %
%04/08/16 . Layout changes including section, contents, abstract.		 %
%%%%%%%%%%%%%%%%%%%%%%%%%%%%%%%%%%%%%%%%%%%%%%%%%%%%%%%%%%%%%%%%%%%%%%%%%%

%Page Setup
\documentclass[
	11pt, 
	oneside, 
	a4paper,
	reqno,
	final
]{amsart}

\usepackage[
	left = 3cm, 
	right = 3cm, 
	top = 3cm, 
	bottom = 3cm
]{geometry}

%Headers and footers
\usepackage{fancyhdr}
	\pagestyle{fancy}
	%Clear fields
	\fancyhf{}
	%Header right
	\fancyhead[R]{
		\footnotesize
		Yannis B\"{a}hni\\
		\href{mailto:yannis.baehni@uzh.ch}{yannis.baehni@uzh.ch}
	}
	%Header left
	\fancyhead[L]{
		\footnotesize
		MAT694 Seminar: Introduction to Harmonic Analysis\\
		HS16
	}
	%Page numbering in footer
	\fancyfoot[C]{\thepage}
	%Separation line header and footer
	\renewcommand{\headrulewidth}{0.4pt}
	%\renewcommand{\footrulewidth}{0.4pt}
	
	\setlength{\headheight}{19pt} 

%Title
\usepackage[foot]{amsaddr}
\usepackage{xspace}
\makeatletter
\def\@textbottom{\vskip \z@ \@plus 1pt}
\let\@texttop\relax
\usepackage{etoolbox}
\patchcmd{\abstract}{\scshape\abstractname}{\textbf{\abstractname}}{}{}

%Switching commands for different section formats
%Mainsectionsytle
\newcommand{\mainsectionstyle}{%
  	\renewcommand{\@secnumfont}{\bfseries}
  	\renewcommand\section{\@startsection{section}{1}%
    	\z@{.5\linespacing\@plus.7\linespacing}{-.5em}%
    	{\normalfont\bfseries}}%
	\renewcommand\subsection{\@startsection{subsection}{2}%
    	\z@{.5\linespacing\@plus.7\linespacing}{-.5em}%
    	{\normalfont\bfseries}}%
	\renewcommand\subsubsection{\@startsection{subsubsection}{3}%
    	\z@{.5\linespacing\@plus.7\linespacing}{-.5em}%
    	{\normalfont\bfseries}}%
}
\newcommand{\originalsectionstyle}{%
\def\@secnumfont{\bfseries}%\mdseries
\def\section{\@startsection{section}{1}%
  \z@{.7\linespacing\@plus\linespacing}{.5\linespacing}%
  {\normalfont\bfseries\centering}}
}
%Formatting title of TOC
\renewcommand{\contentsnamefont}{\bfseries}
%Table of Contents
\setcounter{tocdepth}{3}

% Add bold to \section titles in ToC and remove . after numbers
\renewcommand{\tocsection}[3]{%
  \indentlabel{\@ifnotempty{#2}{\bfseries\ignorespaces#1 #2\quad}}\bfseries#3}
% Remove . after numbers in \subsection
\renewcommand{\tocsubsection}[3]{%
  \indentlabel{\@ifnotempty{#2}{\ignorespaces#1 #2\quad}}#3}
\let\tocsubsubsection\tocsubsection% Update for \subsubsection
%...

\newcommand\@dotsep{4.5}
\def\@tocline#1#2#3#4#5#6#7{\relax
  \ifnum #1>\c@tocdepth % then omit
  \else
    \par \addpenalty\@secpenalty\addvspace{#2}%
    \begingroup \hyphenpenalty\@M
    \@ifempty{#4}{%
      \@tempdima\csname r@tocindent\number#1\endcsname\relax
    }{%
      \@tempdima#4\relax
    }%
    \parindent\z@ \leftskip#3\relax \advance\leftskip\@tempdima\relax
    \rightskip\@pnumwidth plus1em \parfillskip-\@pnumwidth
    #5\leavevmode\hskip-\@tempdima{#6}\nobreak
    \leaders\hbox{$\m@th\mkern \@dotsep mu\hbox{.}\mkern \@dotsep mu$}\hfill
    \nobreak
    \hbox to\@pnumwidth{\@tocpagenum{\ifnum#1=1\bfseries\fi#7}}\par% <-- \bfseries for \section page
    \nobreak
    \endgroup
  \fi}
\AtBeginDocument{%
\expandafter\renewcommand\csname r@tocindent0\endcsname{0pt}
}
\def\l@subsection{\@tocline{2}{0pt}{2.5pc}{5pc}{}}
\def\l@subsubsection{\@tocline{2}{0pt}{4.5pc}{5pc}{}}
\makeatother

\advance\footskip0.4cm
\textheight=54pc    %a4paper
\textheight=50.5pc %letterpaper
\advance\textheight-0.4cm
\calclayout

%Font settings
%\usepackage{anyfontsize}
%Footnote settings
%\usepackage{mathptmx}
\usepackage{footmisc}
%	\renewcommand*{\thefootnote}{\fnsymbol{footnote}}
\usepackage{commath}
%Further math environments
%Further math fonts (loads amsfonts implicitely)
\usepackage{amssymb}
%Redefinition of \text
%\usepackage{amstext}
\usepackage{upref}
%Graphics
%\usepackage{graphicx}
%\usepackage{caption}
%\usepackage{subcaption}
%Frames
\usepackage{mdframed}
\allowdisplaybreaks
%\usepackage{interval}
\newcommand{\toup}{%
  \mathrel{\nonscript\mkern-1.2mu\mkern1.2mu{\uparrow}}%
}
\newcommand{\todown}{%
  \mathrel{\nonscript\mkern-1.2mu\mkern1.2mu{\downarrow}}%
}
\AtBeginDocument{\renewcommand*\d{\mathop{}\!\mathrm{d}}}
\renewcommand{\Re}{\operatorname{Re}}
\renewcommand{\Im}{\operatorname{Im}}
\DeclareMathOperator\Log{Log}
\DeclareMathOperator\Arg{Arg}
\DeclareMathOperator\sech{sech}
\DeclareMathOperator*\esssup{ess.sup}
%\usepackage{hhline}
%\usepackage{booktabs} 
%\usepackage{array}
%\usepackage{xfrac} 
%\everymath{\displaystyle}
%Enumerate
\usepackage{tikz}
\usetikzlibrary{external}
\tikzexternalize % activate!
\usetikzlibrary{patterns}
\pgfdeclarepatternformonly{adjusted lines}{\pgfqpoint{-1pt}{-1pt}}{\pgfqpoint{40pt}{40pt}}{\pgfqpoint{39pt}{39pt}}%
{
  \pgfsetlinewidth{.8pt}
  \pgfpathmoveto{\pgfqpoint{0pt}{0pt}}
  \pgfpathlineto{\pgfqpoint{39.1pt}{39.1pt}}
  \pgfusepath{stroke}
}
\usepackage{enumitem} 
%\renewcommand{\labelitemi}{$\bullet$}
%\renewcommand{\labelitemii}{$\ast$}
%\renewcommand{\labelitemiii}{$\cdot$}
%\renewcommand{\labelitemiv}{$\circ$}
%Colors
%\usepackage{color}
%\usepackage[cmtip, all]{xy}
%Theorems
\newtheoremstyle{bold}              	 %Name
  {}                                     %Space above
  {}                                     %Space below
  {\itshape}		                     %Body font
  {}                                     %Indent amount
  {\scshape}                             %Theorem head font
  {.}                                    %Punctuation after theorem head
  { }                                    %Space after theorem head, ' ', 
  										 %	or \newline
  {} 
\theoremstyle{bold}
\newtheorem*{definition*}{Definition}
\newtheorem{definition}{Definition}[section]
\newtheorem*{lemma*}{Lemma}
\newtheorem{lemma}{Lemma}[section]
\newtheorem{Proof}{Proof}[section]
\newtheorem{proposition}{Proposition}[section]
\newtheorem{properties}{Properties}[section]
\newtheorem{corollary}{Corollary}[section]
\newtheorem*{theorem*}{Theorem}
\newtheorem{theorem}{Theorem}[section]
\newtheorem{example}{Example}[section]
\newtheorem*{remark*}{Remark}
\newtheorem{remark}{Remark}[section]
%German non-ASCII-Characters
%Graphics-Tool
%\usepackage{tikz}
%\usepackage{tikzscale}
%\usepackage{bbm}
%\usepackage{bera}
%Listing-Setup
%Bibliographie
\usepackage[backend=bibtex, style=alphabetic]{biblatex}
%\usepackage[babel, german = swiss]{csquotes}
\bibliography{Bibliography}
%PDF-Linking
%\usepackage[hyphens]{url}
\usepackage[bookmarksopen=true,bookmarksnumbered=true]{hyperref}
%\PassOptionsToPackage{hyphens}{url}\usepackage{hyperref}
\hypersetup{
  colorlinks   = true, %Colours links instead of ugly boxes
  urlcolor     = blue, %Colour for external hyperlinks
  linkcolor    = blue, %Colour of internal links
  citecolor    = blue %Colour of citations
}
%Weierstrass-P symbol for power set
\newcommand{\powerset}{\raisebox{.15\baselineskip}{\Large\ensuremath{\wp}}}


\begin{document}

%Remake Tikz-picture
%\tikzset{external/force remake}

\begin{abstract}
	In this written seminar work I will basically follow the section \emph{Interpolation} in the book \emph{Classical Fourier Analysis, third Edition} by Loukas Grafakos. I will review three basic but important theorems on interpolation of operators on $L^p$ spaces, namely the \emph{Marcinkiewicz Interpolation Theorem}, the \emph{Riesz-Thorin Interpolation Theorem} and finally an extension of the Riesz-Thorin Interpolation Theorem to analytic families of operators (the so-called \emph{Stein's theorem on interpolation of analytic families of operators}). We are mainly concerned with the notion of linear operators as well as slight generalizations of them. 
\end{abstract}

\title{Classical Fourier Analysis: Interpolation of $L^p$ Spaces}
\author{Yannis B\"{a}hni}
\address[Yannis B\"{a}hni]{University of Zurich, R\"{a}mistrasse 71, 8006 Zurich}
\email[Yannis B\"{a}hni]{\href{mailto:yannis.baehni@uzh.ch}{yannis.baehni@uzh.ch}}
\thanks{I would like to thank Dr. Chiara Saffirio for many helpful suggestions, Prof. Dr. Benjamin Schlein for his brilliant Analyis I/II/III courses as well as scripts and proof hints and of course Loukas Grafakos, who helped me a lot with understanding his proofs.}
\maketitle

\tableofcontents
\listoffigures

\mainsectionstyle

\section{Introduction and Basic Definitions}
Suppose $\del{p_0,q_0},\del{ p_1,q_1} \in \intcc{1,\infty} \times \intcc{1,\infty}$ are two pairs of indices and assume that the estimates 

\begin{equation*}
	\norm[0]{T\del{f}}_{L^{q_0}} \leq M_0 \norm[0]{f}_{L^{p_0}} \quad \text{and} \quad \norm[0]{T\del{f}}_{L^{q_1}} \leq M_1\norm{f}_{L^{p_1}}
\end{equation*}

\noindent hold, where $T$ is an appropriately choosen operator. Does this imply that

\begin{equation*}
	\norm[0]{T\del{f}}_{L^q} \leq M \norm{f}_{L^p} \quad \text{for other pairs $\del{p,q} \in \intcc{1,\infty}$?} 
\end{equation*}

Those and similar questions will be answered by a tool called \emph{interpolation}, in our case interpolation of $L^p$ spaces. Using interpolation it is possible to reduce difficult estimates to endpoint estimates and so interpolation can (but not always does) simplify matters. Among the numerous applications of interpolation is by far the shortest proof of \emph{Young's inequality for convolutions} \textup{\cite[22--23]{grafakos:fourier:2014}}. There is not \emph{the} interpolation theorem, merely a family of theorems which can be 
roughly divided into two main categories: \emph{real} and \emph{complex} interpolation methods. Real methods use so called \emph{cut-off} functions to divide the functions in the domain of the operator $T$ into a bounded and unbounded part and then establish bounds on each of those parts whereas complex interpolation theorems are based upon standard results in complex analysis and are more restrictive on the operator $T$ in question but yield more natural bounds (even continuous estimates) and will therefore be considered in this task. First we need a rigorous idea of what an appropriately choosen operator means in the context of Lebesgue spaces. 

\vspace{2mm}

\begin{mdframed}
	\begin{definition}
		Let $\del{X,\mu}$ and $\del{Y,\nu}$ be measure spaces. Further let $T$ be an operator defined on a linear space of complex-valued measurable functions on $X$ and taking values in the set of all complex-valued, finite almost everywhere, measurable functions on $Y$. Then $T$ is called \emph{linear} if for all functions $f$ and $g$ in the domain of $T$ and all $z \in \mathbb{C}$ holds

		\begin{equation}
			T{f + g} = T\del{f} + T\del{g} \qquad T\del{zf} = zT\del{f}
			\label{eq:linear}
		\end{equation}

		\noindent and \emph{quasi-linear} if

		\begin{equation}
			\abs[0]{T \del{f + g}} \leqslant K \del[0]{\abs[0]{ T\del{f}} + \abs[0]{ T\del{g}}} \qquad \abs[0]{ T\del{zf} } = \abs{z}\abs[0]{ T\del{f}}
			\label{eq:quasilinear}
		\end{equation}

		\noindent holds for some real constant $K > 0$. If $K = 1$, $T$ is called \emph{sublinear}.
	\end{definition}
\end{mdframed}

%Complex method
\section{The Complex Method}
This theorem will unfortunately only be applicable to linear operators but will yield quite a natural bound of the operator on the intermediate space. The proof will make strong use of complex variables technique. A major tool will be an application of the maximum modulus principle, known as \emph{Hadamard's three lines lemma}.

\subsection{Hadamard's Three Lines Lemma}
The proof of the Riesz-Thorin interpolation theorem heavily relies on Hadamard's three lines lemma which is itself based on a restatement of the maximum modulus theorem (see \cite[212]{rudin:rc_analysis:1987}) formulated in \cite[253]{rudin:rc_analysis:1987}. To do so, we have first to establish some common terminology. A complex-valued function $f$ is said to be \emph{holomorphic} in $\Omega \subseteq \mathbb{C}$ open, if $f'(z)$ exists for any $z \in \Omega$. By a region we shall mean a nonempty connected open subset of the complex plane. The restatement reads as follows.

\begin{theorem*}
	Let $\Omega \subseteq \mathbb{C}$ be a bounded region and $f$ be a continuous function on $\overline{\Omega}$ which is holomorphic in $\Omega$. Then 

	\begin{equation*}
		\abs[0]{f(z)} \leq \sup\cbr[0]{ \abs[0]{f(z)} : z \in \partial\Omega}
	\end{equation*}

	\noindent for every $z \in \Omega$. If equality holds at one point $z \in \Omega$, then $f$ is constant.
\end{theorem*}

\begin{mdframed}
	\begin{lemma}\emph{(Hadamard's three lines lemma)}
		Let $F$ be a holomorphic function in the strip $S := \cbr[0]{ z \in \mathbb{C}: 0 < \Re z < 1}$, continuous and bounded on $\overline{S}$, such that $\abs[0]{F(z)} \leq B_0$ when $\Re z = 0$ and $\abs[0]{F(z)} \leq B_1$ when $\Re z = 1$, for some $0 < B_0,B_1 < \infty$. Then $\abs[0]{F(z)}\leq B_0^{1 - \theta}B_1^\theta$ when $\Re z = \theta$, for any $0 < \theta < 1$.
		\label{lem:HTL}
	\end{lemma}
\end{mdframed}

\begin{figure}[h!tb]
	\centering
	\begin{tikzpicture}
		\draw [line width = .5mm] (0,-3.5)--(0,3.5);
		\draw [line width = .5mm] (1,-3.5)--(1,3.5);
		\draw [line width = .5mm] (4,-3.5)--(4,3.5);
		\draw (-1,0)--(5,0);
		\fill [pattern = adjusted lines, pattern color = black, line width = .2mm] (0,-3.5) rectangle (4,3.5);
		\node (0) at (-.25,-.25) {$0$};
		\node (1) at (4.25,-.25) {$1$};
		\node (T) at (1.25,-.25) {$\theta$};
		\node (RR) at (4.7,3) {$\mathbb{R} \times \mathbb{R}$};
	\end{tikzpicture}	
	\caption{Sketch of the setting of Hadamard's three lines lemma.}
	\label{fig:Hadamards_three_lines_lemma}
\end{figure}

\begin{proof}
For $z \in \overline{S}$ define 

\begin{equation*}
	G(z) := \frac{F(z)}{B_0^{1 - z}B_1^z} \qquad G_n(z) := G(z) e^{\del[0]{z^2 - 1}/n},~n \in\mathbb{N}_{>0}
\end{equation*}

$G\del{z}$ and $G_n\del{z}$ are holomorphic in $S$ by
	
\begin{equation*}
	G'(z) = \frac{F'(z) - F(z)\log\del{ B_1/B_0}}{B_0^{1 - z}B_1^z} \qquad G_n'(z) = G'(z)e^{\del[0]{z^2 - 1 }/n} + \frac{2}{n}zG_n(z)	
\end{equation*}

\noindent and $e^z \neq 0$ for every $z \in \mathbb{C}$. Further, we have

\begin{equation*}
	\abs[0]{B_0^{1 - z}B_1^z} = \del[0]{B_0^{1 - z}B_0^{1 - \overline{z}} B_1^z B_1^{\overline{z}}}^{1/2} =  B_0^{1 -\Re z}B_1^{\Re z}
\end{equation*}

Consider $0 \leq \Re z \leq 1$ and $B_0 \geq 1$. Then $B_0^{1 - \Re z} \geq 1$ and $B_0^{1 - \Re z } \geq B_0$ in the case $B_0 < 1$. Similarily, $B_1^{\Re z} \geq 1$ if $B_1 \geq 1$ and $B_1^{\Re z} \leq B_1$ if $B_1 < 1$. Hence 

\begin{equation}
	\abs[0]{B_0^{1 - z}B_1^z} \geq \min\cbr[0]{1,B_0}\min\cbr[0]{1,B_1} > 0
	\label{est:denom}
\end{equation}

\noindent for all $z \in \overline{S}$. Since $F$ is bounded on $\overline{S}$, we have $\abs[0]{F(z)} \leq L$ for some $L > 0$ and all $z \in \overline{S}$. Thus by (\ref{est:denom})

\begin{equation*}
	\abs[0]{G(z)} = \frac{\abs[0]{F(z)}}{\abs[0]{B_0^{1 - z}B_1^z}} \leq \frac{L}{\min\cbr[0]{1,B_0}\min\cbr[0]{1,B_1}} =: M
\end{equation*}

\noindent for every $z \in \overline{S}$. Fix $n \in \mathbb{N}_{>0}$ and write $z := x + iy \in \overline{S}$. Then

\begin{gather*}
	\abs[0]{G_n(z)} \leq M \del{e^{\del[0]{x^2 + 2ixy -y^2 - 1}/n} e^{\del[0]{x^2 - 2ixy -y^2 - 1}/n}}^{1/2}= M e^{-y^2/n}e^{\del[0]{x^2 - 1}/n} \leq Me^{-y^2/n}
\end{gather*}

\noindent for $0 \leq x \leq 1$. Thus
	
\begin{equation*}
	\lim_{y \to \pm \infty}\sup\cbr[0]{\abs[0]{G_n(z)} : 0 \leq x \leq 1} = 0
\end{equation*}

\noindent Hence there exist $C_0(n),C_1(n) \in \mathbb{R}$, such that 

\begin{equation*}
	\sup\cbr[0]{\abs[0]{G_n(z)} : 0 \leq x \leq 1} \leq 1
\end{equation*}

\noindent whenever $y > C_0(n)$ or $y < C_1(n)$. Letting

\begin{equation*}
	C(n) := \max\cbr[0]{\abs[0]{ C_0(n)} + 1, \abs[0]{ C_1(n) } + 1}
\end{equation*}

\begin{figure}[h!tb]
	\centering
	\begin{tikzpicture}
		\draw (0,-3.5)--(0,3.5);
		\draw (4,-3.5)--(4,3.5);
		\draw (-1,0)--(5,0);
		\draw [pattern = adjusted lines, pattern color = black, line width = .5mm] (0,-2.5) rectangle (4,2.5);
		\node (0) at (-.25,-.25) {$0$};
		\node (1) at (4.25,-.25) {$1$};
		\node (R) at (.5,2) {$\overline{R}$};
		\node (Cn) at (-.5,2.5) {$C(n)$};
		\node (-Cn) at (-.65,-2.5) {$-C(n)$};
		\node (RR) at (4.7,3) {$\mathbb{R} \times \mathbb{R}$};
	\end{tikzpicture}	
	\caption{Sketch of the rectangle $\overline{R}$.}
	\label{fig:Hadamards_three_lines_lemma_proof}
\end{figure}

\noindent we conclude $\abs[0]{ G_n(z)} \leq 1$ for all $0 \leq x \leq 1$ when $\abs{ y } \geq C(n)$. Now consider the rectangle $R := \intoo{0,1} \times \intoo[0]{-C(n),C(n)}$. We have $\abs[0]{G_n(z)} \leq 1$ on the lines $\intcc{0,1} \times \cbr[0]{\pm C(n)}$. By

\begin{equation*}
	\abs[0]{ G_n(iy)} = \frac{\abs[0]{F(iy)}}{\abs[0]{B_0^{1 - iy} B_1^{iy}}}e^{-\del[0]{y^2 + 1 }/n} \leq 1 \qquad \abs[0]{ G_n(1 + iy)} =	\frac{\abs[0]{ F(1 + iy)}}{\abs[0]{ B_0^{-iy}B_1^{1 + iy}}}e^{-y^2/n} \leq 1
\end{equation*}

\noindent we have $\abs[0]{ G_n(z)} \leq 1$ on the lines $\cbr{0} \times \intcc[0]{-C(n),C(n)}$, $\cbr{1} \times \intcc[0]{-C(n),C(n)}$. Thus $\abs[0]{ G_n(z) } \leq 1$ on $\partial R$. Since $\abs[0]{ G_n(z)}$ is continuous on $\overline{R}$, holomorphic in $R$ and $R$ is a bounded region, the maximum modulus theorem implies

\begin{equation*}
	\abs[0]{ G_n(z)} \leq \sup\cbr[0]{\abs[0]{G_n(z) } : z \in \partial R} \leq 1
\end{equation*}

\noindent for every $z \in R$. Therefore $\abs[0]{ G_n(z)} \leq 1$ on $\overline{R}$ and so $\abs[0]{ G_n(z)}\leq 1$ on $\overline{S}$. Since inequalities are preserved by limits and the modulus is a continuous function, we have that $\abs[0]{ G(z)} = \lim_{n \to \infty} \abs[0]{ G_n(z) } \leq 1$ for $z \in \overline{S}$. We conclude by 

\begin{equation*}
	\abs[0]{ F(\theta + it) } = \abs[0]{ G(\theta + it)} \abs[0]{B_0^{1 - \theta - it}B_1^{\theta + it}} \leq B_0^{1 - \theta} B_1^{\theta}
\end{equation*}

\noindent whenever $0 < \theta < 1$, $t \in \mathbb{R}$.
\end{proof}

\subsection{The Riesz-Thorin Interpolation Theorem}
Here we will answer our question raised in the introduction for the case where $T$ is a linear operator. The theorem as it is in \cite[37]{grafakos:fourier:2014} will be slightly generalized to not necessarily $\sigma$-finite spaces in the manner of \cite[200]{folland:real_analysis:1999}, but not in the most general formulation possible where $T$ is an operator defined on a larger domain. To shorten the argumentation we will introduce some notation. For two measure spaces $\del[0]{ X,\mu }$, $\del[0]{ Y,\nu }$ let $\Sigma_X$ and $\Sigma_Y$ denote the set of all finitely simple functions on $X$, $Y$ respectively. Furthermore recall, that a measure $\mu$ on a measure space $\del{X,\mu}$ is called \emph{semifinite} if for each measurable set $A$ with $\mu(A) = \infty$ there exists a measurable set $B \subseteq A$ with $0 < \mu(B) < \infty$ (see \cite[25]{folland:real_analysis:1999}). 

\vspace{2mm}

\begin{mdframed}
	\begin{theorem}\emph{(Riesz-Thorin interpolation theorem)}
		Suppose that $\del{X,\mu}$, $\del{Y,\nu}$ are measure spaces and $1 \leq p_0,p_1,q_0,q_1 \leq \infty$. If $q_0 = q_1 = \infty$, suppose also that $\nu$ is semifinite. Let $T$ be a linear operator defined on $\Sigma_X$ and taking values in the set of measurable functions on $Y$, such that for some $0 < M_0,M_1 < \infty$ the estimates 

		\begin{equation}
			\norm[0]{T(f)}_{L^{q_0}} \leq M_0\norm[0]{f}_{L^{p_0}} \quad \text{and} \quad \norm[0]{T(f)}_{L^{q_1}} \leq M_1\norm[0]{f}_{L^{p_1}}
			\label{hyp:Lq0Lq1}
		\end{equation}

		\noindent hold for all $f \in \Sigma_X$. Then for all $0 \leq \theta \leq 1$ we have

		\begin{equation}
			\norm[0]{T(f}_{L^q} \leq M_0^{1 - \theta}M_1^\theta\norm[0]{f}_{L^p}
			\label{est:boundTf}
		\end{equation}

		\noindent for all $f \in \Sigma_X$, where

		\begin{equation*}
			\frac{1}{p} = \frac{1 - \theta}{p_0} + \frac{\theta}{p_1} \quad \text{and} \quad \frac{1}{q} = \frac{1 - \theta}{q_0} + \frac{\theta}{q_1}.
		\end{equation*}
		\label{thm:Riesz_Thorin}
	\end{theorem}
\end{mdframed}

\begin{proof}
	The idea is to bound the quantity (see \cite[189]{folland:real_analysis:1999})
	
	\begin{equation*}
		M_q( T(f)) = \sup\cbr{\abs{ \int_Y T(f)g \d \nu} : g \in \Sigma_Y, \norm[0]{ g}_{L^{q'}} = 1}
	\end{equation*}

	\noindent appropriately (recall $q' := q/\del[0]{ q - 1 }$). If either $\theta = 0$ or $\theta = 1$, the estimate (\ref{est:boundTf}) follows directly from the hypotheses (\ref{hyp:Lq0Lq1}) on $T$. Thus we may assume \underline{$0 < \theta < 1$}. Furthermore, if $f \in \Sigma_X$, $\norm{ f}_{L^p} = 0$, then $f = 0$ $\mu$-a.e. and either one of the hypotheses on $T$ in (\ref{hyp:Lq0Lq1}) implies $T(f) = 0$ $\mu$-a.e. and thus the estimate (\ref{est:boundTf}) holds trivially. Therefore we can assume \underline{$\norm{f}_{L^p} \neq 0$}. Fix $f \in \Sigma_X$, $g \in \Sigma_Y$ with representation
	
\begin{equation*}
	f = \sum_{j = 1}^n a_j e^{i\alpha_j}\chi_{A_j} \qquad g = \sum_{k = 1}^m b_k e^{i\beta_k}\chi_{B_k}
\end{equation*}

\noindent where $a_j,b_k \neq 0$, $\alpha_j,\beta_k \in \mathbb{R}$ for any $j = 1,\dots,n$, $k = 1,\dots,m$, the sets $A_j$ and $B_k$ are each pairwise disjoint with $\mu( A_j),\nu(B_k) < \infty$ and so, that $\norm[0]{g}_{L^{q'}} \neq 0$. Define

\begin{equation*}
	P(z) := \frac{p}{p_0}(1 - z) + \frac{p}{p_1}z \qquad Q(z) := \frac{q'}{q'_0}(1 - z) + \frac{q'}{q'_1}z
\end{equation*}

\noindent for $z \in \mathbb{C}$ (since either $p = \infty$ implies $p_0 = p_1 = \infty$ or $q = 1$ implies $q_0 = q_1 = 1$, the functions $P$, $Q$ are well-defined). Further let
				
\begin{equation}
	f_z := \sum_{j = 1}^n a^{P(z)}_j e^{i\alpha_j}\chi_{A_j} \qquad g_z :=  \sum_{k = 1}^m b^{Q(z)}_k e^{i\beta_k}\chi_{B_k}
	\label{eq:def_fzgz}
\end{equation}
				
\noindent and 

\begin{equation}
	F(z) := \int_Y T(f_z)g_z\d\nu
	\label{eq:def_F}
\end{equation}

By (\ref{eq:def_fzgz}), (\ref{eq:def_F}) and the linearity of the operator $T$ we have

\begin{gather*}
	F(z) = \sum_{j = 1}^n\sum_{k = 1}^m a^{P(z)}_j b_k^{Q(z)} e^{i\alpha_j} e^{i\beta_k} \int_YT(\chi_{A_j})\chi_{B_k}\d\nu
\end{gather*}

Applying H\"older's inequality yields

\begin{equation}
	\begin{aligned}
		\abs{ \int_YT(\chi_{A_j})\chi_{B_k}\d\nu} &\leq \int_Y\abs[0]{ T(\chi_{A_j})\chi_{B_k}}\d\nu\\
		&= \norm[0]{T(\chi_{A_j})\chi_{B_k}}_{L^1}\\
		&\leq \norm[0]{T\del[0]{\chi_{A_j}}}_{L^{q_0}} \norm[0]{\chi_{B_k}}_{L^{q_0'}}\\
		&\leq M_0\norm[0]{\chi_{A_j}}_{L^{p_0}} \norm[0]{\chi_{B_k}}_{L^{q_0'}}\\
		&\leq M_0 \mu\del[0]{A_j}^{1/p_0}\nu\del[0]{B_k}^{1/q_0'}
		\label{est:constant_F}		
	\end{aligned}
\end{equation}

\noindent for each $j = 1,\hdots,n$, $k = 1,\hdots,m$ (even in the cases where either $p_0 = \infty$ or $q_0' = \infty$, or both, by observing that $\norm[0]{ \chi_{A}}_{L^\infty} \leq 1$ for any measurable set $A$). Thus the function $F$ is well-defined on $\mathbb{C}$. Let $t \in \mathbb{R}$. For $p,p_0 \neq \infty$

\begin{gather*}
	\begin{aligned}
		\norm[0]{f_{it}}_{L^{p_0}} &= \left(\sum_{j = 1}^n \int_{A_j} \abs[0]{ f_{it}}^{p_0} \d\mu + \int_{X \setminus \bigcup_{j = 1}^n A_j} \abs[0]{ f_{it} }^{p_0} \d\mu\right)^{1/p_0}\\
		&= \del{\sum_{j = 1}^n \abs[0]{ a_j^{P(it)} e^{i\alpha_j}}^{p_0}\int_X \chi_{A_j} \d\mu}^{1/p_0}\\
		&= \del{\sum_{j = 1}^n a_j^{p_0\Re P(it)}\mu(A_j)}^{1/p_0}\\
		&= \del{\sum_{j = 1}^n a_j^p\mu(A_j)}^{p/\left(p_0p\right)}\\
		&= \norm[0]{f}_{L^p}^{p/p_0} 
	\end{aligned}
\end{gather*}

\noindent holds. Let $p_0 = \infty$, $p \neq \infty$. Then $\norm[0]{f_{it}}_{L^{\infty}} = 1$ since $\abs[0]{ a_j^{P(it)}} = a_j^{p/p_0} = 1$ and that there exists some index $j$, such that $\mu( A_j) \neq 0$. If $p = \infty$, then $p_0 = p_1 = \infty$ and thus $P(it) = 1$. By the same considerations we have $\norm[0]{g_{it}}_{L^{q_0'}} = \norm[0]{g}_{L^{q'}}^{q'/q'_0}$. Hence

\begin{gather*}
	\begin{aligned}
		\abs[0]{ F(it)} &\leq \int_Y \abs[0]{ T(f_{it})g_{it}} \d\nu\\
		&= \norm[0]{T(f_{it}) g_{it}}_{L^1}\\
		&\leq \norm[0]{T(f_{it})}_{L^{q_0}}\norm[0]{g_{it}}_{L^{q_0'}}\\
		&\leq M_0 \norm[0]{f_{it}}_{L^{p_0}} \norm[0]{g_{it}}_{L^{q_0'}}\\
		&= M_0 \norm[0]{f}_{L^p}^{p/p_0} \norm[0]g_{L^{q'}}^{q'/q'_0}
	\end{aligned}
\end{gather*}

\noindent by H\"older's inequality. In an analogous manner we derive
				
\begin{equation*}
	\abs[0]{f_{1 + it}}_{L^{p_1}} = \norm[0]{f}_{L^p}^{p/p_1} \qquad \norm[0]{g_{1 + it}}_{L^{q_1'}} = \norm[0]{g}_{L^{q'}}^{q'/q_1'}
\end{equation*}

\noindent and thus 
				
\begin{equation*}
	\abs[0]{ F(1 + it)} \leq M_1 \norm[0]{f}_{L^p}^{p/p_1}\norm[0]{g}_{L^{q'}}^{q'/q_1'}
\end{equation*}	

Further by estimate (\ref{est:constant_F}) 

\begin{equation*}
	\begin{aligned}
		\abs[0]{ F(z)} &\leq \sum_{j = 1}^n\sum_{k = 1}^m \abs[0]{ a_j^{P(z)}} \abs[0]{ b_k^{Q(z)}} \abs{\int_Y T(\chi_{A_j})\chi_{B_k} \d\nu}\\
		&\leq M_0\sum_{j = 1}^n\sum_{k = 1}^m a_j^{\Re P(z)}b_k^{\Re Q(z)}\mu(A_j)^{1/p_0}\nu(B_k)^{1/q_0'}\\
		&\leq M_0\sum_{j = 1}^n\sum_{k = 1}^m \max\cbr[0]{ 1, a_j^{p/p_0 + p/p_1}} \max\cbr[0]{1,b_k^{q'/q_0' + q'/q_1'}}\mu(A_j)^{1/p_0}\nu(B_k)^{1/q_0'}
	\end{aligned}
\end{equation*}


Hence $F$ is bounded on $\overline{S}$ by some constant depending on $f$ and $g$ only. By 

\begin{multline*}
	F'(z) = \sum_{j = 1}^n\sum_{k = 1}^m a^{P(z)}_j\log (a_j) \del{\frac{p}{p_1} - \frac{p}{p_0}} b_k^{Q(z)}e^{i\alpha_j} e^{i\beta_k} \int_YT(\chi_{A_j})\chi_{B_k}\d\nu \\
	+  \sum_{j = 1}^n\sum_{k = 1}^m a^{P(z)}_jb_k^{Q(z)}\log(b_k)\del{ \frac{q'}{q'_1} - \frac{q'}{q'_0}}e^{i\alpha_j} e^{i\beta_k} \int_YT(\chi_{A_j})\chi_{B_k}\d\nu 
\end{multline*}

\noindent it is immediate, that $F$ is an entire function and thus holomorphic in $S$ and continuous on $\overline{S}$. Therefore lemma \ref{lem:HTL} yields

\begin{gather*}
	\abs[0]{ F(z)} \leq \del{ M_0  \norm[0]{f}_{L^p}^{p/p_0} \norm[0]{g}_{L^{q'}}^{q'/q'_0}}^{1 - \theta}\del{ M_1 \norm[0]{f}_{L^p}^{p/p_1}\norm[0]{g}_{L^{q'}}^{q'/q_1'}}^\theta = M_0^{1 - \theta}M_1^\theta \norm[0]{f}_{L^p}\norm[0]{g}_{L^{q'}}
\end{gather*}

\noindent for $\Re z = \theta$, $0 < \theta < 1$. We have

\begin{equation*}
	\cbr[0]{ T(f) \neq 0} = \bigcup_{n = 1}^\infty \cbr[0]{ \abs[0]{ T(f)} > 1/n}
\end{equation*}

\noindent and by Chebychev's inequality (see \cite[193]{folland:real_analysis:1999}) either

\begin{equation*}
	\nu\del[0]{\cbr[0]{\abs[0]{ T(f)} > 1/n}} \leq n^{q_0}\norm[0]{ T(f)}_{L^{q_0}}^{q_0} \leq n^{q_0}M_0^{q_0}\norm[0]{ f}_{L^{p_0}}^{q_0}
\end{equation*}

\noindent or

\begin{equation*}
	\nu\del[0]{\cbr[0]{\abs[0]{ T(f)} > 1/n}} \leq n^{q_1}\norm[0]{ T(f)}_{L^{q_1}}^{q_1} \leq n^{q_1}M_1^{q_1}\norm[0]{ f}_{L^{p_1}}^{q_1}
\end{equation*}

\noindent whenever $q_0 \neq \infty$ or $q_1 \neq \infty$. Therefore, the set $\cbr[0]{ T(f) \neq 0}$ is $\sigma$-finite unless $q_0 = q_1 = \infty$. Further we have $P(\theta) = Q(\theta) = 1$. Thus by

\begin{gather*}
	\begin{aligned}
		M_q( T(f)) &= \sup\cbr{ \abs{\int_Y T(f)g\d\nu} : g \in \Sigma_Y, \norm[0]{g}_{L^{q'}} = 1}\\
		&=  \sup\cbr[0]{\abs[0]{ F(\theta)} : g \in \Sigma_Y, \norm[0]{g}_{L^{q'}} = 1}\\
		&\leq M_0^{1 - \theta}M_1^\theta \norm[0]{f}_{L^p}\\
	\end{aligned}
\end{gather*}

\noindent we conclude 
	
\begin{equation*}
	\norm[0]{ T(f)}_{L^q} = M_q( T(f)) \leq M_0^{1 - \theta}M_1^\theta \norm[0]{f}_{L^p}
\end{equation*}
	
\noindent for any $f \in \Sigma_X$.
\end{proof}

\begin{remark}
	A more general version of the Riesz-Thorin interpolation theorem can be found in \textup{\cite[200--202]{folland:real_analysis:1999}}. There, a linear map $T: L^{p_0} + L^{p_1} \rightarrow L^{q_0} + L^{q_1}$ is considered. This follows using a density argument from the current version of the theorem and will not be prooven here.
	\label{rem:extension}
\end{remark}

\begin{remark}
	Using the previous remark, a standard application of the Riesz-Thorin interpolation theorem is to prove Young's inequality for convolutions \textup{\cite[22--23]{grafakos:fourier:2014}}. Let $G$ be a locally compact group and $\lambda$ be a left invariant Haar measure on $G$, furthermore we assume that $G$ is a countable union of compact subsets, hence the pair $\del{ G,\lambda }$ forms a $\sigma$-finite measure space.

\begin{theorem*}\emph{(Young's inequality)}
	Let $1 \leq p,q,r \leq \infty$ satisfy

	\begin{equation}
		\frac{1}{q} + 1 = \frac{1}{p} + \frac{1}{r}
		\label{hyp:young}
	\end{equation}

	Then for all $f \in L^p(G)$, $g \in L^r( G )$
	satisfying $\norm[0]{g}_{L^r} = \norm[0]{\tilde{g}}$ we have $f \ast g$ exists $\lambda$-a.e. and satisfies

	\begin{equation*}
		\norm[0]{f \ast g}_{L^1} \leq \norm[0]{g}_{L^r} \norm[0]{f}_{L^p}
	\end{equation*}
\end{theorem*}
	
\begin{proof}
	Fix $g \in L^r( G )$ and let $T(f) := f \ast g$ be defined on $L^1( G ) + L^{r'}( G )$. Obviously, $T$ is a linear operator by the linearity of the integral. By Minkowski's integral inequality (see exercise \textbf{1.1.6} \cite[13]{grafakos:fourier:2014}) we get

	\begin{equation*}
		\begin{aligned}
			\norm[0]{T(f)}_{L^r} &= \del{\int_G \abs{ \int_G f(y)g(y^{-1}x) \d\lambda(y)}^r \d\lambda(x)}^{1/r}\\
			&\leq \int_G \del{ \int_G \abs[0]{ f(y) }^r \abs[0]{ g(y^{-1}x) }^r \d\lambda(x)}^{1/r} \d\lambda(y)\\
			&= \int_G \abs[0]{ f(y)} \del{ \int_G \abs[0]{ g(y^{-1}x)}^r \d\lambda(y^{-1}x) }^{1/r} \d\lambda(y)\\
			&= \int_G \abs[0]{ f(y) } \del{ \int_G \abs[0]{ g(z) }^r \d\lambda(z) }^{1/r} \d\lambda(y)\\
			&\leq \norm[0]{f}_{L^1} \norm[0]{g}_{L^r}
		\end{aligned} 
	\end{equation*}

	\noindent for $f \in L^1(g,\mu)$ and $1 \leq p < \infty$. The case $r = \infty$ follows from
	
	\begin{equation*}
		\abs[0]{ (f \ast g)(x) } = \abs{ \int_G f(y)g(y^{-1}x) \d\lambda(y)} \leq \int_G \abs[0]{ f(y)} \abs[0]{g(y^{-1}x)} \d\lambda(y) \leq \norm[0]{g}_{L^\infty}\norm[0]{f}_{L^1}
	\end{equation*}
	
	By stipulating $h(y) := g(y^{-1}x)$ we have 

	\begin{equation*}
		\begin{aligned}
			\abs{(f \ast g)(x)} &= \abs{ \int_G f(y)g(y^{-1}x) \d\lambda(y)} \leq \int_G \abs[0]{ f(y)g(y^{-1}x)} \d\lambda(y)\\
			&= \norm[0]{fh}_{L^1} \leq \norm[0]{f}_{L^{r'}} \norm[0]{h}_{L^r} = \norm[0]{f}_{L^{r'}} \norm[0]{\tilde{g}}_{L^r} = \norm[0]{g}_{L^r} \norm[0]{f}_{L^{r'}}
		\end{aligned}
	\end{equation*}

	\noindent for $r < \infty$ and $f \in L^{r'}( G )$, since

	\begin{equation*}
		\norm[0]{h}^r_{L^r} = \int_G \abs[0]{ g(y^{-1}x)}^r \d\lambda(y) = \int_G \abs[0]{\tilde{g}(x^{-1}y)} \d\lambda(y) = \norm[0]{\tilde{g}}^r_{L^{r}}
	\end{equation*}

	The Riesz-Thorin interpolation theorem now yields for any $0 < \theta < 1$

	\begin{equation}
		\norm[0]{f \ast g}_{L^q} = \norm[0]{T(f)}_{L^q} \leq \norm[0]{g}_{L^r}^{1 - \theta}\norm[0]{g}_{L^r}^{\theta} \norm[0]{f}_{L^p} = \norm[0]{g}_{L^r}\norm[0]{f}_{L^p}
	\end{equation}

	\noindent where 

	\begin{equation*}
		\frac{1}{p} = \frac{1 - \theta}{1} + \frac{\theta}{r'} \qquad \frac{1}{q} = \frac{1 - \theta}{r} + \frac{\theta}{\infty}
	\end{equation*}
\end{proof}
\end{remark}

\begin{remark}
	The proof would be much shorter if we just used Minkowski's inequality \textup{\cite[21--22]{grafakos:fourier:2014}} instead of Minkowski's integral inequality. However, the proof given here is an alternative version of the one given already for Minkowski's inequality.
\end{remark}

%Analytic families
\section{Interpolation of Analytic Families of Operators}
The generalization of the classical Riesz-Thorin interpolation theorem to analytic families of operators is due to \emph{E. M. Stein} and \emph{Guido  Weiss}\footnote{\href{https://projecteuclid.org/euclid.tmj/1178244785}{https://projecteuclid.org/euclid.tmj/1178244785}, last accessed \today.}. Crucial for its proof is again an application of advanced topics in complex analysis.

\subsection{Extension of Hadamard's Three Lines Lemma}
This lemma is inspired by a lemma originally proposed by I.I.Hirschman. I will stick for the most part to the proof given originally in the paper by Stein and Weiss and for some parts to the proof given in \cite[43--45]{grafakos:fourier:2014}.

\subsubsection{Auxiliary Lemmata} To shorten the proof of the extension of Hadamard's three lines lemma, I will summarize the most important facts used during the proof.

\begin{lemma}
	Let $D := \cbr[0]{z \in \mathbb{C} : \abs{z} < 1}$ be the open unit disc and 
	
	\begin{equation*}
		h\del{z} := \frac{1}{\pi i}\log\del{i\frac{1 + z}{1 - z}}
	\end{equation*}

	for $z \in \overline{D} \setminus \{\pm 1\}$ where we are taking that continuous branch of $\log z$ in the complex plane slit along the negative imaginary axis, $\mathbb{C} \setminus \del{\cbr{0} \times \intco{0,\infty}}$. Then $h$ is a holomorphic function in $D$ which maps $\overline{D}\setminus \cbr{\pm 1}$ bijectively onto the closure $\overline{S}$ of the strip $S := \cbr{z \in \mathbb{C} : 0 < \Re z < 1}$.
	\label{lem:h}
\end{lemma}

\begin{proof}
	Consider the mapping $\varphi: \mathbb{S}^2 \to \mathbb{S}^2$ defined by

	\begin{equation*}
		\varphi\del{z} := \frac{1 + z}{1 - z}
	\end{equation*}

	This $\varphi$ maps $\cbr{-1,0,1}$ to $\cbr{0,1,\infty}$ and is a conformal mapping by \cite[278--279]{rudin:rc_analysis:1987}. The segment $\intoo{-1,1}$ maps onto the positive real axis as can be verified by considering the corresponding real limits and the unit circle $\mathbb{S}^1$ passes through $-1$ and $1$, hence $\varphi\del[0]{\mathbb{S}^1}$ is a straight line through $\varphi\del{-1} = 0$. Since $\mathbb{S}^1$ makes a right angle with the real axis at $-1$ so does $\varphi\del[0]{\mathbb{S}^1}$ at $0$ by the conformality of $\varphi$. Thus $\varphi\del[0]{\mathbb{S}^1}$ is the imaginary axis. Since $\varphi\del{0} = 1$, it follows that $\varphi$ is a conformal one-to-one mapping of the open unit disc onto the open right half plane. Furthermore, $\varphi\del[0]{\overline{D}\setminus \cbr{\pm 1}} = \mathbb{H}^\times$ where $\mathbb{H}^\times := \cbr{z \in \mathbb{C} : \Re z \geq 0} \setminus \cbr{0}$. Thus $i\varphi\del[0]{\overline{D}\setminus \cbr{\pm 1}} = i\mathbb{H}^\times = \cbr{z \in \mathbb{C} : \Im z \geq 0} \setminus \cbr{0}$ and so 
	
	\begin{equation*}
		\log i\mathbb{H}^\times = \log \abs[0]{\mathbb{H}^\times} + i \arg i\mathbb{H}^\times = \cbr{z \in \mathbb{C} : 0 \leq \Im z \leq \pi}
	\end{equation*}

	Finally, multiplication with the preceeding factor $1/\del{\pi i}$ yields $h\del[0]{\overline{D} \setminus \cbr{\pm 1}} = \overline{S}$. Furthermore, we have

	\begin{equation*}
		h^{-1}\del{z} = \frac{e^{\pi iz} - i}{e^{\pi i z} + i}
	\end{equation*}
\end{proof}

\begin{lemma*}
	Let $X$ be a topological space. A function $f: X \to \intco[0]{-\infty,\infty}$ is upper semicontinuous if and only if for all $\alpha \in \mathbb{R}$ the set $f^{-1}\del[0]{\intco[0]{-\infty,\alpha}}$ is open. 
\end{lemma*}

\begin{proof}
	Suppose $f: X \to \intco[0]{-\infty,\infty}$ is upper semicontinuous and fix $\alpha \in \mathbb{R}$. We have that 

		\begin{equation*}
			f^{-1}\del[0]{ \intco[0]{-\infty,\alpha}} = \bigcup_{x \in \cbr[0]{ f < \alpha }} U_x
		\end{equation*}

		\noindent where $U_x$ is a neighbourhood of $x$ such that $f < \alpha$ for any element in $U_x$. Conversly, for $x_0 \in X$ and $\alpha > f(x_0)$ we have that $f^{-1}\del[0]{ \intco[0]{-\infty,\alpha }}$ is open and $x_0 \in f^{-1}\del[0]{\intco[0]{-\infty,\alpha}}$.
\end{proof}

\begin{lemma*}
	An upper semiconutinuous function $f: X \to \intco[0]{-\infty,\infty}$ on a compact topological space attains its supremum. In particular it is bounded from above.
	\label{lem:semicom}
\end{lemma*}

\begin{proof}
	 $f(X)$ is bounded from above since otherwise

	\begin{equation*}
		X = \bigcup_{n \in \mathbb{N}} f^{-1}\del[0]{\intco[0]{-\infty,n}}
	\end{equation*}

	\noindent would not have any finite subcover. Therefore $\sup_{x \in X}f(x)$ exists. Further we have $f(x_0) = \sup_{x \in X}f(x)$ for some $x_0 \in X$ since otherwise

	\begin{equation*}
		X = \bigcup_{n \in \mathbb{N}} \textstyle f^{-1}\del[0]{\intco[0]{-\infty,\sup_{x \in X}f(x) - 1/n}}
	\end{equation*}

	\noindent would not have any finite subcover.	
\end{proof}

\begin{lemma}
	Let $\Omega \subseteq \mathbb{C}$ and $f: \Omega \to \mathbb{C}$ continuous. Then $\log\abs[0]{ f}$ is upper semicontinuous on $\Omega$.
	\label{lem:uppersemcont}
\end{lemma}

\begin{proof}
	Let us consider the topological space $\del{ \Omega, \abs[0]{ \cdot}}$. Let $z_0 \in \Omega$ so such that $f(z_0) \neq 0$. Then $\log\abs[0]{ f}$ is continuous as a composition of continuous functions. If $M > f(z_0)$, then $M - \log\abs[0]{ f(z_0)} > 0$ and thus there exists some $\delta > 0$ such that $z \in B_\delta(z_0)$ implies $\abs[0]{ \log\abs[0]{ f(z)} - \log\abs[0]{ f(z_0)}} < M- \log\abs[0]{ f(z_0)}$ or equivalently $\log\abs[0]{ f(z)} < M$. Now let $z_0 \in \Omega$ so such that $f(z_0) = 0$. By convention $\log\abs[0]{ f(z_0)} = -\infty$. Furthermore, $M > \log\abs[0]{ f(z_0)}$ for any $M \in \mathbb{R}$. The condition $M > \log\abs[0]{ f(z)}$ is equivalent to $\abs[0]{ f(z)} < e^M$. But $f(z_0) = 0$ and so

	\begin{equation*}
		\abs[0]{ f(z)} = \abs[0]{ f(z) - f(z_0)}< e^M
	\end{equation*}

	Since $f$ is continuous at $z_0$ and $e^M > 0$ we find $\delta > 0$ such that $z \in B_\delta(z_0)$ implies $\abs[0]{ f(z)} < e^M$.
\end{proof}

\begin{lemma}
	The mapping $\Phi: \mathbb{R} \to \intoo{-\pi,0}$ defined by $\Phi(t) := -i\log( h^{-1}(it))$ is a $C^1$-Diffeomorphism with $\abs[0]{ D\Phi(t)} = \pi \sech(\pi t)$. In an analogous manner we have that $\Psi: \mathbb{R} \to \intoo{0,\pi}$, $\Psi(t) := -i\log( h^{-1}(1 + it) )$ is a $C^1$-Diffeomorphism with $\abs[0]{ D\Psi(t)} = \pi \sech(\pi t)$.
	\label{lem:change_of_variables}
\end{lemma}

\begin{proof}
	It is easier to consider $\Phi^{-1}(\varphi) = -i h(e^{i\varphi})$ and $\Psi^{-1}(\varphi) = i - i h(e^{i\varphi})$ (this already shows that $\Phi$, $\Psi$ are bijective mappings). Let us consider $\varphi$ only since the argumentation for $\Psi$ is similar. By $\abs[0]{ e^{i\varphi}} = 1$ it is immediate that $\Phi^{-1}$ is a real valued function. Furthermore, $\lim_{\varphi \to -\pi} \Phi^{-1}(\varphi) = \infty$, $\lim_{\varphi \rightarrow 0} \Phi^{-1}(\varphi) = -\infty$ and $\Phi^{-1}$ is clearly continuously differentiable. Using
	
	\begin{equation*}
		h^{-1}(it) = \frac{e^{-\pi t} - i}{e^{-\pi t} + i}
	\end{equation*}
	
	\noindent we get

	\begin{gather*}
		\abs[0]{ D\Phi(t)} = \pi\abs{ \frac{e^{-\pi t}}{e^{\pi t} - i} - \frac{e^{-\pi t}}{e^{-\pi t} + i}} = \pi\abs{ \frac{2e^{-\pi t}}{e^{-2\pi t} + 1}} = \pi \abs{ \frac{2}{e^{-\pi t} + e^{\pi t}}}= \pi \sech(\pi t)
	\end{gather*}
\end{proof}

\begin{lemma}
	Let $1/(2e - 1) \leq \rho < 1$ and $\zeta = \rho e^{i\theta}$. Then

	\begin{equation*}
		\abs[3]{ \log \abs{ \frac{1 + \zeta}{1 - \zeta}}} \leq 1 + \log \frac{1}{\abs[0]{ \cos(\theta/2)}} + \log \frac{1}{\abs[0]{ \sin(\theta/2)}}
	\end{equation*}
	\label{lem:upper_bound}
\end{lemma}

\begin{proof}
	This proof is due to Prof. Schlein. We have on the one hand

	\begin{equation*}
		\abs[0]{ 1 + \zeta} \leq 1 + \abs[0]{ \zeta} = 1 + \rho
	\end{equation*}

	\noindent and on the other hand

	\begin{equation*}
		\abs[0]{ 1 - \zeta } \geq \abs[0]{\Im\zeta} = \rho\abs[0]{\sin(\theta)}
	\end{equation*}

	Hence
	
	\begin{gather*}
		\begin{aligned}
			\log \frac{\abs[0]{ 1 + \zeta}}{\abs[0]{ 1 - \zeta }} &\leq \log\frac{1 + \rho}{\rho \abs[0]{ \sin(\theta)}}\\
			&=  \log\frac{1 + \rho}{2\rho \abs[0]{ \sin(\theta/2)}\abs[0]{ \cos(\theta/2)}}\\
			&=  \log\frac{1 + \rho}{2\rho}  +  \log \frac{1}{\abs[0]{ \sin(\theta/2)}} +  \log\frac{1}{\abs[0]{ \cos(\theta/2)}}\\
			&\leq  1 + \log \frac{1}{\abs[0]{ \sin(\theta/2)}} + \log \frac{1}{\abs[0]{ \cos(\theta/2)}}	
		\end{aligned}
	\end{gather*}

	\noindent since

	\begin{equation*}
		\frac{1 + \rho}{2\rho} = \frac{1}{2} + \frac{1}{2\rho} \leq e
	\end{equation*}

	Now by

	\begin{equation*}
		-\log \frac{\abs[0]{ 1 + \zeta }}{\abs[0]{ 1 - \zeta }}  = 	\log \frac{\abs[0]{ 1 - \zeta }}{\abs[0]{ 1 + \zeta }} 
	\end{equation*}

	\noindent which corresponds to considering $-\zeta = e^{i\pi}\zeta = e^{i(\pi + \theta)}$ in the first case, yields by invoking the identities

	\begin{equation*}
		\cos\del{ \frac{\pi + \theta}{2}} = -\sin(\theta/2) \qquad \sin\del{ \frac{\pi + \theta}{2} } = \cos(\theta/2)
	\end{equation*}

	\noindent the bound 

	\begin{equation*}
		-\log \frac{\abs[0]{ 1 + \zeta }}{\abs[0]{ 1 - \zeta }} \leq 1 + \log \frac{1}{\abs[0]{ \sin(\theta/2)}} + \log \frac{1}{\abs[0]{ \cos(\theta/2)}}	 
	\end{equation*}

	\noindent and we are done.
\end{proof}

\begin{lemma}
	Let $0 \leq \tau_0 < \pi$. Then 

	\begin{equation*}
		\frac{1}{\abs[0]{ \cos(\theta/2) }^{\tau_0/\pi}}\frac{1}{\abs[0]{ \sin(\theta/2) }^{\tau_0/\pi}} \in L^1[-\pi,\pi]
	\end{equation*}
\end{lemma}

\begin{proof}
	The case $\tau_0 = 0$ is trivial. We use \cite[153--154]{elstrodt:mass:2011}. By the symmetry of the integrand it is enough to consider 

	\begin{equation*}
		\int_{0}^\pi \frac{1}{\abs[0]{ \cos(\theta/2) }^{\tau_0/\pi}}\frac{1}{\abs[0]{ \sin(\theta/2) }^{\tau_0/\pi}} \d \theta
	\end{equation*}

	Thus we may perform the splitting 
	
	\begin{equation*}
		\int_{0}^1 \frac{1}{\abs[0]{ \cos(\theta/2) }^{\tau_0/\pi}}\frac{1}{\abs[0]{ \sin(\theta/2) }^{\tau_0/\pi}} \d \theta + \int_{1}^\pi \frac{1}{\abs[0]{ \cos(\theta/2) }^{\tau_0/\pi}}\frac{1}{\abs[0]{ \sin(\theta/2) }^{\tau_0/\pi}} \d\theta
	\end{equation*}
	
	Let us consider only the first integral, the second one is similar. Then we have 
	
	\begin{equation*}
		\frac{1}{\abs[0]{ \cos(\theta/2) }^{\tau_0/\pi}} \leq 1 	
	\end{equation*}
	
	\noindent for $0 \leq \theta \leq 1$ and 
	
	\begin{equation*}
		\lim_{\theta \searrow 0} \frac{(\theta/2)^{\tau_0/\pi}}{\sin(\theta/2)^{\tau_0/\pi}} = \lim_{\theta \searrow 0} \frac{(\theta/2)^{\tau_0/\pi}}{\del[2]{\frac{\theta}{2} + O\del{\frac{\theta^3}{8}}}^{\tau_0/\pi}} = \lim_{\theta \searrow 0} \frac{(\theta/2)^{\tau_0/\pi}}{(\theta/2)^{\tau_0/\pi}\del[2]{1 + O\del{\frac{\theta^2}{4}}}^{\tau_0/\pi}} = 1 
	\end{equation*}
	
	Thus 
	
	\begin{equation*}
		\frac{1}{\abs[0]{ \sin(\theta/2) }^{\tau_0/\pi}} \sim\frac{1}{( \theta/2 )^{\tau_0/\pi}}
	\end{equation*}
	
	\noindent when $\theta \searrow 0$ and since $\tau_0 < \pi$, the latter integral converges. Thus we conclude by the comparison theorem.
\end{proof}

\subsubsection{The Lemma}
Now we are ready to prove the extension of the lemma \ref{lem:HTL}.  Recall, that a function $f: X \to \intco{-\infty,\infty}$ defined on a topological space $X$ is said to be \emph{upper semicontinuous} if for every point $x_0 \in X$ and each real $M > f( x_0 )$ a neighbourhood $U$ of $x_0$ exists such that $M > f( x )$ for every $x \in U$ (see \cite[199]{rao:complex_analysis:1991}).

\vspace{2mm}

\begin{mdframed}
	\begin{lemma}\emph{(Hadamard's three lines lemma, extension)}
		Let $F$ be a holomorphic function in the strip $S := \cbr[0]{z \in \mathbb{C}: 0 < \mathrm{Re}z < 1}$ and continuous on $\overline{S}$, such that for some $0 < A < \infty$ and $0 \leq \tau_0 < \pi$ we have $\log \abs[0]{ F( z )} \leq A e^{\tau_0 \abs[0]{ \Im z}}$ for every $z \in \overline{S}$. Then

			\begin{equation*}
				\abs[0]{ F(z) } \leq \exp\del{ \frac{\sin(\pi x)}{2} \int_{-\infty}^\infty \sbr{ \frac{\log \abs[0]{ F(it + iy)}}{\cosh(\pi t) - \cos(\pi x)} + \frac{\log \abs[0]{ F(1 + it + iy)}}{\cosh(\pi t) + \cos(\pi x)} } \d t}
			\end{equation*}

			\noindent whenever $z := x + iy \in S$.
			\label{lem:EHTL}
	\end{lemma}
\end{mdframed}

\begin{proof}
	We will first prove the case \underline{$y = 0$.} Assume $F$ to be not identically zero (the case where $F$ is identically zero is trivial). Let $h$ be as in lemma (\ref{lem:h}) and let $\zeta := \rho e^{i\theta}$, $0 \leq \rho < 1$. Since $\zeta \in D$, we have $0 < \Re h(\zeta) < 1$ and thus the hypothesis on $F$ and lemma (\ref{lem:upper_bound}) yields

\begin{gather}
	\log \abs[0]{ F(h(\zeta))} \leq Ae^{\frac{\tau_0}{\pi}\abs[0]{ \log \abs[0]{\del[0]{1 + \zeta}/\del[0]{1 - \zeta}}}} \leq Ae^{\tau_0/\pi}\frac{1}{\abs[0]{ \cos(\theta/2) }^{\tau_0/\pi}}\frac{1}{\abs{ \sin(\theta/2) }^{\tau_0/\pi}}
	\label{est:admissible}
\end{gather}

\noindent for $1/(2e - 1) \leq \rho$. Since $0 < \tau_0 < \pi$, inequality (\ref{est:admissible}) asserts, that $\log \abs[0]{ F(h(\zeta))}$ is bounded from above by an integrable function of $\theta$, independently of $\rho \geq 1/(2e - 1)$. Furthermore we have 

	\begin{equation}
		M := \sup\cbr[1]{\log \abs[0]{ F(h(\zeta))} : \zeta \in \overline{B}_{1/\del[0]{2e - 1 }}} < \infty
	\end{equation}

	\noindent since a upper semicontinuous function on a compact space attains its supremum (see lemma \ref{lem:semicom}). Hence

	\begin{equation}
		\log\abs[0]{ F( h( \rho e^{i\theta} ))} \leq \max\cbr[4]{ M,   Ae^{\tau_0/\pi}\frac{1}{\abs[0]{ \cos(\theta/2) }^{\tau_0/\pi}}\frac{1}{\abs[0]{ \sin(\theta/2) }^{\tau_0/\pi}}} =: g( \theta )
		\label{est:log}
	\end{equation}

	\noindent for any $0 \leq \rho < 1$ where $g \in L^1\intcc{-\pi,\pi}$. Let $0 \leq \rho < R < 1$ and $a_1,\dots,a_n$ denote the zeros of $F(h(\zeta))$ for $\abs[0]{ \zeta} < R$ (since $F \circ h$ is holomorphic for $\abs[0]{ \zeta} < 1$ there are indeed only finitely many ones) multiple zeros being repeated. Then for $F(h(\zeta)) \neq 0$ we have by the \emph{Poisson-Jensen formula} (see \cite[208]{ahlfors:complex_analysis:1979})

	\begin{equation}
		\log\abs[0]{ F(h(\zeta))} = - \sum_{k = 1}^n \log\abs[3]{ \frac{R^2 - \overline{a}_k\zeta}{R\del[0]{ \zeta - a_k }}}	+ \frac{1}{2\pi}\int_{-\pi}^\pi \Re \sbr[3]{\frac{R e^{it} + \zeta}{R e^{it} - \zeta}}\log\abs[0]{ F(h(R e^{it}))}\d t
	\end{equation}

	Therefore by

	\begin{equation*}
		\begin{aligned}
			\Re \sbr[3]{\frac{R e^{it} + \zeta}{R e^{it} - \zeta}} &= \Re \sbr[3]{\frac{R^2 - 2i \Im \sbr[0]{ \zeta R e^{-it} } - \abs[0]{ \zeta }^2}{R^2 - 2\Re \sbr[0]{ \zeta R e^{-it} } + \abs[0]{ \zeta }^2}}\\ &= \Re \sbr[3]{\frac{R^2 - 2iR\rho \sin( \theta - t ) - \rho^2 }{R^2 - 2R\rho \cos( \theta - t ) + \rho^2}}\\
			&= \frac{R^2 - \rho^2}{R^2 - 2R\rho \cos( \theta - t ) + \rho^2}
		\end{aligned}
	\end{equation*}

	\noindent and since $\del[0]{ R^2 - \abs[0]{ a_k }^2 }\del[0]{ R^2 - \rho^2 }\geq 0$ for all $k = 1,\dots,n$ implies $\abs[0]{ R^2 - \overline{a}_k\zeta } \geq \abs[0]{ R\del[0]{ \zeta - a_k}}$ the estimate 

	\begin{equation}
		\log\abs[0]{ F(h(\zeta))} \leq \frac{1}{2\pi}\int_{-\pi}^\pi \frac{R^2 - \rho^2}{R^2 - 2R\rho \cos( \theta - t ) + \rho^2}\log\abs[0]{ F(h(R e^{it}))}\d t
	\end{equation}

	\noindent is valid for every $\abs[0]{ \zeta } < R$. By \cite[236]{rudin:rc_analysis:1987} we have

	\begin{equation}
		\frac{R - \rho}{R + \rho} \leq \frac{R^2 - \rho^2}{R^2 - 2R\rho\cos(\theta - \varphi) + \rho^2} \leq \frac{R + \rho}{R - \rho}
		\label{est:poisson}
	\end{equation}
	
	\noindent for $0 \leq \rho < R$. Combining (\ref{est:log}) and (\ref{est:poisson}) yields 

\begin{equation}
	\frac{R^2 - \rho^2}{R^2 - 2R\rho \cos( \theta - t ) + \rho^2}\log\abs[0]{ F(h(\zeta))} \leq \frac{R + \rho}{R - \rho}g( \theta ) =: G( \theta )
\end{equation}

\noindent where $G \in L^1\intcc{-\pi,\pi}$. For $0 < R < 1$ let

\begin{equation*}
	f_R(\varphi) := \frac{R^2 - \rho^2}{R^2 - 2R\rho\cos(\theta - \varphi) + \rho^2}\log\abs[0]{ F(h(Re^{i\varphi}))}
\end{equation*}

\noindent and for $\varphi \notin \cbr[0]{ 0,\pi}$

\begin{equation*}
	f(\varphi) := \frac{1 - \rho^2}{1 - 2\rho\cos(\theta - \varphi) + \rho^2}\log\abs[0]{ F(h(e^{i\varphi}))}
\end{equation*}

Since $\log\abs[0]{ F(h(\zeta))}$ is upper semicontinuous on $\overline{D} \setminus \cbr[0]{ \pm 1}$ by lemma \ref{lem:uppersemcont} we get

\begin{equation*}
	\begin{aligned}
		\limsup_{R \nearrow 1}f_R(\varphi) &= \limsup_{R \nearrow 1} \sbr[3]{ \frac{R^2 - \rho^2}{R^2 - 2R\rho\cos(\theta - \varphi) + \rho^2}\log\abs[0]{ F(h(Re^{i\varphi}))}}\\
		&= \lim_{R \nearrow 1} \frac{R^2 - \rho^2}{R^2 - 2R\rho\cos(\theta - \varphi) + \rho^2}\limsup_{R \nearrow 1} \log\abs[0]{ F(h(Re^{i\varphi}))}\\
		&= \frac{1 - \rho^2}{1 - 2\rho\cos(\theta - \varphi) + \rho^2}\log\abs[0]{ F(h(e^{i\varphi}))} = f(\varphi)
	\end{aligned}
\end{equation*}

\noindent using \cite[363]{bourbaki:general_topology:1995} and proposition \ref{prop:limsup}. The functions $G - f_R$ being non-negative, an application of Fatou's lemma yields

\begin{equation*}
	\int_{-\pi}^\pi \liminf_{R \nearrow 1}\sbr[0]{ G(\varphi) - f_R(\varphi)}\d\varphi \leq \liminf_{R \nearrow 1} \int_{-\pi}^\pi \sbr[0]{ G(\varphi) - f_R(\varphi)}\d\varphi
\end{equation*}

By \cite[354]{bourbaki:general_topology:1995}, we get

\begin{equation*}
	\limsup_{R \nearrow 1} \int_{-\pi}^\pi \sbr[0]{ f_R(\varphi) - G(\varphi)}\d \varphi \leq \int_{-\pi}^\pi \limsup_{R \nearrow 1}\sbr[0]{ f_R(\varphi) - G(\varphi)}\d\varphi
\end{equation*}

\noindent and thus

\begin{equation*}
\begin{aligned}
	\limsup_{R \nearrow 1} \int_{-\pi}^\pi f_R(\varphi) \d\varphi - \int_{-\pi}^\pi G(\varphi)\d\varphi &= \limsup_{R \nearrow 1} \int_{-\pi}^\pi f_R(\varphi) \d\varphi - \lim_{R \nearrow 1}\int_{-\pi}^\pi G(\varphi)\d\varphi\\
	&= \limsup_{R \nearrow 1}\int_{-\pi}^\pi \sbr[0]{ f_R(\varphi) - G(\varphi)}\d\varphi\\
	&\leq \int_{-\pi}^\pi \limsup_{R \nearrow 1} \sbr[0]{f_R(\varphi) - G(\varphi)}\d\varphi\\
	&\leq \int_{-\pi}^\pi \limsup_{R \nearrow 1} f_R(\varphi) \d\varphi - \int_{-\pi}^\pi\lim_{R \nearrow 1} G(\varphi)\d\varphi\\
	&= \int_{-\pi}^\pi \limsup_{R \nearrow 1}f_R(\varphi) \d\varphi - \int_{-\pi}^\pi G(\varphi)\d\varphi
\end{aligned}
\end{equation*}

\noindent by \cite[358]{bourbaki:general_topology:1995}. Hence

\begin{equation*}
	\limsup_{R \nearrow 1} \int_{-\pi}^\pi f_R(\varphi) \d\varphi \leq\int_{-\pi}^\pi \limsup_{R \nearrow 1}f_R(\varphi) \d\varphi
\end{equation*}

\noindent and so 

\begin{equation}
	\log\abs[0]{ F( h( \zeta ) )} \leq \frac{1}{2\pi} \int_{-\pi}^\pi \frac{1 - \rho^2}{1 - 2\rho\cos( \theta - \varphi ) + \rho^2}\log\abs[0]{ F( h( e^{i\varphi}))}\d\varphi	
	\label{eq:limit_case}
\end{equation}

The lemma now follows from (\ref{eq:limit_case}) by a change of variables. By stipulating $x := h( \zeta )$ we obtain 

\begin{multline}
	\zeta = h^{-1}(x) = \frac{e^{\pi i x}- i}{e^{\pi i x} + i} = \frac{\cos(\pi x) + i\sin(\pi x) - i}{\cos(\pi x) + i\sin(\pi x) + i}\\
		= \frac{\cos(\pi x) + i\sin(\pi x) - i}{\cos(\pi x) + i\sin(\pi x) + i}\frac{\cos(\pi x) - i\sin(\pi x) - i}{\cos(\pi x) - i\sin(\pi x) - i}\\
		= -i \frac{\cos(\pi x)}{1 + \sin(\pi x)} = \del[3]{ \frac{\cos(\pi x)}{1 + \sin(\pi x)} }e^{-i\pi/2}
	\label{eq:radius_angle}
\end{multline}

\noindent by

\begin{multline*}
	\del[0]{\cos(\pi x) + i\sin(\pi x) - i}\del[0]{\cos(\pi x) - i\sin(\pi x) - i}\\ = \cos^2(\pi x) - i\sin(\pi x)\cos(\pi x)	
		-i\cos(\pi x) + i\sin(\pi x)\cos(\pi x)\\
	 	+ \sin^2(\pi x) + \sin(\pi x) - i \cos(\pi x) - \sin(\pi x) - 1 = -2i \cos(\pi x)  
\end{multline*}

\noindent and
	
\begin{multline*}
	\del[0]{ \cos(\pi x) + i\sin(\pi x) + i }\del[0]{ \cos(\pi x) - i\sin(\pi x) - i }\\ = \cos^2(\pi x) - i\sin(\pi x)\cos(\pi x) - i\cos(\pi x) + i\sin(\pi x)\cos(\pi x)\\
	+ \sin^2(\pi x) + \sin(\pi x)+ i \cos(\pi x) + \sin(\pi x) + 1 = 2 + 2\sin(\pi x)
\end{multline*}

	From equality (\ref{eq:radius_angle}) we deduce $\rho = \frac{\cos(\pi x)}{1 + \sin(\pi x)}$, $\theta = -\frac{\pi}{2}$ if $0 < x \leq \frac{1}{2}$ and $\rho = -\frac{\cos(\pi x)}{1 + \sin(\pi x)}$, $\theta = \frac{\pi}{2}$ if $\frac{1}{2} \leq x < 1$. Let $0 < x \leq \frac{1}{2}$. Then we have

\begin{multline*}
	\frac{1 - \rho^2}{1 - 2\rho\cos(\theta - \varphi) + \rho^2} \\= \frac{1 + 2\sin(\pi x) + \sin^2(\pi x) - \cos^2(\pi x)}{1 + 2\sin(\pi x) + \sin^2(\pi x) + 2\cos(\pi x)\sin(\varphi)(1 + \sin(\pi x)) + \cos^2(\pi x)}\\
	= \frac{\sin(\pi x) + \sin^2(\pi x)}{1 + \sin(\pi x) + \cos(\pi x)\sin(\varphi)(1 + \sin(\pi x))}
	= \frac{\sin(\pi x)}{1 + \cos(\pi x)\sin(\varphi)}
\end{multline*}

\noindent and also for $\frac{1}{2} \leq x < 1$. Let $\Phi$ and $\Psi$ be defined as in lemma (\ref{lem:change_of_variables}). We have 
				
\begin{gather*}
	\begin{aligned}
		e^{i\Phi(t)} &= h^{-1}(it) = \frac{e^{-\pi t} - i}{e^{-\pi t} + i} \frac{e^{-\pi t} - i}{e^{-\pi t} - i} = \frac{e^{-2\pi t} - 2ie^{-\pi t} - 1}{e^{-2\pi t} + 1} = \frac{e^{-2\pi t} - 1}{e^{-2\pi t} + 1} - \frac{2ie^{-\pi t}}{e^{-2\pi t} + 1}\\ 
		&= \frac{e^{-2\pi t} - 1}{e^{-2\pi t} + 1} - \frac{2i}{e^{-\pi t} + e^{\pi t}} = \frac{1 - e^{2\pi t}}{1 + e^{2\pi t}} - \frac{2i}{e^{-\pi t} + e^{\pi t}} = -\tanh (\pi t) - i \sech(\pi t)
	\end{aligned}
\end{gather*}
	
\noindent and thus

\begin{gather*}
	\begin{aligned}
		\sin( \Phi(t) )\cosh(\pi t) &= \sin( -i\log( -\tanh (\pi t) - i \sech(\pi t) )) \cosh(\pi t)\\
		&= \frac{1}{2i} \sbr[3]{ -\tanh(\pi t) - i\sech(\pi t) + \frac{1}{\tanh(\pi t) + i\sech(\pi t) }}\cosh(\pi t)\\
		&= \frac{1}{2i} \sbr[3]{ \frac{\cosh(\pi t) - \tanh(\pi t) \sinh(\pi t) - 2i\tanh(\pi t) + \sech(\pi t)}{\tanh(\pi t) + i\sech(\pi t)}}\\
		&= \frac{1}{2i} \sbr[3]{ \frac{\cosh^2(\pi t) - \sinh^2(\pi t) - 2i \sinh(\pi t) + 1}{\sinh(\pi t) + i}}\\
		&= \frac{1 - i\sinh(\pi t)}{i\sinh(\pi t) - 1} = -1
	\end{aligned}
\end{gather*}

Therefore the transformation formula yields

\begin{equation*}
	\frac{1}{2\pi} \int_{-\pi}^0 \frac{\sin(\pi x)}{1 + \cos(\pi x)\sin(\varphi)} \log \abs[0]{ F(h(e^{i\varphi}))} \d\varphi = \frac{1}{2}\int_{-\infty}^\infty\frac{\sin(\pi x)}{\cosh(\pi t) - \cos(\pi x)} \log\abs[0]{ F(it)} \d t
\end{equation*}
				
\noindent and in a similar manner
		
\begin{equation*}
	\frac{1}{2\pi} \int_0^\pi \frac{\sin(\pi x)}{1 + \cos(\pi x)\sin(\varphi)} \log \abs[0]{ F(h(e^{i\varphi}))} \d\varphi = \frac{1}{2}\int_{-\infty}^\infty\frac{\sin(\pi x)}{\cosh(\pi t) + \cos(\pi x)} \log\abs[0]{ F(1 + it) } \d t
\end{equation*}

\noindent holds since

\begin{gather*}
	\begin{aligned}
		\sin(\Psi(t))\cosh(\pi t) &= \sin( -i\log( -\tanh (\pi t) + i \sech(\pi t) )) \cosh(\pi t)\\
		&= \frac{1}{2i} \sbr[3]{ -\tanh(\pi t) + i\sech(\pi t) - \frac{1}{-\tanh(\pi t) + i\sech(\pi t) }}\cosh(\pi t)\\
		&= \frac{1}{2i} \sbr[3]{ \frac{-\cosh(\pi t) + \tanh(\pi t) \sinh(\pi t) - 2i\tanh(\pi t) - \sech(\pi t)}{-\tanh(\pi t) + i\sech(\pi t)}}\\
		&= \frac{1}{2i} \sbr[3]{ \frac{- \cosh^2(\pi t) + \sinh^2(\pi t) - 2i \sinh(\pi t) - 1}{i - \sinh(\pi t)}}\\
		&= \frac{1 + i\sinh(\pi t)}{1 + i\sinh(\pi t)} = 1
	\end{aligned}
\end{gather*}

Thus the case $y = 0$ is prooven.\\
The case \underline{$y \neq 0$.} follows easily from the previous one. Fix $y \neq 0$ and define $G(z) := F(z + iy)$ for $z \in \overline{S}$. Then $G$ is a holomorphic function in $S$ and continuous on $\overline{S}$ as a composition of continuous and holomorphic functions. Moreover, the hypothesis on $F$ yields

		\begin{equation}
			\log \abs[0]{ G(z) } = \log \abs[0]{ F(z + iy) } \leq Ae^{\tau_0 \abs[0]{ \Im z + y}} \leq Ae^{\tau_0 \abs[0]{ \Im z }}e^{\tau_0 \abs[0]{ y }}
		\end{equation}

		for all $z \in \overline{S}$. The previous case yields for $G$ with $A$ replaced by $Ae^{\tau_0\abs[0]{ y }}$

		\begin{equation}
			\abs[0]{ G(x) } \leq \exp\del[3]{ \frac{\sin(\pi x)}{2} \int_{-\infty}^\infty \sbr{ \frac{\log \abs[0]{ G(it)}}{\cosh(\pi t) - \cos(\pi x)} + \frac{\log \abs[0]{ G(1 + it)}}{\cosh(\pi t) + \cos(\pi x)} } \d t}
		\end{equation}

		Now, observing $G(x) = F(x + iy)$, $G(it) = F(it + iy)$ and $G(1 + it) = F(1 + it + iy)$ yields the desired result.
\end{proof}

\begin{remark}
	Exercise \textbf{1.3.8.} \textup{\cite[48]{grafakos:fourier:2014}} shows that the name extension is an appropriate choice. For $0 < x < 1$ consider

	\begin{equation*}
		\begin{aligned}
			\frac{\sin( \pi x )}{2} \int_{-\infty}^\infty \frac{1}{\cosh(\pi t ) + \cos( \pi x )} \d t &= \frac{\sin( \pi x )}{2} \int_{-\infty}^\infty \frac{1}{\frac{1}{2}( e^{\pi t} + e^{-\pi t} ) + \cos( \pi x )} \d t\\
			&= \frac{\sin( \pi x )}{\pi} \int_{0}^\infty \frac{1}{s^2 + 2\cos( \pi x )s + 1} \d s\\
			&= \frac{\sin( \pi x )}{\pi} \int_{0}^\infty \frac{1}{( s + \cos( \pi x ) )^2 + \sin^2( \pi x )} \d s\\
			&= \frac{1}{\pi\sin( \pi x )} \int_{0}^\infty \frac{1}{\del{ \frac{s + \cos( \pi x )}{\sin( \pi x )}}^2 + 1} \d s\\
			&= \frac{1}{\pi} \int_{\cot( \pi x )}^\infty \frac{1}{u^2 + 1} \d u\\
			&= \frac{1}{\pi}\sbr{ \frac{\pi}{2} - \arctan( \cot( \pi x )) }\\
			&= x
		\end{aligned}
	\end{equation*}

	\noindent and in the same manner

	\begin{equation*}
		\frac{\sin( \pi x )}{2} \int_{-\infty}^\infty \frac{1}{\cosh( \pi t ) - \cos( \pi x )} \d t = 1 - x	
	\end{equation*}

	Assume that $F$ is holomorphic in $S$, continuous and bounded on $\overline{S}$ with $\abs[0]{F( z )} \leq B_0$ when $\Re z = 0$ and $\abs[0]{F( z )} \leq B_1$ when $\Re z = 1$ for some $0 < B_0, B_1 < \infty$. If $\abs[0]{F( z )} \leq M$ for $0 < M < \infty$, $F$ satisfies the hypothesis of lemma \ref{lem:EHTL} with $A := \log( M )$ and $\tau_0 = 0$. Therefore 
	
	\begin{equation*}
		\begin{aligned}
			\abs[0]{ F(z) } &\leq \exp\del[3]{ \frac{\sin(\pi x)}{2} \int_{-\infty}^\infty \sbr{ \frac{\log \abs[0]{ F(it + iy)}}{\cosh(\pi t) - \cos(\pi x)} + \frac{\log \abs[0]{ F(1 + it + iy)}}{\cosh(\pi t) + \cos(\pi x)} } \d t}\\
			&\leq \exp\del[3]{ \frac{\sin(\pi x)}{2} \int_{-\infty}^\infty \sbr{ \frac{\log B_0}{\cosh(\pi t) - \cos(\pi x)} + \frac{\log B_1}{\cosh(\pi t) + \cos(\pi x)} } \d t}\\
		&= \exp( x\log B_0 + ( 1 - x)\log B_1 )\\
		&= B_0^xB_1^{1 - x}
		\end{aligned}
	\end{equation*}

	\noindent whenever $z := x + iy \in S$. Hence lemma \ref{lem:EHTL} reduces to lemma \ref{lem:HTL}.
\end{remark}

\subsection{Stein-Weiss Interpolation Theorem of Analytic Families of Operators}
Because of the complex nature of its proof, the Riesz-Thorin theorem \ref{thm:Riesz_Thorin} can be extended to appropriate families of linear operators $(T_z)_{z \in \Omega}$ depending on a parameter $z \in \Omega \subseteq \mathbb{C}$. This result is due to Elias M. Stein and Guido Weiss. First we need to establish some common terminology.  

\vspace{2mm}

\begin{mdframed}
	\begin{definition}\emph{(Analytic family, admissible growth)}
		Let $\del{X,\mu}$, $\del{Y,\nu}$ be two $\sigma$-finite measure spaces and $( T_z )_{z \in \overline{S}}$, where $T_z$ is defined $\Sigma_X$ and taking values in the space of all measurable functions on $Y$ such that

		\begin{equation}
			\int_Y \abs[0]{ T_z(\chi_A)\chi_B} \d\nu
		\end{equation}

		\noindent whenever $\mu(A),\nu(B) < \infty$. The family $( T_z )_{z \in \overline{S}}$ is said to be \emph{analytic} if for all $f \in \Sigma_X$, $g \in \Sigma_Y$ we have that

		\begin{equation}
			z \mapsto \int_Y T_z(f)g\d\nu
		\end{equation}

		\noindent is analytic on $S$ and continuous on $\overline{S}$. Further, an analytic family $( T_z )_{z \in \overline{S}}$ is called of \emph{admissible growth}, if there is a constant $\tau_0 \in \intco{0,\pi}$, such that for all $f \in \Sigma_X$, $g \in \Sigma_Y$ a constant $C(f,g)$ exists with

			\begin{equation}
				\log\abs[3]{ \int_Y T_z(f) g \d\nu} \leq C(f,g)e^{\tau_0\abs[0]{ \Im z}}
			\end{equation}

			\noindent for all $z \in \overline{S}$.
	\end{definition}
\end{mdframed}

\vspace{2mm}

Now we are able to formulate the theorem.

\vspace{2mm}

\begin{mdframed}
	\begin{theorem}\emph{(Stein-Weiss interpolation theorem of Analytic Families of Operators)}
		Let $( T_z )_{z \in \overline{S}}$ be an analytic family of admissible growth, $1 \leq p_0,p_1,q_0,q_1 \leq \infty$ and suppose that $M_0$, $M_1$ are positive functions on the real line such that for some $\tau_1 \in \intco{0,\pi}$

			\begin{equation}
				\sup_{-\infty < y <\infty} e^{-\tau_1 \abs[0]{ y }} \log M_0(y) < \infty \qquad \text{and} \qquad \sup_{-\infty < y < \infty} e^{-\tau_1 \abs[0]{ y }} \log M_1(y) < \infty.
				\label{eq:growth_conditions}
			\end{equation}

			Fix $0 < \theta < 1$ and define

			\begin{equation}
				\frac{1}{p} := \frac{1 - \theta}{p_0} + \frac{\theta}{p_1} \qquad \text{and} \qquad \frac{1}{q} := \frac{1 - \theta}{q_0} + \frac{\theta}{q_1}.
			\end{equation}

			Further suppose that for all $f \in \Sigma_X$ and $y \in \mathbb{R}$ we have

			\begin{equation}
				\norm[0]{T_{iy}(f)}_{L^{q_0}} \leq M_0(y)\norm[0]{f}_{L^{p_0}} \qquad \text{and} \qquad \norm[0]{T_{1 + iy}(f)}_{L^{q_1}} \leq M_1(y)\norm[0]{f}_{L^{p_1}}.
			\end{equation}

			Then for all $f \in \Sigma_X$ we have

			\begin{equation*}
				\norm[0]{T_\theta(f)}_{L^q} \leq M(\theta)\norm[0]{f}_{L^p}
			\end{equation*}

			\noindent where for $0 < x < 1$

			\begin{equation*}
				M(x) = \exp\del[3]{ \frac{\sin(\pi x)}{2} \int_{-\infty}^\infty \sbr[3]{ \frac{\log M_0(t)}{\cosh(\pi t) - \cos(\pi x)} + \frac{\log M_1(t)}{\cosh(\pi t) + \cos(\pi x)}} \d t }.
			\end{equation*}
	\end{theorem}
\end{mdframed}

\begin{proof}
	Fix $0 < \theta < 1$ and $f \in \Sigma_X$, $g \in \Sigma_Y$ with $\norm[0]{f}_{L^p} = \norm[0]{g}_{L^{q'}} = 1$. Define $f_z$, $g_z$ as in (\ref{eq:def_fzgz}) and for $z \in \overline{S}$

	\begin{equation*}
		F(z) := \int_Y T_z(f_z)g_z \d\nu	
	\end{equation*}

	Since the family $(T_z)_{z \in \overline{S}}$ is of admissible growth we have that there exist constants $c(\chi_{A_j},\chi_{B_k})$ for any $j = 1,\dots,n$ and $k = 1,\dots,m$ such that 

	\begin{equation*}
		\log\abs[3]{\int_{B_k}T_z\del[0]{\chi_{A_j}}\d \nu} \leq c\del[0]{\chi_{A_j},\chi_{B_k}}e^{\tau_0\abs{\Im z}}
	\end{equation*}

	For shortness we will denote these constants simply by $c(A_j,B_k)$ and get

	\begin{gather*}
		\begin{aligned}
			\log \abs[0]{ F(z) } &= \log \abs[4]{ \sum_{j = 1}^n\sum_{k = 1}^m a^{P(z)}_j b_j^{Q(z)} e^{i\alpha_j} e^{i\beta_k} \int_YT_z(\chi_{A_j})(y)\chi_{B_k}(y)\d\nu(y)}\\
			&\leq \log \sbr[4]{ \sum_{j = 1}^n\sum_{k = 1}^m \max\cbr[0]{1,a_j^{p/p_0 + p/p_1}}\max\cbr[0]{1,b_k^{q'/q'_0 + q'/q'_1}} \abs[3]{\int_{B_k} T_z(\chi_{A_j}) \d\nu}}\\
			&\leq \log\sbr[4]{ \sum_{j = 1}^n\sum_{k = 1}^m \del[0]{ 1 + a_j}^{p/p_0 + p/p_1} \del[0]{1 + b_k}^{q'/q'_0 + q'/q'_1} e^{c(A_j,B_k)e^{\tau_0 \abs[0]{ \Im z}}} }\\
			&\leq  \log\sbr[4]{ \sum_{j = 1}^n\sum_{k = 1}^m e^{\log\del[0]{\del[0]{ 1 + a_j}^{p/p_0 + p/p_1}\del[0]{ 1 + b_k}^{q'/q'_0 + q'/q'_1}} + c(A_j,B_k)e^{\tau_0 \abs[0]{ \Im z}}} }\\
			&\leq \log\del[1]{ mn e^{\sum_{j = 1}^n\sum_{k = 1}^m\log\del[0]{\del[0]{ 1 + a_j}^{p/p_0 + p/p_1} \del[0]{1 + b_k}^{q'/q'_0 + q'/q'_1}} + c(A_j,B_k)e^{\tau_0 \abs[0]{ \Im z}}} }\\
			&= \log(mn) + \sum_{j = 1}^n\sum_{k = 1}^m\log\del[0]{\del[0]{1 + a_j}^{p/p_0 + p/p_1}\del[0]{1 +  b_k}^{q'/q'_0 + q'/q'_1}} + c(A_j,B_k)e^{\tau_0 \abs[0]{ \Im z}}
		\end{aligned}
	\end{gather*}

	\noindent since $\tau_0 \in \intco{0,\pi}$ and thus $e^{\tau_0 \abs[0]{ \Im z}} \geq 1$, $F$ satisfies the hypotheses of the extension of Hadamard's three lines lemma \ref{lem:EHTL} with 

		\begin{equation*}
			A =  \log( mn ) + \sum_{j = 1}^n\sum_{k = 1}^m\del[2]{ \frac{p}{p_0} + \frac{p}{p_1}}\log(1 + a_j) + \del[2]{ \frac{q'}{q'_0} + \frac{q'}{q'_1} } \log( 1 + b_k) + c(A_j,B_k)
	\end{equation*}

	The same calculations as in the proof of the Riesz-Thorin interpolation theorem \ref{thm:Riesz_Thorin} yield for $y \in \mathbb{R}$

	\begin{equation*}
		\norm[0]{ f_{iy}}_{L^{p_0}} = \norm[0]{ f}^{p/p_0}_{L^p} = 1 = \norm[0]{ g}_{L^{q'}}^{q'/q'_0} = \norm[0]{ g_{iy}}_{L^{q_0'}}
	\end{equation*}

	\noindent and

	\begin{equation*}
		\norm[0]{ f_{1 + iy}}_{L^{p_1}} = \norm[0]{ f}^{p/p_1}_{L^p} = 1 = \norm[0]{ g}_{L^{q'}}^{q'/q'_1} = \norm[0]{ g_{1 + iy}}_{L^{q_1'}}
	\end{equation*}

	Further

	\begin{equation*}
		\norm[0]{ F(iy)} \leq \norm[0]{ T_{iy}(f_{iy})}_{L^{q_0}} \norm[0]{ g_{iy}}_{L^{q'_0}} \leq M_0(y) \norm[0]{f_{iy}}_{L^{p_0}}\norm[0]{ g_{iy}}_{L^{q_0'}} = M_0(y)
	\end{equation*}

	\noindent and

	\begin{equation*}
		\norm[0]{ F(1 + iy)} \leq \norm[0]{ T_{1 + iy}(f_{1 + iy})}_{L^{q_1}} \norm[0]{ g_{1 + iy}}_{L^{q'_1}} \leq M_1(y) \norm[0]{f_{1 + iy}}_{L^{p_1}}\norm[0]{ g_{1 + iy}}_{L^{q_1'}} = M_1(y)
	\end{equation*}

	\noindent by H\"older's inequality and the hypotheses on the analytic family $(T_z)_{z \in \overline{S}}$. Therefore the extension of Hadamard's three lines lemma \ref{lem:EHTL} yields

	\begin{equation*}
		\abs[0]{ F(x)} \leq \exp\del[3]{ \frac{\sin(\pi x)}{2} \int_{-\infty}^\infty \sbr[3]{ \frac{\log M_0(t)}{\cosh(\pi t) - \cos(\pi x)} + \frac{\log M_1(t)}{\cosh(\pi t) + \cos(\pi x)}} \d t } = M(x)
	\end{equation*}

	\noindent for every $0 < x < 1$. Furthermore observe that

	\begin{equation*}
		F(\theta) = \int_Y T_\theta(f)g\d\nu
	\end{equation*}

	\noindent and thus by \cite[189]{folland:real_analysis:1999} 

	\begin{gather*}
		\begin{aligned}
			M_q(T_\theta(f)) &= \sup\cbr[3]{ \abs[3]{ \int_Y T_\theta(f) g \d \nu} : g \in \Sigma_Y, \norm[0]{g}_{L^{q'}}}\\
			&= \sup\cbr[0]{ \abs[0]{ F(\theta)} : g \in \Sigma_Y, \norm[0]{ g }_{L^{q'}}}\\
			&\leq M(\theta)
		\end{aligned}
	\end{gather*}

	Since $M(\theta)$ is an absolutely convergent integral (this is immediate by the growth conditions (\ref{eq:growth_conditions})) for any $0 < \theta < 1$, $M_q(T_\theta(f)) < \infty$ and thus $M_q(T_\theta(f)) = \norm[0]{ T_\theta(f)}_{L^q}$. The general statement follows by replacing $f$ with $f/\norm[0]{ f}_{L^p}$ when $\norm[0]{f}_{L^p} \neq 0$. The theorem is trivially true when $\norm[0]{ f}_{L^p} = 0$.
\end{proof}

%Real method
\section{The Real Method}
In this last section we are concerned with one interpolation theorem which uses real variables techniques for its proof. This stands in contrast with the complex variables techniques used for the previous two theorems. Proofs by real variables techniques are often more straight forward and do not make use of advanced theorems. Therefore the proofs are likely longer and less natural.

\subsection{The Marcinkiewicz Interpolation Theorem}
This theorem applies to sublinear operators (aswell as for quasilinear operators by a slight change of the constant) which is in comparison to the linearity assumed by the other interpolation theorems less restrictive on the nature of the operator in question. Unfortunately, this theorem will not provide an exact answer to the question raised in the introduction since the constant will differ as well as the continuity property of the estimate.

\vspace{2mm}

\begin{mdframed}
	\begin{theorem}\emph{(The Marcinkiewicz interpolation theorem)}
		Let $(X,\mu)$, $(Y,\nu)$ be two $\sigma$-finite measure spaces and $0 < p_0 < p_1 \leq \infty$. Further let $T$ be a sublinear operator defined on
		
		\begin{equation*}
			L^{p_0} + L^{p_1} := \cbr[0]{ f_0 + f_1 : f_0 \in L^{p_0}(X,\mu), f_1 \in L^{p_1}(X,\mu)}
		\end{equation*}
		
		\noindent and taking values in the space of measurable functions on $Y$. Assume that there exist $0 \leq A_0,A_1 < \infty$ such that

		\begin{equation}
			\norm[0]{T(f)}_{L^{p_0,\infty}} \leq A_0 \norm[0]{f}_{L^{p_0}} \qquad \text{and} \qquad \norm[0]{T(f)}_{L^{p_1,\infty}} \leq A_1 \norm[0]{f}_{L^{p_1}}\label{hyp:fp_0fp_1}
		\end{equation}

		\noindent for all $f \in L^{p_0}$, $f \in L^{p_1}$ respectively. Then for all $p_0 < p < p_1$ and for all $f \in L^p$ we have the estimate

		\begin{equation}
			\norm[0]{T(f)}_{L^p} \leq A \norm[0]{f}_{L^p}
		\end{equation}

		\noindent where

		\begin{equation}
			A := 2\del[3]{ \frac{p}{p - p_0} + \frac{p}{p_1 - p} }^{1/p}A_0^{\frac{\frac{1}{p} - \frac{1}{p_1}}{\frac{1}{p_0}-\frac{1}{p_1}}}A_1^{\frac{\frac{1}{p_0}-\frac{1}{p}}{\frac{1}{p_0}-\frac{1}{p_1}}}.
			\label{eq:constant}
		\end{equation}
	\end{theorem}
\end{mdframed}

\begin{proof}
Let us first consider the case \underline{$p_1 < \infty$}. Fix $f \in L^p(X,\mu)$, $\alpha > 0$ and $\delta > 0$ ($\delta$ will be determined later). We split $f$ using so-called \emph{cut-off} functions, by stipulating $f \equiv f_0(\cdot;\alpha,\delta) + f_1(\cdot;\alpha,\delta)$, where $f_0(\cdot;\alpha,\delta)$ is the \emph{unbounded part of $f$} and $f_1(\cdot;\alpha,\delta)$ is the \emph{bounded part of $f$}, defined by

\begin{gather}
	\begin{aligned}
		f_0(x;\alpha,\delta) &:= \begin{cases}
			f(x), & \vert f(x) \vert > \delta \alpha,\\
			0, & \vert f(x)\vert \leq \delta \alpha.
		\end{cases}\\
		f_1(x;\alpha,\delta) &:= \begin{cases}
			f(x), & \vert f(x) \vert \leq \delta \alpha,\\
			0, & \vert f(x)\vert > \delta \alpha.
		\end{cases}
	\end{aligned}
	\label{eq:cut_off}
\end{gather}

for $x \in X$. To facilitate reading I will omit the dependency of $f_0(\cdot;\alpha,\delta)$ and $f_1(\cdot;\alpha,\delta)$ upon the parameters $\alpha$ and $\delta$ in what follows and simply write $f_0$, $f_1$ respectively. 
	
\begin{lemma}
The functions $f_0$ and $f_1$ defined above satisfy $f_0 \in L^{p_0}(X,\mu)$ and $f_1 \in L^{p_1}(X,\mu)$ respectively.
\label{lem:f0f1}
\end{lemma}
	
\begin{proof}
Since $p_0 < p$ we have 

\begin{gather}
	\begin{aligned}
		\|f_0\|^{p_0}_{L^{p_0}} &= \int_{X} \vert f_0\vert^{p_0} d\mu =\int_{X} \vert f \vert^{p_0} \cdot \chi_{\left\{\vert f\vert > \delta\alpha \right\}} d\mu \overset{(\dagger)}{=} \int_{\left\{\vert f \vert > \delta\alpha \right\}} \vert f \vert^{p_0}d\mu\\ 
		&= \int_{\left\{\vert f\vert > \delta\alpha \right\}} \vert f \vert^p \vert f \vert^{p_0 - p} d\mu = \int_{\left\{\vert f\vert > \delta\alpha \right\}} \frac{\vert f \vert^p}{\vert f \vert^{p - p_0}} d\mu\\
		&\leq \frac{1}{(\delta\alpha)^{p - p_0}} \int_{\left\{\vert f\vert > \delta\alpha \right\}} \vert f \vert^p d\mu = (\delta\alpha)^{p_0 - p} \int_{X} \vert f \vert^p \cdot \chi_{\left\{\vert f\vert > \delta\alpha \right\}} d\mu\\
		& \leq (\delta\alpha)^{p_0 - p} \int_{X} \vert f \vert^p d\mu = (\delta\alpha)^{p_0 - p} \|f\|^p_{L^p} < \infty
	\end{aligned}
	\label{est:f0}
\end{gather}

Thus $f_0 \in L^{p_0}(X,\mu)$. Analogously it can be checked, that $f_1 \in L^{p_1}(X,\mu)$ by the estimate $\|f_1\|^{p_1}_{L^{p_1}} \leq (\delta\alpha)^{p_1 - p}\|f\|_{L^p}^p$.\\

	\emph{Proof of the equality $(\dagger)$.} Assume $\mu$ is defined on the $\sigma$-algebra $\mathcal{A}$. We have to proove that $\{\vert f \vert > \delta\alpha\} \in \mathcal{A}$\footnote{
		For $Y \in \mathcal{A}$ the $\mu$-integral of $f: X \rightarrow \mathbb{C}$ over $Y$ is defined to be $\displaystyle \int_Y fd\mu := \int_X f \cdot \chi_Y d\mu$. For more details see \cite[135--136]{elstrodt:mass:2011}.}.
		Since $f$ is complex-valued, we may write $f \equiv \mathrm{Re} f + i\mathrm{Im}f$ and thus $\vert f\vert^2 \equiv \mathrm{Re}^2 f + \mathrm{Im}^2f$. Since $f$ is measurable by hypothesis this implies that $ \mathrm{Re} f$ and $\mathrm{Im}f$ are measurable\footnote{For a proof see \cite[106]{elstrodt:mass:2011}}. Further for measurable real-valued functions $f,g: (X,\mathcal{A}) \rightarrow (\overline{\mathbb{R}},\overline{\mathfrak{B}})$\footnote{$\overline{\mathfrak{B}} := \sigma(\overline{\mathbb{R}})$ and $\overline{\mathfrak{B}} = \{B \cup E : B \in \mathfrak{B}, E \subseteq \{\pm \infty\}\}$.} 
		the functions $f + g$ and $f \cdot g$ are measurable\footnote{For a proof see \cite[107]{elstrodt:mass:2011}.}
		and thus $\vert f \vert^2$ is measurable. Hence $\{ \mathrm{Re}^2 f + \mathrm{Im}^2f > \lambda\} \in \mathcal{A}$\footnote{For a proof see \cite[105--106]{elstrodt:mass:2011}} for any $\lambda \in \mathbb{R}$. So especially for $\lambda := (\delta\alpha)^2$ we have $\{\vert f \vert > \delta\alpha\} \in \mathcal{A}$\footnote{This follows from the fact that $x < y$ if and only if $x^n < y^n$ for $n \in \mathbb{N}_{>0}$ and some real numbers $x,y > 0$ (see \cite[119]{zorich:analysis_I:2004}).}.
	In a similar manner it can also be prooven that $\{\vert f\vert \leq \delta \alpha\} \in \mathcal{A}$. By Lemma \ref{lem:charfun} and the fact that $f \cdot g$ is measurable for two measurable functions $f,g: (X,\mathcal{A}) \rightarrow (\mathbb{C},\mathfrak{B}^2)$\footcite[107]{elstrodt:mass:2011}, $f_0$ and $f_1$ are measurable since $f_0 \equiv f \cdot \chi_{\{\vert f \vert > \delta \alpha\}}$ and $f_1 \equiv f \cdot \chi_{\{\vert f \vert \leq \delta \alpha\}}$.\\

	One subtility is left to clear: the $\mu$-integrability of either $\vert f_1\vert^{p_0}$ or $\vert f_1 \vert^{p_1}$ requires that $\vert f_0 \vert^{p_0}$ and $\vert f_1 \vert^{p_1}$ are measurable functions. By the fact that any continuous map $g: (X,d_X) \rightarrow (Y,d_Y)$ between metric spaces is Borel-measurable (see \cite[86]{elstrodt:mass:2011}) and that the composition of measurable functions is again measurable (see \cite[87]{elstrodt:mass:2011}), the measurability of either $f_0$ or $f_1$ follows by $\vert f_0 \vert^{p_0} \equiv \cdot^{p_0} \circ \vert f \cdot \chi_{\{\vert f\vert > \delta\alpha\}}\vert$ and $\vert f_1 \vert^{p_1} \equiv \cdot^{p_1} \circ \vert f \cdot \chi_{\{\vert f \vert \leq \delta \alpha\}}\vert$ by stipulating $\cdot^{p}: (\mathbb{R}_{\geq 0},\vert \cdot \vert) \rightarrow (\mathbb{C},\vert \cdot \vert)$, $x^{p} := \exp(p \log(x))$ for $p > 0$ and $x \in \mathbb{R}_{> 0}$ and $x^p := 0$ if $x = 0$.
\end{proof}

By lemma (\ref{lem:f0f1}) we therefore have $f \equiv f_0 + f_1 \in L^{p_0} + L^{p_1}$. 

\begin{lemma}
	For fixed $\alpha > 0$, the distribution function $d_{T(f)}(\alpha)$ obeys an upper bound of the form

	\begin{equation*}
		d_{T(f)}(\alpha) \leq  \left(\frac{A_0}{\alpha/2}\right)^{p_0} \left\|f_0\right\|^{p_0}_{L^{p_0}} + \left(\frac{A_1}{\alpha/2}\right)^{p_1} \left\|f_1\right\|^{p_1}_{L^{p_1}}
	\end{equation*}
\end{lemma}

\begin{proof}
Since $T$ is a sublinear operator we have $\vert T(f) \vert = \vert T(f_0 + f_1) \vert \leq \vert T(f_0) \vert + \vert T(f_1)\vert$. Thus for any $y \in Y$ with $\vert T(f)(y) \vert > \alpha$ we therefore have either $\vert T(f_0)(y) \vert > \alpha/2$ or $\vert T(f_1)(y) \vert > \alpha/2$ 
		\footnote{Without loss of generality assume $\vert T(f_0)(y) \vert \leq \vert T(f_1)(y) \vert $. Then we have $\alpha < \vert T(f)(y)\vert \leq \vert T(f_0)(y) \vert + \vert T(f_1)(y)\vert \leq 2\vert T(f_1)(y)\vert$ (this is possible since $\mathbb{R}$ is an ordered field).}
		. Hence

\begin{equation*}
	\{\vert T(f)\vert > \alpha \} \subseteq \{\vert T(f_0) \vert > \alpha/2 \} \cup \{\vert T(f_1) \vert > \alpha/2 \}
\end{equation*}

and so by the monotonicity and subadditivity property of the measure $\mu$ we have

\begin{gather}
	\begin{aligned}
	d_{T(f)}(\alpha) &= \mu(\{\vert T(f)\vert > \alpha\})\\
	&\leq \mu(\{\vert T(f_0)\vert > \alpha/2 \} \cup \{\vert T(f_1)\vert > \alpha/2 \})\\
	&\leq \mu(\{\vert T(f_0) \vert > \alpha/2 \}) + \mu(\{\vert T(f_1)\vert > \alpha/2 \})\\
	&= d_{T(f_0)}(\alpha/2) + d_{T(f_1)}(\alpha/2)
	\label{est:T}
	\end{aligned}
\end{gather}

Now by hypothesis (\ref{hyp:fp_0}) we can estimate $d_{T(f_0)}(\alpha/2)$ as follows

\begin{gather}
	\begin{aligned}
		d_{T(f_0)}(\alpha/2) &= \left(\frac{\alpha/2}{\alpha/2}\right)^{p_0} d_{T(f_0)}(\alpha/2)\\
		&\leq \left(\frac{1}{\alpha/2}\right)^{p_0} \left[\sup\left\{ \gamma d_{T(f_0)}(\gamma)^{1/p_0}: \gamma > 0\right\}\right]^{p_0}\\
	 & = \left(\frac{1}{\alpha/2}\right)^{p_0} \|T(f_0)\|^{p_0}_{L^{p_0,\infty}}\\
	 & \leq \left(\frac{A_0}{\alpha/2}\right)^{p_0} \|f_0\|^{p_0}_{L^{p_0}}
	\label{est:p_0}
	\end{aligned}
\end{gather}

Analogously, we get  $d_{T(f_1)}(\alpha/2) \leq \left(\frac{A_1}{\alpha/2}\right)^{p_1} \|f_1\|^{p_1}_{L^{p_1}}\label{est:p_1}$ by hypothesis (\ref{hyp:fp_1}).
\end{proof}

By

\begin{gather}
	\begin{aligned}
		\int_0^{\frac{1}{\delta}\left| f\right|}\alpha^{p-p_0-1} d\lambda = 
			\begin{cases}
				\frac{1}{p-p_0}\frac{1}{\delta^{p-p_0}}\left| f \right|^{p - p_0}, & p \geq p_0 + 1\\
				=\lim_{\omega \to 0^+} \int_\omega^{\frac{1}{\delta}\left| f\right|}\alpha^{p-p_0-1} d\lambda\\
				= \lim_{\omega \to 0^+}\left[\frac{1}{p-p_0}\alpha^{p - p_0}\right]_\omega^{\frac{1}{\delta}\left| f\right|}\\
				= \frac{1}{p-p_0}\left[\frac{1}{\delta^{p-p_0}}\left| f \right|^{p - p_0} - \lim_{\omega \to 0^+} \omega^{p-p_0}\right]\\
				= \frac{1}{p-p_0}\frac{1}{\delta^{p-p_0}} \left| f\right|^{p - p_0}, & p_0 < p < p_0 + 1
			\end{cases}
	\end{aligned}
\end{gather}

and

\begin{gather}
	\begin{aligned}
		\int_{\frac{1}{\delta}\left| f\right|}^{\infty}\alpha^{p-p_1-1} d\lambda &= \lim_{\omega \to \infty} \left[ \frac{1}{p - p_1} \alpha^{p - p_1}\right]^\omega_{\frac{1}{\delta}\left| f\right|}\\
		&= \frac{1}{p - p_1} \left[ \lim_{\omega \to \infty} \omega^{p - p_1} - \frac{1}{\delta^{p - p_1}} \left| f\right|^{p - p_1}\right]\\
		&= \frac{1}{p_1 - p}\frac{1}{\delta^{p-p_1}} \left| f \right|^{p - p_1}
	\end{aligned}
\end{gather}

and the representation $\displaystyle \left\|f\right\|^p_{L^p} = p \int_0^{\infty} \alpha^{p-1}d_f(\alpha) d\lambda$ for $ 0 < p < \infty$ we get

\begin{gather}
	\begin{aligned}
		\left\|T(f)\right\|^p_{L^p} = &~p\int_0^{\infty}\alpha^{p-1}d_{T(f)} d\lambda\\
		\leq &~p\left(2A_0\right)^{p_0}\int_0^{\infty}\alpha^{p-p_0-1} \int_{\{\left| f \right| > \delta \alpha\}} \left| f\right|^{p_0}d\mu d\lambda\\
		&+ p(2A_1)^{p_1}\int_0^{\infty}\alpha^{p-p_1-1} \int_{\{\left| f \right| \leq \delta \alpha\}} \left| f \right|^{p_1}d\mu d\lambda\\
		= &~p\left(2A_0\right)^{p_0}\int_{\{\left| f \right| > 0\}} \left| f \right|^{p_0} \int_0^{\frac{1}{\delta}\left| f\right|}\alpha^{p-p_0-1} d\lambda d\mu\\
		&+ p\left(2A_0\right)^{p_0}\int_{\{\left| f \right| = 0\}} \left| f \right|^{p_0} \int_0^{\frac{1}{\delta}\left| f\right|}\alpha^{p-p_0-1} d\lambda d\mu\\
		&+ p\left(2A_1\right)^{p_1}\int_X \left| f\right|^{p_1} \int_{\frac{1}{\delta}\left| f\right|}^{\infty}\alpha^{p - p_1 - 1} d\lambda d\mu\\
		= &~p\left(2A_0\right)^{p_0}\int_X \left| f \right|^{p_0} \int_0^{\frac{1}{\delta}\left| f\right|}\alpha^{p-p_0-1} d\lambda d\mu\\
		&+ p\left(2A_1\right)^{p_1}\int_X \left| f\right|^{p_1} \int_{\frac{1}{\delta}\left| f\right|}^{\infty}\alpha^{p - p_1 - 1} d\lambda d\mu\\
		=&~\frac{p\left(2A_0\right)^{p_0}}{p-p_0}\frac{1}{\delta^{p-p_0}}\int_X \left| f \right|^{p_0}\left| f \right|^{p-p_0} d\mu\\
		&+ \frac{p\left(2A_1\right)^{p_1}}{p_1-p}\frac{1}{\delta^{p-p_1}}\int_X \left| f \right|^{p_1} \left| f\right|^{p-p_1}d\mu\\
		= &~p\left( \frac{\left(2A_0\right)^{p_0}}{p - p_0}\frac{1}{\delta^{p - p_0}} + \frac{\left(2A_1\right)^{p_1}}{p_1 - p}\delta^{p_1 - p} \right)\left\|f\right\|_{L^p}^p
	\end{aligned}
	\label{est:Tfp}
\end{gather}

		We pick $\delta > 0$ such that $\left(2A_0\right)^{p_0}\delta^{p_0 - p} = \left(2A_1\right)^{p_1}\delta^{p_1 - p}$. Solving for $\delta$ yields 

		\begin{equation}
			\delta = \frac{1}{2} \left( \frac{A_0}{A_1}\right)^{p_1/(p_1 - p_0)}
		\end{equation}

		Substituting this in estimate (\ref{est:Tfp}) leads to

		\begin{gather}
			\begin{aligned}
				\|T(f)\|_{L^p}^p &\leq p\left( \frac{(2A_0)^{p_0}}{p - p_0}\frac{2^{p - p_0}A_1^\frac{p_1(p-p_0)}{p_1-p_0}}{A_0^\frac{p_0(p-p_0)}{p_1 - p_0}} + \frac{(2A_1)^{p_1}}{p_1 - p} \frac{A_0^\frac{p_0(p_1 - p)}{p_1 - p_0}}{2^{p_1 - p}A_1^\frac{p_1(p_1 - p)}{p_1 - p_0}} \right)\|f\|_{L^p}^p\\
				&=  2^pp\left( \frac{A_0^\frac{p_0(p_1 - p)}{p_1 - p_0}A_1^\frac{p_1(p-p_0)}{p_1-p_0}}{p - p_0} + \frac{A_0^\frac{p_0(p_1 - p)}{p_1 - p_0}A_1^\frac{p_1(p - p_0)}{p_1 - p_0}}{p_1- p} \right)\|f\|_{L^p}^p\\
			\end{aligned}
		\end{gather}

		And taking the $p$-th power further

		\begin{gather}
			\begin{aligned}
				\|T(f)\|_{L^p} &\leq 2\left( \frac{p}{p - p_0} + \frac{p}{p_1- p} \right)^{1/p} A_0^\frac{p_0(p_1 - p)}{p(p_1 - p_0)}A_1^\frac{p_1(p - p_0)}{p(p_1 - p_0)}\|f\|_{L^p}\\
				&= 2\left( \frac{p}{p - p_0} + \frac{p}{p_1- p} \right)^{1/p} A_0^{\frac{p_0(p_1 - p)}{p(p_1 - p_0)}\frac{p_1}{p_1}}A_1^{\frac{p_1(p - p_0)}{p(p_1 - p_0)}\frac{p_0}{p_0}}\|f\|_{L^p}\\
				&= 2\left( \frac{p}{p - p_0} + \frac{p}{p_1- p} \right)^{1/p} A_0^{\frac{\frac{p_1 - p}{pp_1}}{\frac{p_1 - p_0}{p_0p_1}}}A_1^{\frac{\frac{p - p_0}{p_0p}}{\frac{p_1 - p_0}{p_0p_1}}}\|f\|_{L^p}\\
				&= 2\left( \frac{p}{p - p_0} + \frac{p}{p_1- p} \right)^{1/p} A_0^{\frac{\frac{1}{p} - \frac{1}{p_1}}{\frac{1}{p_0} - \frac{1}{p_1}}}A_1^{\frac{\frac{1}{p_0} - \frac{1}{p}}{\frac{1}{p_0} - \frac{1}{p_1}}}\|f\|_{L^p}
			\end{aligned}
		\end{gather}
		Assume \underline{$p_1 = \infty$}. We again use the cut-off functions defined in (\ref{eq:cut_off}) to decompose $f$.  Since $\{\vert f_1\vert > \delta\alpha \} = \emptyset$, we have 

\begin{equation*}
	\|T(f_1)\|_{L^\infty} \leq A_1 \|f_1\|_{L^\infty} = A_1 \inf \left\{B > 0: \mu(\{\vert f_1 \vert > B\}) = 0 \right\} \leq A_1\delta\alpha = \alpha/2
\end{equation*}

Provided we stipulate $\delta := 1/(2A_1)$. Therefore the set $\{\vert T(f_1) \vert > \alpha/2\}$ has measure zero (this is immediate since $\|T(f_1)\|_{L^\infty} =  \inf \left\{B > 0: \mu(\{\vert T(f_1) \vert > B\}) = 0 \right\} \leq \alpha/2 $ and any subset of a set with measure zero has itself measure zero). Thus similar to part \textbf{b.} of \textbf{(i.)} we get $d_{T(f)}(\alpha) \leq d_{T(f_0)}(\alpha/2)$.

	Hypothesis (\ref{hyp:fp_0}) yields the estimate $\displaystyle d_{T(f_0)}(\alpha/2) \leq \left(\frac{A_0}{\alpha/2}\right)^{p_0} \int_{\{2A_1\vert f \vert > \alpha\}} \vert f \vert^{p_0}d\mu$.

	Thus by \textbf{a.} and \textbf{b.}

	\begin{gather}
		\begin{aligned}
			\|T(f)\|_{L^p}^p &= p \int_0^{\infty} \alpha^{p-1}d_{T(f)} d\lambda\\
			&\leq p (2A_0)^{p_0} \int_0^{\infty} \alpha^{p-p_0-1} \int_{\{2A_1\vert f \vert > \alpha\}} \vert f \vert^{p_0}d\mu d\lambda\\
			&= p(2A_0)^{p_0} \int_X \vert f\vert^{p_0} \int_0^{2A_1\vert f \vert} \alpha^{p - p_0 - 1}d\lambda d\mu\\
			&= \frac{2^ppA_0^{p_0}A_1^{p - p_0}}{p - p_0} \int_X \vert f\vert^{p} d\mu\\
			&= \frac{2^ppA_0^{p_0}A_1^{p - p_0}}{p - p_0} \|f\|_{L^p}^p
			\label{est:p_1_infty}
		\end{aligned}
	\end{gather}

	That the constant $2^ppA_0^{p_0}A_1^{p - p_0}/(p - p_0)$ found in (\ref{est:p_1_infty}) is the $p$-th power of the one stated in the theorem can be seen by passing the constant (\ref{eq:constant}) to the limit $p_1 \rightarrow \infty$:

	\begin{gather*}
		\begin{aligned}
			\lim\limits_{p_1 \rightarrow \infty} A =& \lim\limits_{p_1 \rightarrow \infty}	\left[2\left( \frac{p}{p - p_0} + \frac{p}{p_1 - p} \right)^{1/p}A_0^{\frac{\frac{1}{p} - \frac{1}{p_1}}{\frac{1}{p_0}-\frac{1}{p_1}}}A_1^{\frac{\frac{1}{p_0}-\frac{1}{p}}{\frac{1}{p_0}-\frac{1}{p_1}}}\right]\\
			=& 2\exp\left[ \frac{1}{p} \log\left(\frac{p}{p - p_0} + \lim\limits_{p_1 \rightarrow \infty} \frac{1}{p_1}\frac{p}{1 - p\lim\limits_{p_1 \rightarrow \infty}\frac{1}{p_1}}\right)\right]\\
			& \cdot \lim\limits_{p_1 \rightarrow \infty}A_0^{\frac{\frac{1}{p} - \frac{1}{p_1}}{\frac{1}{p_0}-\frac{1}{p_1}}}\cdot\lim\limits_{p_1 \rightarrow \infty}A_1^{\frac{\frac{1}{p_0}-\frac{1}{p}}{\frac{1}{p_0}-\frac{1}{p_1}}}\\
		=& 2\left( \frac{p}{p - p_0} \right)^{1/p} \exp\left[\displaystyle\frac{\frac{1}{p} - \lim\limits_{p_1 \rightarrow + \infty}\frac{1}{p_1}}{\frac{1}{p_0}-\lim\limits_{p_1 \rightarrow \infty}\frac{1}{p_1}}\log(A_0)\right]\\
		& \cdot \exp\left[\frac{\frac{1}{p_0}-\frac{1}{p}}{\frac{1}{p_0}-\lim\limits_{p_1 \rightarrow \infty}\frac{1}{p_1}}\log(A_1)\right]\\
		=& 2\left( \frac{p}{p - p_0} \right)^{1/p} A_0^{\frac{p_0}{p}} A_1^{1 - \frac{p_0}{p}}
		\end{aligned}
	\end{gather*}
\end{proof}


\originalsectionstyle

%Appendix
\appendix
\begin{appendix}
	\section{Limit superior and limit inferior revisited}
	\begin{definition}
		Let $(X,d)$ a metric space, $E \subseteq X$, $f: E \rightarrow \mathbb{R}$ and $a \in X$ be a limit point of $E$. Then we define the \emph{upper limit of  $f$ at $a$} as

		\begin{equation*}
			\limsup_{x \to a}f(x) := \lim_{\varepsilon \todown 0} \left[ \sup\left\{ f(x) : x \in E \cap \dot{B}_\varepsilon(a)\right\}\right]
		\end{equation*}

		and the \emph{lower limit of  $f$ at $a$} as

		\begin{equation*}
			\liminf_{x \to a}f(x) := -\limsup_{x \to a}\left( -f \right)(x)
		\end{equation*}
	\end{definition}

	\begin{proposition}
		Let $(X,d)$ a metric space, $E \subseteq X$, $f,g: E \rightarrow \mathbb{R}$, where $f$ is bounded and $a \in X$ be a limit point of $E$. Then 

		\begin{equation*}
			\limsup_{x \to a} (fg)(x) = \limsup_{x \to a} f(x) \lim_{x \to a} g(x)
		\end{equation*}

		whenever both sides exist and $\lim_{x \to a}g(x) \geqslant 0$.
		\label{prop:limsup}
	\end{proposition}

	\begin{proof}
		Write

		\begin{equation*}
			fg = f\lim_{x\to a}g(x) + f\left[ g - \lim_{x \to a}g(x)\right]
		\end{equation*}

		By \cite[358]{bourbaki:general_topology:1995} we have

		\begin{gather*}
			\begin{aligned}
				\limsup_{x \to a} \left( fg \right)(x) &= \limsup_{x \to a}\left( f(x)\lim_{x\to a}g(x) + f(x)\left[ g(x) - \lim_{x \to a}g(x)\right]
 \right)\\
				&= \limsup_{x \to a}\left( f(x)\lim_{x \to a}g(x) \right) + \lim_{x \to a}\left( f(x)\left[ g(x) - \lim_{x \to a}g(x)\right]
 \right)\\
 				&= \limsup_{x \to a}\left( f(x) \lim_{x \to a}g(x) \right)
			\end{aligned}
		\end{gather*}

		since $\lim_{x \to a}\left[ g(x) - \lim_{x \to a} g(x)\right] = 0$ and $f$ is bounded. Fix $\varepsilon > 0$. Further by \cite[357]{bourbaki:general_topology:1995} and $\lim_{x \to a}g(x) \geqslant 0$ 

		\begin{equation*}
			\sup\left\{ f(x)\lim_{x \to a}g(x) : x \in E \cap \dot{B}_\varepsilon(a)\right\} = \sup\left\{ f(x) : x \in E \cap \dot{B}_\varepsilon(a)\right\}\lim_{x \to a}g(x)
		\end{equation*}

		Hence

		\begin{gather}
			\begin{aligned}
				\limsup_{x \to a} (fg)(x) &= \limsup_{x \to a}\left( f(x) \lim_{x \to a}g(x) \right)\\
				&= \lim_{\varepsilon \todown 0} \left[ \sup\left\{ f(x)\lim_{x \to a}g(x) : x \in E \cap \dot{B}_\varepsilon(a)\right\}\right]\\
				&= \lim_{\varepsilon \todown 0} \left[ \sup\left\{ f(x) : x \in E \cap \dot{B}_\varepsilon(a)\right\}\right]\lim_{x \to a}g(x)\\
				&= \limsup_{x \to a}f(x) \lim_{x \to a} g(x)
			\end{aligned}
		\end{gather}
	\end{proof}

	\section{Measure Theory}
	Let $(X,\mu)$ be a measure space. Recall, that if for each measurable set $E$ with $\mu(E) = \infty$ there exists a measurable set $F \subseteq E$ and $0 < \mu(F) < \infty$, $\mu$ is called \emph{semifinite}.

		\begin{lemma}
			Every $\sigma$-finite measure is semifinite.
		\end{lemma}

		\begin{proof}
			Let $X = \bigcup_{n \in \mathbb{N}} X_n$ where $\mu(X_n) < \infty$ and $E$ be measurabele with $\mu(E) = \infty$. By letting $Y_n := \bigcup_{k \leqslant n} X_k$, $Y_n$ is an increasing sequence. Then $E \cap Y_n$ is measurable and since $E \cap Y_n \subseteq Y_n$, $\mu(E \cap Y_n) < \infty$ for each $n \in \mathbb{N}$. By the continuity from below (see \cite[10]{cohn:measure_theory:2013} or \cite[26]{folland:real_analysis:1999}) we have

			\begin{equation*}
				\infty = \mu(E) = \mu(E \cap X) = \mu\left(E \cap \left(\bigcup_{n \in \mathbb{N}} Y_n\right)\right) = \mu\left(\bigcup_{n \in \mathbb{N}} \left( E \cap Y_n \right)\right) = \lim\limits_{n \rightarrow \infty} \mu\left( E \cap Y_n \right)
			\end{equation*}

			Hence for every $C > 0$ there exists $N \in \mathbb{N}$, such that $\infty > \mu(E \cap Y_n) > C$ for $n > N$.
		\end{proof}

\end{appendix}

\printbibliography
\end{document}
