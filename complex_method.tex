\section{The Complex Method}
This theorem will unfortunately only be applicable to linear operators but will yield quite a natural bound of the operator on the intermediate space. The proof will make strong use of complex variables technique. A major tool will be an application of the maximum modulus principle, known as \emph{Hadamard's three lines lemma}.

\subsection{Hadamard's Three Lines Lemma}
The proof of the Riesz-Thorin interpolation theorem heavily relies on Hadamard's three lines lemma which is itself based on a restatement of the maximum modulus theorem (see \cite[212]{rudin:rc_analysis:1987}) formulated in \cite[253]{rudin:rc_analysis:1987}. To do so, we have first to establish some common terminology. A complex-valued function $f$ is said to be \emph{holomorphic} in $\Omega \subseteq \mathbb{C}$ open, if $f'(z)$ exists for any $z \in \Omega$. By a region we shall mean a nonempty connected open subset of the complex plane. The restatement reads as follows.

\begin{theorem*}
	Let $\Omega \subseteq \mathbb{C}$ be a bounded region and $f$ be a continuous function on $\overline{\Omega}$ which is holomorphic in $\Omega$. Then 

	\begin{equation*}
		\abs[0]{f(z)} \leq \sup\cbr[0]{ \abs[0]{f(z)} : z \in \partial\Omega}
	\end{equation*}

	\noindent for every $z \in \Omega$. If equality holds at one point $z \in \Omega$, then $f$ is constant.
\end{theorem*}

\begin{mdframed}
	\begin{lemma}\emph{(Hadamard's three lines lemma)}
		Let $F$ be a holomorphic function in the strip $S := \cbr[0]{ z \in \mathbb{C}: 0 < \Re z < 1}$, continuous and bounded on $\overline{S}$, such that $\abs[0]{F(z)} \leq B_0$ when $\Re z = 0$ and $\abs[0]{F(z)} \leq B_1$ when $\Re z = 1$, for some $0 < B_0,B_1 < \infty$. Then $\abs[0]{F(z)}\leq B_0^{1 - \theta}B_1^\theta$ when $\Re z = \theta$, for any $0 < \theta < 1$.
		\label{lem:HTL}
	\end{lemma}
\end{mdframed}

\begin{figure}[h!tb]
	\centering
	\begin{tikzpicture}
		\draw [line width = .5mm] (0,-3.5)--(0,3.5);
		\draw [line width = .5mm] (1,-3.5)--(1,3.5);
		\draw [line width = .5mm] (4,-3.5)--(4,3.5);
		\draw (-1,0)--(5,0);
		\fill [pattern = adjusted lines, pattern color = black, line width = .2mm] (0,-3.5) rectangle (4,3.5);
		\node (0) at (-.25,-.25) {$0$};
		\node (1) at (4.25,-.25) {$1$};
		\node (T) at (1.25,-.25) {$\theta$};
		\node (RR) at (4.7,3) {$\mathbb{R} \times \mathbb{R}$};
	\end{tikzpicture}	
	\caption{Sketch of the setting of Hadamard's three lines lemma.}
	\label{fig:Hadamards_three_lines_lemma}
\end{figure}

\begin{proof}
For $z \in \overline{S}$ define 

\begin{equation*}
	G(z) := \frac{F(z)}{B_0^{1 - z}B_1^z} \qquad G_n(z) := G(z) e^{\del[0]{z^2 - 1}/n},~n \in\mathbb{N}_{>0}
\end{equation*}

$G\del{z}$ and $G_n\del{z}$ are holomorphic in $S$ by
	
\begin{equation*}
	G'(z) = \frac{F'(z) - F(z)\log\del{ B_1/B_0}}{B_0^{1 - z}B_1^z} \qquad G_n'(z) = G'(z)e^{\del[0]{z^2 - 1 }/n} + \frac{2}{n}zG_n(z)	
\end{equation*}

\noindent and $e^z \neq 0$ for every $z \in \mathbb{C}$. Further, we have

\begin{equation*}
	\abs[0]{B_0^{1 - z}B_1^z} = \del[0]{B_0^{1 - z}B_0^{1 - \overline{z}} B_1^z B_1^{\overline{z}}}^{1/2} =  B_0^{1 -\Re z}B_1^{\Re z}
\end{equation*}

Consider $0 \leq \Re z \leq 1$ and $B_0 \geq 1$. Then $B_0^{1 - \Re z} \geq 1$ and $B_0^{1 - \Re z } \geq B_0$ in the case $B_0 < 1$. Similarily, $B_1^{\Re z} \geq 1$ if $B_1 \geq 1$ and $B_1^{\Re z} \leq B_1$ if $B_1 < 1$. Hence 

\begin{equation}
	\abs[0]{B_0^{1 - z}B_1^z} \geq \min\cbr[0]{1,B_0}\min\cbr[0]{1,B_1} > 0
	\label{est:denom}
\end{equation}

\noindent for all $z \in \overline{S}$. Since $F$ is bounded on $\overline{S}$, we have $\abs[0]{F(z)} \leq L$ for some $L > 0$ and all $z \in \overline{S}$. Thus by (\ref{est:denom})

\begin{equation*}
	\abs[0]{G(z)} = \frac{\abs[0]{F(z)}}{\abs[0]{B_0^{1 - z}B_1^z}} \leq \frac{L}{\min\cbr[0]{1,B_0}\min\cbr[0]{1,B_1}} =: M
\end{equation*}

\noindent for every $z \in \overline{S}$. Fix $n \in \mathbb{N}_{>0}$ and write $z := x + iy \in \overline{S}$. Then

\begin{gather*}
	\abs[0]{G_n(z)} \leq M \del{e^{\del[0]{x^2 + 2ixy -y^2 - 1}/n} e^{\del[0]{x^2 - 2ixy -y^2 - 1}/n}}^{1/2}= M e^{-y^2/n}e^{\del[0]{x^2 - 1}/n} \leq Me^{-y^2/n}
\end{gather*}

\noindent for $0 \leq x \leq 1$. Thus
	
\begin{equation*}
	\lim_{y \to \pm \infty}\sup\cbr[0]{\abs[0]{G_n(z)} : 0 \leq x \leq 1} = 0
\end{equation*}

\noindent Hence there exist $C_0(n),C_1(n) \in \mathbb{R}$, such that 

\begin{equation*}
	\sup\cbr[0]{\abs[0]{G_n(z)} : 0 \leq x \leq 1} \leq 1
\end{equation*}

\noindent whenever $y > C_0(n)$ or $y < C_1(n)$. Letting

\begin{equation*}
	C(n) := \max\cbr[0]{\abs[0]{ C_0(n)} + 1, \abs[0]{ C_1(n) } + 1}
\end{equation*}

\begin{figure}[h!tb]
	\centering
	\begin{tikzpicture}
		\draw (0,-3.5)--(0,3.5);
		\draw (4,-3.5)--(4,3.5);
		\draw (-1,0)--(5,0);
		\draw [pattern = adjusted lines, pattern color = black, line width = .5mm] (0,-2.5) rectangle (4,2.5);
		\node (0) at (-.25,-.25) {$0$};
		\node (1) at (4.25,-.25) {$1$};
		\node (R) at (.5,2) {$\overline{R}$};
		\node (Cn) at (-.5,2.5) {$C(n)$};
		\node (-Cn) at (-.65,-2.5) {$-C(n)$};
		\node (RR) at (4.7,3) {$\mathbb{R} \times \mathbb{R}$};
	\end{tikzpicture}	
	\caption{Sketch of the rectangle $\overline{R}$.}
	\label{fig:Hadamards_three_lines_lemma_proof}
\end{figure}

\noindent we conclude $\abs[0]{ G_n(z)} \leq 1$ for all $0 \leq x \leq 1$ when $\abs{ y } \geq C(n)$. Now consider the rectangle $R := \intoo{0,1} \times \intoo[0]{-C(n),C(n)}$. We have $\abs[0]{G_n(z)} \leq 1$ on the lines $\intcc{0,1} \times \cbr[0]{\pm C(n)}$. By

\begin{equation*}
	\abs[0]{ G_n(iy)} = \frac{\abs[0]{F(iy)}}{\abs[0]{B_0^{1 - iy} B_1^{iy}}}e^{-\del[0]{y^2 + 1 }/n} \leq 1 \qquad \abs[0]{ G_n(1 + iy)} =	\frac{\abs[0]{ F(1 + iy)}}{\abs[0]{ B_0^{-iy}B_1^{1 + iy}}}e^{-y^2/n} \leq 1
\end{equation*}

\noindent we have $\abs[0]{ G_n(z)} \leq 1$ on the lines $\cbr{0} \times \intcc[0]{-C(n),C(n)}$, $\cbr{1} \times \intcc[0]{-C(n),C(n)}$. Thus $\abs[0]{ G_n(z) } \leq 1$ on $\partial R$. Since $\abs[0]{ G_n(z)}$ is continuous on $\overline{R}$, holomorphic in $R$ and $R$ is a bounded region, the maximum modulus theorem implies

\begin{equation*}
	\abs[0]{ G_n(z)} \leq \sup\cbr[0]{\abs[0]{G_n(z) } : z \in \partial R} \leq 1
\end{equation*}

\noindent for every $z \in R$. Therefore $\abs[0]{ G_n(z)} \leq 1$ on $\overline{R}$ and so $\abs[0]{ G_n(z)}\leq 1$ on $\overline{S}$. Since inequalities are preserved by limits and the modulus is a continuous function, we have that $\abs[0]{ G(z)} = \lim_{n \to \infty} \abs[0]{ G_n(z) } \leq 1$ for $z \in \overline{S}$. We conclude by 

\begin{equation*}
	\abs[0]{ F(\theta + it) } = \abs[0]{ G(\theta + it)} \abs[0]{B_0^{1 - \theta - it}B_1^{\theta + it}} \leq B_0^{1 - \theta} B_1^{\theta}
\end{equation*}

\noindent whenever $0 < \theta < 1$, $t \in \mathbb{R}$.
\end{proof}

\subsection{The Riesz-Thorin Interpolation Theorem}
Here we will answer our question raised in the introduction for the case where $T$ is a linear operator. The theorem as it is in \cite[37]{grafakos:fourier:2014} will be slightly generalized to not necessarily $\sigma$-finite spaces in the manner of \cite[200]{folland:real_analysis:1999}, but not in the most general formulation possible where $T$ is an operator defined on a larger domain. To shorten the argumentation we will introduce some notation. For two measure spaces $\del[0]{ X,\mu }$, $\del[0]{ Y,\nu }$ let $\Sigma_X$ and $\Sigma_Y$ denote the set of all finitely simple functions on $X$, $Y$ respectively. Furthermore recall, that a measure $\mu$ on a measure space $\del{X,\mu}$ is called \emph{semifinite} if for each measurable set $A$ with $\mu(A) = \infty$ there exists a measurable set $B \subseteq A$ with $0 < \mu(B) < \infty$ (see \cite[25]{folland:real_analysis:1999}). 

\vspace{2mm}

\begin{mdframed}
	\begin{theorem}\emph{(Riesz-Thorin interpolation theorem)}
		Suppose that $\del{X,\mu}$, $\del{Y,\nu}$ are measure spaces and $1 \leq p_0,p_1,q_0,q_1 \leq \infty$. If $q_0 = q_1 = \infty$, suppose also that $\nu$ is semifinite. Let $T$ be a linear operator defined on $\Sigma_X$ and taking values in the set of measurable functions on $Y$, such that for some $0 < M_0,M_1 < \infty$ the estimates 

		\begin{equation}
			\norm[0]{T(f)}_{L^{q_0}} \leq M_0\norm[0]{f}_{L^{p_0}} \quad \text{and} \quad \norm[0]{T(f)}_{L^{q_1}} \leq M_1\norm[0]{f}_{L^{p_1}}
			\label{hyp:Lq0Lq1}
		\end{equation}

		\noindent hold for all $f \in \Sigma_X$. Then for all $0 \leq \theta \leq 1$ we have

		\begin{equation}
			\norm[0]{T(f}_{L^q} \leq M_0^{1 - \theta}M_1^\theta\norm[0]{f}_{L^p}
			\label{est:boundTf}
		\end{equation}

		\noindent for all $f \in \Sigma_X$, where

		\begin{equation*}
			\frac{1}{p} = \frac{1 - \theta}{p_0} + \frac{\theta}{p_1} \quad \text{and} \quad \frac{1}{q} = \frac{1 - \theta}{q_0} + \frac{\theta}{q_1}.
		\end{equation*}
		\label{thm:Riesz_Thorin}
	\end{theorem}
\end{mdframed}

\begin{proof}
	The idea is to bound the quantity (see \cite[189]{folland:real_analysis:1999})
	
	\begin{equation*}
		M_q( T(f)) = \sup\cbr{\abs{ \int_Y T(f)g \d \nu} : g \in \Sigma_Y, \norm[0]{ g}_{L^{q'}} = 1}
	\end{equation*}

	\noindent appropriately (recall $q' := q/\del[0]{ q - 1 }$). If either $\theta = 0$ or $\theta = 1$, the estimate (\ref{est:boundTf}) follows directly from the hypotheses (\ref{hyp:Lq0Lq1}) on $T$. Thus we may assume \underline{$0 < \theta < 1$}. Furthermore, if $f \in \Sigma_X$, $\norm{ f}_{L^p} = 0$, then $f = 0$ $\mu$-a.e. and either one of the hypotheses on $T$ in (\ref{hyp:Lq0Lq1}) implies $T(f) = 0$ $\mu$-a.e. and thus the estimate (\ref{est:boundTf}) holds trivially. Therefore we can assume \underline{$\norm{f}_{L^p} \neq 0$}. Fix $f \in \Sigma_X$, $g \in \Sigma_Y$ with representation
	
\begin{equation*}
	f = \sum_{j = 1}^n a_j e^{i\alpha_j}\chi_{A_j} \qquad g = \sum_{k = 1}^m b_k e^{i\beta_k}\chi_{B_k}
\end{equation*}

\noindent where $a_j,b_k \neq 0$, $\alpha_j,\beta_k \in \mathbb{R}$ for any $j = 1,\dots,n$, $k = 1,\dots,m$, the sets $A_j$ and $B_k$ are each pairwise disjoint with $\mu( A_j),\nu(B_k) < \infty$ and so, that $\norm[0]{g}_{L^{q'}} \neq 0$. Define

\begin{equation*}
	P(z) := \frac{p}{p_0}(1 - z) + \frac{p}{p_1}z \qquad Q(z) := \frac{q'}{q'_0}(1 - z) + \frac{q'}{q'_1}z
\end{equation*}

\noindent for $z \in \mathbb{C}$ (since either $p = \infty$ implies $p_0 = p_1 = \infty$ or $q = 1$ implies $q_0 = q_1 = 1$, the functions $P$, $Q$ are well-defined). Further let
				
\begin{equation}
	f_z := \sum_{j = 1}^n a^{P(z)}_j e^{i\alpha_j}\chi_{A_j} \qquad g_z :=  \sum_{k = 1}^m b^{Q(z)}_k e^{i\beta_k}\chi_{B_k}
	\label{eq:def_fzgz}
\end{equation}
				
\noindent and 

\begin{equation}
	F(z) := \int_Y T(f_z)g_z\d\nu
	\label{eq:def_F}
\end{equation}

By (\ref{eq:def_fzgz}), (\ref{eq:def_F}) and the linearity of the operator $T$ we have

\begin{gather*}
	F(z) = \sum_{j = 1}^n\sum_{k = 1}^m a^{P(z)}_j b_k^{Q(z)} e^{i\alpha_j} e^{i\beta_k} \int_YT(\chi_{A_j})\chi_{B_k}\d\nu
\end{gather*}

Applying H\"older's inequality yields

\begin{equation}
	\begin{aligned}
		\abs{ \int_YT(\chi_{A_j})\chi_{B_k}\d\nu} &\leq \int_Y\abs[0]{ T(\chi_{A_j})\chi_{B_k}}\d\nu\\
		&= \norm[0]{T(\chi_{A_j})\chi_{B_k}}_{L^1}\\
		&\leq \norm[0]{T\del[0]{\chi_{A_j}}}_{L^{q_0}} \norm[0]{\chi_{B_k}}_{L^{q_0'}}\\
		&\leq M_0\norm[0]{\chi_{A_j}}_{L^{p_0}} \norm[0]{\chi_{B_k}}_{L^{q_0'}}\\
		&\leq M_0 \mu\del[0]{A_j}^{1/p_0}\nu\del[0]{B_k}^{1/q_0'}
		\label{est:constant_F}		
	\end{aligned}
\end{equation}

\noindent for each $j = 1,\hdots,n$, $k = 1,\hdots,m$ (even in the cases where either $p_0 = \infty$ or $q_0' = \infty$, or both, by observing that $\norm[0]{ \chi_{A}}_{L^\infty} \leq 1$ for any measurable set $A$). Thus the function $F$ is well-defined on $\mathbb{C}$. Let $t \in \mathbb{R}$. For $p,p_0 \neq \infty$

\begin{gather*}
	\begin{aligned}
		\norm[0]{f_{it}}_{L^{p_0}} &= \left(\sum_{j = 1}^n \int_{A_j} \abs[0]{ f_{it}}^{p_0} \d\mu + \int_{X \setminus \bigcup_{j = 1}^n A_j} \abs[0]{ f_{it} }^{p_0} \d\mu\right)^{1/p_0}\\
		&= \del{\sum_{j = 1}^n \abs[0]{ a_j^{P(it)} e^{i\alpha_j}}^{p_0}\int_X \chi_{A_j} \d\mu}^{1/p_0}\\
		&= \del{\sum_{j = 1}^n a_j^{p_0\Re P(it)}\mu(A_j)}^{1/p_0}\\
		&= \del{\sum_{j = 1}^n a_j^p\mu(A_j)}^{p/\left(p_0p\right)}\\
		&= \norm[0]{f}_{L^p}^{p/p_0} 
	\end{aligned}
\end{gather*}

\noindent holds. Let $p_0 = \infty$, $p \neq \infty$. Then $\norm[0]{f_{it}}_{L^{\infty}} = 1$ since $\abs[0]{ a_j^{P(it)}} = a_j^{p/p_0} = 1$ and that there exists some index $j$, such that $\mu( A_j) \neq 0$. If $p = \infty$, then $p_0 = p_1 = \infty$ and thus $P(it) = 1$. By the same considerations we have $\norm[0]{g_{it}}_{L^{q_0'}} = \norm[0]{g}_{L^{q'}}^{q'/q'_0}$. Hence

\begin{gather*}
	\begin{aligned}
		\abs[0]{ F(it)} &\leq \int_Y \abs[0]{ T(f_{it})g_{it}} \d\nu\\
		&= \norm[0]{T(f_{it}) g_{it}}_{L^1}\\
		&\leq \norm[0]{T(f_{it})}_{L^{q_0}}\norm[0]{g_{it}}_{L^{q_0'}}\\
		&\leq M_0 \norm[0]{f_{it}}_{L^{p_0}} \norm[0]{g_{it}}_{L^{q_0'}}\\
		&= M_0 \norm[0]{f}_{L^p}^{p/p_0} \norm[0]g_{L^{q'}}^{q'/q'_0}
	\end{aligned}
\end{gather*}

\noindent by H\"older's inequality. In an analogous manner we derive
				
\begin{equation*}
	\abs[0]{f_{1 + it}}_{L^{p_1}} = \norm[0]{f}_{L^p}^{p/p_1} \qquad \norm[0]{g_{1 + it}}_{L^{q_1'}} = \norm[0]{g}_{L^{q'}}^{q'/q_1'}
\end{equation*}

\noindent and thus 
				
\begin{equation*}
	\abs[0]{ F(1 + it)} \leq M_1 \norm[0]{f}_{L^p}^{p/p_1}\norm[0]{g}_{L^{q'}}^{q'/q_1'}
\end{equation*}	

Further by estimate (\ref{est:constant_F}) 

\begin{equation*}
	\begin{aligned}
		\abs[0]{ F(z)} &\leq \sum_{j = 1}^n\sum_{k = 1}^m \abs[0]{ a_j^{P(z)}} \abs[0]{ b_k^{Q(z)}} \abs{\int_Y T(\chi_{A_j})\chi_{B_k} \d\nu}\\
		&\leq M_0\sum_{j = 1}^n\sum_{k = 1}^m a_j^{\Re P(z)}b_k^{\Re Q(z)}\mu(A_j)^{1/p_0}\nu(B_k)^{1/q_0'}\\
		&\leq M_0\sum_{j = 1}^n\sum_{k = 1}^m \max\cbr[0]{ 1, a_j^{p/p_0 + p/p_1}} \max\cbr[0]{1,b_k^{q'/q_0' + q'/q_1'}}\mu(A_j)^{1/p_0}\nu(B_k)^{1/q_0'}
	\end{aligned}
\end{equation*}


Hence $F$ is bounded on $\overline{S}$ by some constant depending on $f$ and $g$ only. By 

\begin{multline*}
	F'(z) = \sum_{j = 1}^n\sum_{k = 1}^m a^{P(z)}_j\log (a_j) \del{\frac{p}{p_1} - \frac{p}{p_0}} b_k^{Q(z)}e^{i\alpha_j} e^{i\beta_k} \int_YT(\chi_{A_j})\chi_{B_k}\d\nu \\
	+  \sum_{j = 1}^n\sum_{k = 1}^m a^{P(z)}_jb_k^{Q(z)}\log(b_k)\del{ \frac{q'}{q'_1} - \frac{q'}{q'_0}}e^{i\alpha_j} e^{i\beta_k} \int_YT(\chi_{A_j})\chi_{B_k}\d\nu 
\end{multline*}

\noindent it is immediate, that $F$ is an entire function and thus holomorphic in $S$ and continuous on $\overline{S}$. Therefore lemma \ref{lem:HTL} yields

\begin{gather*}
	\abs[0]{ F(z)} \leq \del{ M_0  \norm[0]{f}_{L^p}^{p/p_0} \norm[0]{g}_{L^{q'}}^{q'/q'_0}}^{1 - \theta}\del{ M_1 \norm[0]{f}_{L^p}^{p/p_1}\norm[0]{g}_{L^{q'}}^{q'/q_1'}}^\theta = M_0^{1 - \theta}M_1^\theta \norm[0]{f}_{L^p}\norm[0]{g}_{L^{q'}}
\end{gather*}

\noindent for $\Re z = \theta$, $0 < \theta < 1$. We have

\begin{equation*}
	\cbr[0]{ T(f) \neq 0} = \bigcup_{n = 1}^\infty \cbr[0]{ \abs[0]{ T(f)} > 1/n}
\end{equation*}

\noindent and by Chebychev's inequality (see \cite[193]{folland:real_analysis:1999}) either

\begin{equation*}
	\nu\del[0]{\cbr[0]{\abs[0]{ T(f)} > 1/n}} \leq n^{q_0}\norm[0]{ T(f)}_{L^{q_0}}^{q_0} \leq n^{q_0}M_0^{q_0}\norm[0]{ f}_{L^{p_0}}^{q_0}
\end{equation*}

\noindent or

\begin{equation*}
	\nu\del[0]{\cbr[0]{\abs[0]{ T(f)} > 1/n}} \leq n^{q_1}\norm[0]{ T(f)}_{L^{q_1}}^{q_1} \leq n^{q_1}M_1^{q_1}\norm[0]{ f}_{L^{p_1}}^{q_1}
\end{equation*}

\noindent whenever $q_0 \neq \infty$ or $q_1 \neq \infty$. Therefore, the set $\cbr[0]{ T(f) \neq 0}$ is $\sigma$-finite unless $q_0 = q_1 = \infty$. Further we have $P(\theta) = Q(\theta) = 1$. Thus by

\begin{gather*}
	\begin{aligned}
		M_q( T(f)) &= \sup\cbr{ \abs{\int_Y T(f)g\d\nu} : g \in \Sigma_Y, \norm[0]{g}_{L^{q'}} = 1}\\
		&=  \sup\cbr[0]{\abs[0]{ F(\theta)} : g \in \Sigma_Y, \norm[0]{g}_{L^{q'}} = 1}\\
		&\leq M_0^{1 - \theta}M_1^\theta \norm[0]{f}_{L^p}\\
	\end{aligned}
\end{gather*}

\noindent we conclude 
	
\begin{equation*}
	\norm[0]{ T(f)}_{L^q} = M_q( T(f)) \leq M_0^{1 - \theta}M_1^\theta \norm[0]{f}_{L^p}
\end{equation*}
	
\noindent for any $f \in \Sigma_X$.
\end{proof}

\begin{remark}
	A more general version of the Riesz-Thorin interpolation theorem can be found in \textup{\cite[200--202]{folland:real_analysis:1999}}. There, a linear map $T: L^{p_0} + L^{p_1} \rightarrow L^{q_0} + L^{q_1}$ is considered. This follows using a density argument from the current version of the theorem and will not be prooven here.
	\label{rem:extension}
\end{remark}

\begin{remark}
	Using the previous remark, a standard application of the Riesz-Thorin interpolation theorem is to prove Young's inequality for convolutions \textup{\cite[22--23]{grafakos:fourier:2014}}. Let $G$ be a locally compact group and $\lambda$ be a left invariant Haar measure on $G$, furthermore we assume that $G$ is a countable union of compact subsets, hence the pair $\del{ G,\lambda }$ forms a $\sigma$-finite measure space.

\begin{theorem*}\emph{(Young's inequality)}
	Let $1 \leq p,q,r \leq \infty$ satisfy

	\begin{equation}
		\frac{1}{q} + 1 = \frac{1}{p} + \frac{1}{r}
		\label{hyp:young}
	\end{equation}

	Then for all $f \in L^p(G)$, $g \in L^r( G )$
	satisfying $\norm[0]{g}_{L^r} = \norm[0]{\tilde{g}}$ we have $f \ast g$ exists $\lambda$-a.e. and satisfies

	\begin{equation*}
		\norm[0]{f \ast g}_{L^1} \leq \norm[0]{g}_{L^r} \norm[0]{f}_{L^p}
	\end{equation*}
\end{theorem*}
	
\begin{proof}
	Fix $g \in L^r( G )$ and let $T(f) := f \ast g$ be defined on $L^1( G ) + L^{r'}( G )$. Obviously, $T$ is a linear operator by the linearity of the integral. By Minkowski's integral inequality (see exercise \textbf{1.1.6} \cite[13]{grafakos:fourier:2014}) we get

	\begin{equation*}
		\begin{aligned}
			\norm[0]{T(f)}_{L^r} &= \del{\int_G \abs{ \int_G f(y)g(y^{-1}x) \d\lambda(y)}^r \d\lambda(x)}^{1/r}\\
			&\leq \int_G \del{ \int_G \abs[0]{ f(y) }^r \abs[0]{ g(y^{-1}x) }^r \d\lambda(x)}^{1/r} \d\lambda(y)\\
			&= \int_G \abs[0]{ f(y)} \del{ \int_G \abs[0]{ g(y^{-1}x)}^r \d\lambda(y^{-1}x) }^{1/r} \d\lambda(y)\\
			&= \int_G \abs[0]{ f(y) } \del{ \int_G \abs[0]{ g(z) }^r \d\lambda(z) }^{1/r} \d\lambda(y)\\
			&\leq \norm[0]{f}_{L^1} \norm[0]{g}_{L^r}
		\end{aligned} 
	\end{equation*}

	\noindent for $f \in L^1(g,\mu)$ and $1 \leq p < \infty$. The case $r = \infty$ follows from
	
	\begin{equation*}
		\abs[0]{ (f \ast g)(x) } = \abs{ \int_G f(y)g(y^{-1}x) \d\lambda(y)} \leq \int_G \abs[0]{ f(y)} \abs[0]{g(y^{-1}x)} \d\lambda(y) \leq \norm[0]{g}_{L^\infty}\norm[0]{f}_{L^1}
	\end{equation*}
	
	By stipulating $h(y) := g(y^{-1}x)$ we have 

	\begin{equation*}
		\begin{aligned}
			\abs{(f \ast g)(x)} &= \abs{ \int_G f(y)g(y^{-1}x) \d\lambda(y)} \leq \int_G \abs[0]{ f(y)g(y^{-1}x)} \d\lambda(y)\\
			&= \norm[0]{fh}_{L^1} \leq \norm[0]{f}_{L^{r'}} \norm[0]{h}_{L^r} = \norm[0]{f}_{L^{r'}} \norm[0]{\tilde{g}}_{L^r} = \norm[0]{g}_{L^r} \norm[0]{f}_{L^{r'}}
		\end{aligned}
	\end{equation*}

	\noindent for $r < \infty$ and $f \in L^{r'}( G )$, since

	\begin{equation*}
		\norm[0]{h}^r_{L^r} = \int_G \abs[0]{ g(y^{-1}x)}^r \d\lambda(y) = \int_G \abs[0]{\tilde{g}(x^{-1}y)} \d\lambda(y) = \norm[0]{\tilde{g}}^r_{L^{r}}
	\end{equation*}

	The Riesz-Thorin interpolation theorem now yields for any $0 < \theta < 1$

	\begin{equation}
		\norm[0]{f \ast g}_{L^q} = \norm[0]{T(f)}_{L^q} \leq \norm[0]{g}_{L^r}^{1 - \theta}\norm[0]{g}_{L^r}^{\theta} \norm[0]{f}_{L^p} = \norm[0]{g}_{L^r}\norm[0]{f}_{L^p}
	\end{equation}

	\noindent where 

	\begin{equation*}
		\frac{1}{p} = \frac{1 - \theta}{1} + \frac{\theta}{r'} \qquad \frac{1}{q} = \frac{1 - \theta}{r} + \frac{\theta}{\infty}
	\end{equation*}
\end{proof}
\end{remark}

\begin{remark}
	The proof would be much shorter if we just used Minkowski's inequality \textup{\cite[21--22]{grafakos:fourier:2014}} instead of Minkowski's integral inequality. However, the proof given here is an alternative version of the one given already for Minkowski's inequality.
\end{remark}
