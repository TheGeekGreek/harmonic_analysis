\section{Interpolation of Analytic Families of Operators}
The generalization of the classical Riesz-Thorin interpolation theorem to analytic families of operators is due to \emph{E. M. Stein} and \emph{Guido  Weiss}\footnote{\href{https://projecteuclid.org/euclid.tmj/1178244785}{https://projecteuclid.org/euclid.tmj/1178244785}, last accessed \today.}. Crucial for its proof is again an application of advanced topics in complex analysis.

\subsection{Extension of Hadamard's Three Lines Lemma}
This lemma is inspired by a lemma originally proposed by I.I.Hirschman. I will stick for the most part to the proof given originally in the paper by Stein and Weiss and for some parts to the proof given in \cite[43--45]{grafakos:fourier:2014}.

\subsubsection{Auxiliary Lemmata} To shorten the proof of the extension of Hadamard's three lines lemma, I will summarize the most important facts used during the proof.

\begin{lemma}
	Let $D := \cbr[0]{z \in \mathbb{C} : \abs{z} < 1}$ be the open unit disc and 
	
	\begin{equation*}
		h\del{z} := \frac{1}{\pi i}\log\del{i\frac{1 + z}{1 - z}}
	\end{equation*}

	for $z \in \overline{D} \setminus \{\pm 1\}$ where we are taking that continuous branch of $\log z$ in the complex plane slit along the negative imaginary axis, $\mathbb{C} \setminus \del{\cbr{0} \times \intco{0,\infty}}$. Then $h$ is a holomorphic function in $D$ which maps $\overline{D}\setminus \cbr{\pm 1}$ bijectively onto the closure $\overline{S}$ of the strip $S := \cbr{z \in \mathbb{C} : 0 < \Re z < 1}$.
	\label{lem:h}
\end{lemma}

\begin{proof}
	Consider the mapping $\varphi: \mathbb{S}^2 \to \mathbb{S}^2$ defined by

	\begin{equation*}
		\varphi\del{z} := \frac{1 + z}{1 - z}
	\end{equation*}

	This $\varphi$ maps $\cbr{-1,0,1}$ to $\cbr{0,1,\infty}$ and is a conformal mapping by \cite[278--279]{rudin:rc_analysis:1987}. The segment $\intoo{-1,1}$ maps onto the positive real axis as can be verified by considering the corresponding real limits and the unit circle $\mathbb{S}^1$ passes through $-1$ and $1$, hence $\varphi\del[0]{\mathbb{S}^1}$ is a straight line through $\varphi\del{-1} = 0$. Since $\mathbb{S}^1$ makes a right angle with the real axis at $-1$ so does $\varphi\del[0]{\mathbb{S}^1}$ at $0$ by the conformality of $\varphi$. Thus $\varphi\del[0]{\mathbb{S}^1}$ is the imaginary axis. Since $\varphi\del{0} = 1$, it follows that $\varphi$ is a conformal one-to-one mapping of the open unit disc onto the open right half plane. Furthermore, $\varphi\del[0]{\overline{D}\setminus \cbr{\pm 1}} = \mathbb{H}^\times$ where $\mathbb{H}^\times := \cbr{z \in \mathbb{C} : \Re z \geq 0} \setminus \cbr{0}$. Thus $i\varphi\del[0]{\overline{D}\setminus \cbr{\pm 1}} = i\mathbb{H}^\times = \cbr{z \in \mathbb{C} : \Im z \geq 0} \setminus \cbr{0}$ and so 
	
	\begin{equation*}
		\log i\mathbb{H}^\times = \log \abs[0]{\mathbb{H}^\times} + i \arg i\mathbb{H}^\times = \cbr{z \in \mathbb{C} : 0 \leq \Im z \leq \pi}
	\end{equation*}

	Finally, multiplication with the preceeding factor $1/\del{\pi i}$ yields $h\del[0]{\overline{D} \setminus \cbr{\pm 1}} = \overline{S}$. Furthermore, we have

	\begin{equation*}
		h^{-1}\del{z} = \frac{e^{\pi iz} - i}{e^{\pi i z} + i}
	\end{equation*}
\end{proof}

\begin{lemma*}
	Let $X$ be a topological space. A function $f: X \to \intco[0]{-\infty,\infty}$ is upper semicontinuous if and only if for all $\alpha \in \mathbb{R}$ the set $f^{-1}\del[0]{\intco[0]{-\infty,\alpha}}$ is open. 
\end{lemma*}

\begin{proof}
	Suppose $f: X \to \intco[0]{-\infty,\infty}$ is upper semicontinuous and fix $\alpha \in \mathbb{R}$. We have that 

		\begin{equation*}
			f^{-1}\del[0]{ \intco[0]{-\infty,\alpha}} = \bigcup_{x \in \cbr[0]{ f < \alpha }} U_x
		\end{equation*}

		\noindent where $U_x$ is a neighbourhood of $x$ such that $f < \alpha$ for any element in $U_x$. Conversly, for $x_0 \in X$ and $\alpha > f(x_0)$ we have that $f^{-1}\del[0]{ \intco[0]{-\infty,\alpha }}$ is open and $x_0 \in f^{-1}\del[0]{\intco[0]{-\infty,\alpha}}$.
\end{proof}

\begin{lemma*}
	An upper semiconutinuous function $f: X \to \intco[0]{-\infty,\infty}$ on a compact topological space attains its supremum. In particular it is bounded from above.
	\label{lem:semicom}
\end{lemma*}

\begin{proof}
	 $f(X)$ is bounded from above since otherwise

	\begin{equation*}
		X = \bigcup_{n \in \mathbb{N}} f^{-1}\del[0]{\intco[0]{-\infty,n}}
	\end{equation*}

	\noindent would not have any finite subcover. Therefore $\sup_{x \in X}f(x)$ exists. Further we have $f(x_0) = \sup_{x \in X}f(x)$ for some $x_0 \in X$ since otherwise

	\begin{equation*}
		X = \bigcup_{n \in \mathbb{N}} \textstyle f^{-1}\del[0]{\intco[0]{-\infty,\sup_{x \in X}f(x) - 1/n}}
	\end{equation*}

	\noindent would not have any finite subcover.	
\end{proof}

\begin{lemma}
	Let $\Omega \subseteq \mathbb{C}$ and $f: \Omega \to \mathbb{C}$ continuous. Then $\log\abs[0]{ f}$ is upper semicontinuous on $\Omega$.
	\label{lem:uppersemcont}
\end{lemma}

\begin{proof}
	Let us consider the topological space $\del{ \Omega, \abs[0]{ \cdot}}$. Let $z_0 \in \Omega$ so such that $f(z_0) \neq 0$. Then $\log\abs[0]{ f}$ is continuous as a composition of continuous functions. If $M > f(z_0)$, then $M - \log\abs[0]{ f(z_0)} > 0$ and thus there exists some $\delta > 0$ such that $z \in B_\delta(z_0)$ implies $\abs[0]{ \log\abs[0]{ f(z)} - \log\abs[0]{ f(z_0)}} < M- \log\abs[0]{ f(z_0)}$ or equivalently $\log\abs[0]{ f(z)} < M$. Now let $z_0 \in \Omega$ so such that $f(z_0) = 0$. By convention $\log\abs[0]{ f(z_0)} = -\infty$. Furthermore, $M > \log\abs[0]{ f(z_0)}$ for any $M \in \mathbb{R}$. The condition $M > \log\abs[0]{ f(z)}$ is equivalent to $\abs[0]{ f(z)} < e^M$. But $f(z_0) = 0$ and so

	\begin{equation*}
		\abs[0]{ f(z)} = \abs[0]{ f(z) - f(z_0)}< e^M
	\end{equation*}

	Since $f$ is continuous at $z_0$ and $e^M > 0$ we find $\delta > 0$ such that $z \in B_\delta(z_0)$ implies $\abs[0]{ f(z)} < e^M$.
\end{proof}

\begin{lemma}
	The mapping $\Phi: \mathbb{R} \to \intoo{-\pi,0}$ defined by $\Phi(t) := -i\log( h^{-1}(it))$ is a $C^1$-Diffeomorphism with $\abs[0]{ D\Phi(t)} = \pi \sech(\pi t)$. In an analogous manner we have that $\Psi: \mathbb{R} \to \intoo{0,\pi}$, $\Psi(t) := -i\log( h^{-1}(1 + it) )$ is a $C^1$-Diffeomorphism with $\abs[0]{ D\Psi(t)} = \pi \sech(\pi t)$.
	\label{lem:change_of_variables}
\end{lemma}

\begin{proof}
	It is easier to consider $\Phi^{-1}(\varphi) = -i h(e^{i\varphi})$ and $\Psi^{-1}(\varphi) = i - i h(e^{i\varphi})$ (this already shows that $\Phi$, $\Psi$ are bijective mappings). Let us consider $\varphi$ only since the argumentation for $\Psi$ is similar. By $\abs[0]{ e^{i\varphi}} = 1$ it is immediate that $\Phi^{-1}$ is a real valued function. Furthermore, $\lim_{\varphi \to -\pi} \Phi^{-1}(\varphi) = \infty$, $\lim_{\varphi \rightarrow 0} \Phi^{-1}(\varphi) = -\infty$ and $\Phi^{-1}$ is clearly continuously differentiable. Using
	
	\begin{equation*}
		h^{-1}(it) = \frac{e^{-\pi t} - i}{e^{-\pi t} + i}
	\end{equation*}
	
	\noindent we get

	\begin{gather*}
		\abs[0]{ D\Phi(t)} = \pi\abs{ \frac{e^{-\pi t}}{e^{\pi t} - i} - \frac{e^{-\pi t}}{e^{-\pi t} + i}} = \pi\abs{ \frac{2e^{-\pi t}}{e^{-2\pi t} + 1}} = \pi \abs{ \frac{2}{e^{-\pi t} + e^{\pi t}}}= \pi \sech(\pi t)
	\end{gather*}
\end{proof}

\begin{lemma}
	Let $1/(2e - 1) \leq \rho < 1$ and $\zeta = \rho e^{i\theta}$. Then

	\begin{equation*}
		\abs[3]{ \log \abs{ \frac{1 + \zeta}{1 - \zeta}}} \leq 1 + \log \frac{1}{\abs[0]{ \cos(\theta/2)}} + \log \frac{1}{\abs[0]{ \sin(\theta/2)}}
	\end{equation*}
	\label{lem:upper_bound}
\end{lemma}

\begin{proof}
	This proof is due to Prof. Schlein. We have on the one hand

	\begin{equation*}
		\abs[0]{ 1 + \zeta} \leq 1 + \abs[0]{ \zeta} = 1 + \rho
	\end{equation*}

	\noindent and on the other hand

	\begin{equation*}
		\abs[0]{ 1 - \zeta } \geq \abs[0]{\Im\zeta} = \rho\abs[0]{\sin(\theta)}
	\end{equation*}

	Hence
	
	\begin{gather*}
		\begin{aligned}
			\log \frac{\abs[0]{ 1 + \zeta}}{\abs[0]{ 1 - \zeta }} &\leq \log\frac{1 + \rho}{\rho \abs[0]{ \sin(\theta)}}\\
			&=  \log\frac{1 + \rho}{2\rho \abs[0]{ \sin(\theta/2)}\abs[0]{ \cos(\theta/2)}}\\
			&=  \log\frac{1 + \rho}{2\rho}  +  \log \frac{1}{\abs[0]{ \sin(\theta/2)}} +  \log\frac{1}{\abs[0]{ \cos(\theta/2)}}\\
			&\leq  1 + \log \frac{1}{\abs[0]{ \sin(\theta/2)}} + \log \frac{1}{\abs[0]{ \cos(\theta/2)}}	
		\end{aligned}
	\end{gather*}

	\noindent since

	\begin{equation*}
		\frac{1 + \rho}{2\rho} = \frac{1}{2} + \frac{1}{2\rho} \leq e
	\end{equation*}

	Now by

	\begin{equation*}
		-\log \frac{\abs[0]{ 1 + \zeta }}{\abs[0]{ 1 - \zeta }}  = 	\log \frac{\abs[0]{ 1 - \zeta }}{\abs[0]{ 1 + \zeta }} 
	\end{equation*}

	\noindent which corresponds to considering $-\zeta = e^{i\pi}\zeta = e^{i(\pi + \theta)}$ in the first case, yields by invoking the identities

	\begin{equation*}
		\cos\del{ \frac{\pi + \theta}{2}} = -\sin(\theta/2) \qquad \sin\del{ \frac{\pi + \theta}{2} } = \cos(\theta/2)
	\end{equation*}

	\noindent the bound 

	\begin{equation*}
		-\log \frac{\abs[0]{ 1 + \zeta }}{\abs[0]{ 1 - \zeta }} \leq 1 + \log \frac{1}{\abs[0]{ \sin(\theta/2)}} + \log \frac{1}{\abs[0]{ \cos(\theta/2)}}	 
	\end{equation*}

	\noindent and we are done.
\end{proof}

\begin{lemma}
	Let $0 \leq \tau_0 < \pi$. Then 

	\begin{equation*}
		\frac{1}{\abs[0]{ \cos(\theta/2) }^{\tau_0/\pi}}\frac{1}{\abs[0]{ \sin(\theta/2) }^{\tau_0/\pi}} \in L^1[-\pi,\pi]
	\end{equation*}
\end{lemma}

\begin{proof}
	The case $\tau_0 = 0$ is trivial. We use \cite[153--154]{elstrodt:mass:2011}. By the symmetry of the integrand it is enough to consider 

	\begin{equation*}
		\int_{0}^\pi \frac{1}{\abs[0]{ \cos(\theta/2) }^{\tau_0/\pi}}\frac{1}{\abs[0]{ \sin(\theta/2) }^{\tau_0/\pi}} \d \theta
	\end{equation*}

	Thus we may perform the splitting 
	
	\begin{equation*}
		\int_{0}^1 \frac{1}{\abs[0]{ \cos(\theta/2) }^{\tau_0/\pi}}\frac{1}{\abs[0]{ \sin(\theta/2) }^{\tau_0/\pi}} \d \theta + \int_{1}^\pi \frac{1}{\abs[0]{ \cos(\theta/2) }^{\tau_0/\pi}}\frac{1}{\abs[0]{ \sin(\theta/2) }^{\tau_0/\pi}} \d\theta
	\end{equation*}
	
	Let us consider only the first integral, the second one is similar. Then we have 
	
	\begin{equation*}
		\frac{1}{\abs[0]{ \cos(\theta/2) }^{\tau_0/\pi}} \leq 1 	
	\end{equation*}
	
	\noindent for $0 \leq \theta \leq 1$ and 
	
	\begin{equation*}
		\lim_{\theta \searrow 0} \frac{(\theta/2)^{\tau_0/\pi}}{\sin(\theta/2)^{\tau_0/\pi}} = \lim_{\theta \searrow 0} \frac{(\theta/2)^{\tau_0/\pi}}{\del[2]{\frac{\theta}{2} + O\del{\frac{\theta^3}{8}}}^{\tau_0/\pi}} = \lim_{\theta \searrow 0} \frac{(\theta/2)^{\tau_0/\pi}}{(\theta/2)^{\tau_0/\pi}\del[2]{1 + O\del{\frac{\theta^2}{4}}}^{\tau_0/\pi}} = 1 
	\end{equation*}
	
	Thus 
	
	\begin{equation*}
		\frac{1}{\abs[0]{ \sin(\theta/2) }^{\tau_0/\pi}} \sim\frac{1}{( \theta/2 )^{\tau_0/\pi}}
	\end{equation*}
	
	\noindent when $\theta \searrow 0$ and since $\tau_0 < \pi$, the latter integral converges. Thus we conclude by the comparison theorem.
\end{proof}

\subsubsection{The Lemma}
Now we are ready to prove the extension of the lemma \ref{lem:HTL}.  Recall, that a function $f: X \to \intco{-\infty,\infty}$ defined on a topological space $X$ is said to be \emph{upper semicontinuous} if for every point $x_0 \in X$ and each real $M > f( x_0 )$ a neighbourhood $U$ of $x_0$ exists such that $M > f( x )$ for every $x \in U$ (see \cite[199]{rao:complex_analysis:1991}).

\vspace{2mm}

\begin{mdframed}
	\begin{lemma}\emph{(Hadamard's three lines lemma, extension)}
		Let $F$ be a holomorphic function in the strip $S := \cbr[0]{z \in \mathbb{C}: 0 < \mathrm{Re}z < 1}$ and continuous on $\overline{S}$, such that for some $0 < A < \infty$ and $0 \leq \tau_0 < \pi$ we have $\log \abs[0]{ F( z )} \leq A e^{\tau_0 \abs[0]{ \Im z}}$ for every $z \in \overline{S}$. Then

			\begin{equation*}
				\abs[0]{ F(z) } \leq \exp\del{ \frac{\sin(\pi x)}{2} \int_{-\infty}^\infty \sbr{ \frac{\log \abs[0]{ F(it + iy)}}{\cosh(\pi t) - \cos(\pi x)} + \frac{\log \abs[0]{ F(1 + it + iy)}}{\cosh(\pi t) + \cos(\pi x)} } \d t}
			\end{equation*}

			\noindent whenever $z := x + iy \in S$.
			\label{lem:EHTL}
	\end{lemma}
\end{mdframed}

\begin{proof}
	We will first prove the case \underline{$y = 0$.} Assume $F$ to be not identically zero (the case where $F$ is identically zero is trivial). Let $h$ be as in lemma (\ref{lem:h}) and let $\zeta := \rho e^{i\theta}$, $0 \leq \rho < 1$. Since $\zeta \in D$, we have $0 < \Re h(\zeta) < 1$ and thus the hypothesis on $F$ and lemma (\ref{lem:upper_bound}) yields

\begin{gather}
	\log \abs[0]{ F(h(\zeta))} \leq Ae^{\frac{\tau_0}{\pi}\abs[0]{ \log \abs[0]{\del[0]{1 + \zeta}/\del[0]{1 - \zeta}}}} \leq Ae^{\tau_0/\pi}\frac{1}{\abs[0]{ \cos(\theta/2) }^{\tau_0/\pi}}\frac{1}{\abs{ \sin(\theta/2) }^{\tau_0/\pi}}
	\label{est:admissible}
\end{gather}

\noindent for $1/(2e - 1) \leq \rho$. Since $0 < \tau_0 < \pi$, inequality (\ref{est:admissible}) asserts, that $\log \abs[0]{ F(h(\zeta))}$ is bounded from above by an integrable function of $\theta$, independently of $\rho \geq 1/(2e - 1)$. Furthermore we have 

	\begin{equation}
		M := \sup\cbr[1]{\log \abs[0]{ F(h(\zeta))} : \zeta \in \overline{B}_{1/\del[0]{2e - 1 }}} < \infty
	\end{equation}

	\noindent since a upper semicontinuous function on a compact space attains its supremum (see lemma \ref{lem:semicom}). Hence

	\begin{equation}
		\log\abs[0]{ F( h( \rho e^{i\theta} ))} \leq \max\cbr[4]{ M,   Ae^{\tau_0/\pi}\frac{1}{\abs[0]{ \cos(\theta/2) }^{\tau_0/\pi}}\frac{1}{\abs[0]{ \sin(\theta/2) }^{\tau_0/\pi}}} =: g( \theta )
		\label{est:log}
	\end{equation}

	\noindent for any $0 \leq \rho < 1$ where $g \in L^1\intcc{-\pi,\pi}$. Let $0 \leq \rho < R < 1$ and $a_1,\dots,a_n$ denote the zeros of $F(h(\zeta))$ for $\abs[0]{ \zeta} < R$ (since $F \circ h$ is holomorphic for $\abs[0]{ \zeta} < 1$ there are indeed only finitely many ones) multiple zeros being repeated. Then for $F(h(\zeta)) \neq 0$ we have by the \emph{Poisson-Jensen formula} (see \cite[208]{ahlfors:complex_analysis:1979})

	\begin{equation}
		\log\abs[0]{ F(h(\zeta))} = - \sum_{k = 1}^n \log\abs[3]{ \frac{R^2 - \overline{a}_k\zeta}{R\del[0]{ \zeta - a_k }}}	+ \frac{1}{2\pi}\int_{-\pi}^\pi \Re \sbr[3]{\frac{R e^{it} + \zeta}{R e^{it} - \zeta}}\log\abs[0]{ F(h(R e^{it}))}\d t
	\end{equation}

	Therefore by

	\begin{equation*}
		\begin{aligned}
			\Re \sbr[3]{\frac{R e^{it} + \zeta}{R e^{it} - \zeta}} &= \Re \sbr[3]{\frac{R^2 - 2i \Im \sbr[0]{ \zeta R e^{-it} } - \abs[0]{ \zeta }^2}{R^2 - 2\Re \sbr[0]{ \zeta R e^{-it} } + \abs[0]{ \zeta }^2}}\\ &= \Re \sbr[3]{\frac{R^2 - 2iR\rho \sin( \theta - t ) - \rho^2 }{R^2 - 2R\rho \cos( \theta - t ) + \rho^2}}\\
			&= \frac{R^2 - \rho^2}{R^2 - 2R\rho \cos( \theta - t ) + \rho^2}
		\end{aligned}
	\end{equation*}

	\noindent and since $\del[0]{ R^2 - \abs[0]{ a_k }^2 }\del[0]{ R^2 - \rho^2 }\geq 0$ for all $k = 1,\dots,n$ implies $\abs[0]{ R^2 - \overline{a}_k\zeta } \geq \abs[0]{ R\del[0]{ \zeta - a_k}}$ the estimate 

	\begin{equation}
		\log\abs[0]{ F(h(\zeta))} \leq \frac{1}{2\pi}\int_{-\pi}^\pi \frac{R^2 - \rho^2}{R^2 - 2R\rho \cos( \theta - t ) + \rho^2}\log\abs[0]{ F(h(R e^{it}))}\d t
	\end{equation}

	\noindent is valid for every $\abs[0]{ \zeta } < R$. By \cite[236]{rudin:rc_analysis:1987} we have

	\begin{equation}
		\frac{R - \rho}{R + \rho} \leq \frac{R^2 - \rho^2}{R^2 - 2R\rho\cos(\theta - \varphi) + \rho^2} \leq \frac{R + \rho}{R - \rho}
		\label{est:poisson}
	\end{equation}
	
	\noindent for $0 \leq \rho < R$. Combining (\ref{est:log}) and (\ref{est:poisson}) yields 

\begin{equation}
	\frac{R^2 - \rho^2}{R^2 - 2R\rho \cos( \theta - t ) + \rho^2}\log\abs[0]{ F(h(\zeta))} \leq \frac{R + \rho}{R - \rho}g( \theta ) =: G( \theta )
\end{equation}

\noindent where $G \in L^1\intcc{-\pi,\pi}$. For $0 < R < 1$ let

\begin{equation*}
	f_R(\varphi) := \frac{R^2 - \rho^2}{R^2 - 2R\rho\cos(\theta - \varphi) + \rho^2}\log\abs[0]{ F(h(Re^{i\varphi}))}
\end{equation*}

\noindent and for $\varphi \notin \cbr[0]{ 0,\pi}$

\begin{equation*}
	f(\varphi) := \frac{1 - \rho^2}{1 - 2\rho\cos(\theta - \varphi) + \rho^2}\log\abs[0]{ F(h(e^{i\varphi}))}
\end{equation*}

Since $\log\abs[0]{ F(h(\zeta))}$ is upper semicontinuous on $\overline{D} \setminus \cbr[0]{ \pm 1}$ by lemma \ref{lem:uppersemcont} we get

\begin{equation*}
	\begin{aligned}
		\limsup_{R \nearrow 1}f_R(\varphi) &= \limsup_{R \nearrow 1} \sbr[3]{ \frac{R^2 - \rho^2}{R^2 - 2R\rho\cos(\theta - \varphi) + \rho^2}\log\abs[0]{ F(h(Re^{i\varphi}))}}\\
		&= \lim_{R \nearrow 1} \frac{R^2 - \rho^2}{R^2 - 2R\rho\cos(\theta - \varphi) + \rho^2}\limsup_{R \nearrow 1} \log\abs[0]{ F(h(Re^{i\varphi}))}\\
		&= \frac{1 - \rho^2}{1 - 2\rho\cos(\theta - \varphi) + \rho^2}\log\abs[0]{ F(h(e^{i\varphi}))} = f(\varphi)
	\end{aligned}
\end{equation*}

\noindent using \cite[363]{bourbaki:general_topology:1995} and proposition \ref{prop:limsup}. The functions $G - f_R$ being non-negative, an application of Fatou's lemma yields

\begin{equation*}
	\int_{-\pi}^\pi \liminf_{R \nearrow 1}\sbr[0]{ G(\varphi) - f_R(\varphi)}\d\varphi \leq \liminf_{R \nearrow 1} \int_{-\pi}^\pi \sbr[0]{ G(\varphi) - f_R(\varphi)}\d\varphi
\end{equation*}

By \cite[354]{bourbaki:general_topology:1995}, we get

\begin{equation*}
	\limsup_{R \nearrow 1} \int_{-\pi}^\pi \sbr[0]{ f_R(\varphi) - G(\varphi)}\d \varphi \leq \int_{-\pi}^\pi \limsup_{R \nearrow 1}\sbr[0]{ f_R(\varphi) - G(\varphi)}\d\varphi
\end{equation*}

\noindent and thus

\begin{equation*}
\begin{aligned}
	\limsup_{R \nearrow 1} \int_{-\pi}^\pi f_R(\varphi) \d\varphi - \int_{-\pi}^\pi G(\varphi)\d\varphi &= \limsup_{R \nearrow 1} \int_{-\pi}^\pi f_R(\varphi) \d\varphi - \lim_{R \nearrow 1}\int_{-\pi}^\pi G(\varphi)\d\varphi\\
	&= \limsup_{R \nearrow 1}\int_{-\pi}^\pi \sbr[0]{ f_R(\varphi) - G(\varphi)}\d\varphi\\
	&\leq \int_{-\pi}^\pi \limsup_{R \nearrow 1} \sbr[0]{f_R(\varphi) - G(\varphi)}\d\varphi\\
	&\leq \int_{-\pi}^\pi \limsup_{R \nearrow 1} f_R(\varphi) \d\varphi - \int_{-\pi}^\pi\lim_{R \nearrow 1} G(\varphi)\d\varphi\\
	&= \int_{-\pi}^\pi \limsup_{R \nearrow 1}f_R(\varphi) \d\varphi - \int_{-\pi}^\pi G(\varphi)\d\varphi
\end{aligned}
\end{equation*}

\noindent by \cite[358]{bourbaki:general_topology:1995}. Hence

\begin{equation*}
	\limsup_{R \nearrow 1} \int_{-\pi}^\pi f_R(\varphi) \d\varphi \leq\int_{-\pi}^\pi \limsup_{R \nearrow 1}f_R(\varphi) \d\varphi
\end{equation*}

\noindent and so 

\begin{equation}
	\log\abs[0]{ F( h( \zeta ) )} \leq \frac{1}{2\pi} \int_{-\pi}^\pi \frac{1 - \rho^2}{1 - 2\rho\cos( \theta - \varphi ) + \rho^2}\log\abs[0]{ F( h( e^{i\varphi}))}\d\varphi	
	\label{eq:limit_case}
\end{equation}

The lemma now follows from (\ref{eq:limit_case}) by a change of variables. By stipulating $x := h( \zeta )$ we obtain 

\begin{multline}
	\zeta = h^{-1}(x) = \frac{e^{\pi i x}- i}{e^{\pi i x} + i} = \frac{\cos(\pi x) + i\sin(\pi x) - i}{\cos(\pi x) + i\sin(\pi x) + i}\\
		= \frac{\cos(\pi x) + i\sin(\pi x) - i}{\cos(\pi x) + i\sin(\pi x) + i}\frac{\cos(\pi x) - i\sin(\pi x) - i}{\cos(\pi x) - i\sin(\pi x) - i}\\
		= -i \frac{\cos(\pi x)}{1 + \sin(\pi x)} = \del[3]{ \frac{\cos(\pi x)}{1 + \sin(\pi x)} }e^{-i\pi/2}
	\label{eq:radius_angle}
\end{multline}

\noindent by

\begin{multline*}
	\del[0]{\cos(\pi x) + i\sin(\pi x) - i}\del[0]{\cos(\pi x) - i\sin(\pi x) - i}\\ = \cos^2(\pi x) - i\sin(\pi x)\cos(\pi x)	
		-i\cos(\pi x) + i\sin(\pi x)\cos(\pi x)\\
	 	+ \sin^2(\pi x) + \sin(\pi x) - i \cos(\pi x) - \sin(\pi x) - 1 = -2i \cos(\pi x)  
\end{multline*}

\noindent and
	
\begin{multline*}
	\del[0]{ \cos(\pi x) + i\sin(\pi x) + i }\del[0]{ \cos(\pi x) - i\sin(\pi x) - i }\\ = \cos^2(\pi x) - i\sin(\pi x)\cos(\pi x) - i\cos(\pi x) + i\sin(\pi x)\cos(\pi x)\\
	+ \sin^2(\pi x) + \sin(\pi x)+ i \cos(\pi x) + \sin(\pi x) + 1 = 2 + 2\sin(\pi x)
\end{multline*}

	From equality (\ref{eq:radius_angle}) we deduce $\rho = \frac{\cos(\pi x)}{1 + \sin(\pi x)}$, $\theta = -\frac{\pi}{2}$ if $0 < x \leq \frac{1}{2}$ and $\rho = -\frac{\cos(\pi x)}{1 + \sin(\pi x)}$, $\theta = \frac{\pi}{2}$ if $\frac{1}{2} \leq x < 1$. Let $0 < x \leq \frac{1}{2}$. Then we have

\begin{multline*}
	\frac{1 - \rho^2}{1 - 2\rho\cos(\theta - \varphi) + \rho^2} \\= \frac{1 + 2\sin(\pi x) + \sin^2(\pi x) - \cos^2(\pi x)}{1 + 2\sin(\pi x) + \sin^2(\pi x) + 2\cos(\pi x)\sin(\varphi)(1 + \sin(\pi x)) + \cos^2(\pi x)}\\
	= \frac{\sin(\pi x) + \sin^2(\pi x)}{1 + \sin(\pi x) + \cos(\pi x)\sin(\varphi)(1 + \sin(\pi x))}
	= \frac{\sin(\pi x)}{1 + \cos(\pi x)\sin(\varphi)}
\end{multline*}

\noindent and also for $\frac{1}{2} \leq x < 1$. Let $\Phi$ and $\Psi$ be defined as in lemma (\ref{lem:change_of_variables}). We have 
				
\begin{gather*}
	\begin{aligned}
		e^{i\Phi(t)} &= h^{-1}(it) = \frac{e^{-\pi t} - i}{e^{-\pi t} + i} \frac{e^{-\pi t} - i}{e^{-\pi t} - i} = \frac{e^{-2\pi t} - 2ie^{-\pi t} - 1}{e^{-2\pi t} + 1} = \frac{e^{-2\pi t} - 1}{e^{-2\pi t} + 1} - \frac{2ie^{-\pi t}}{e^{-2\pi t} + 1}\\ 
		&= \frac{e^{-2\pi t} - 1}{e^{-2\pi t} + 1} - \frac{2i}{e^{-\pi t} + e^{\pi t}} = \frac{1 - e^{2\pi t}}{1 + e^{2\pi t}} - \frac{2i}{e^{-\pi t} + e^{\pi t}} = -\tanh (\pi t) - i \sech(\pi t)
	\end{aligned}
\end{gather*}
	
\noindent and thus

\begin{gather*}
	\begin{aligned}
		\sin( \Phi(t) )\cosh(\pi t) &= \sin( -i\log( -\tanh (\pi t) - i \sech(\pi t) )) \cosh(\pi t)\\
		&= \frac{1}{2i} \sbr[3]{ -\tanh(\pi t) - i\sech(\pi t) + \frac{1}{\tanh(\pi t) + i\sech(\pi t) }}\cosh(\pi t)\\
		&= \frac{1}{2i} \sbr[3]{ \frac{\cosh(\pi t) - \tanh(\pi t) \sinh(\pi t) - 2i\tanh(\pi t) + \sech(\pi t)}{\tanh(\pi t) + i\sech(\pi t)}}\\
		&= \frac{1}{2i} \sbr[3]{ \frac{\cosh^2(\pi t) - \sinh^2(\pi t) - 2i \sinh(\pi t) + 1}{\sinh(\pi t) + i}}\\
		&= \frac{1 - i\sinh(\pi t)}{i\sinh(\pi t) - 1} = -1
	\end{aligned}
\end{gather*}

Therefore the transformation formula yields

\begin{equation*}
	\frac{1}{2\pi} \int_{-\pi}^0 \frac{\sin(\pi x)}{1 + \cos(\pi x)\sin(\varphi)} \log \abs[0]{ F(h(e^{i\varphi}))} \d\varphi = \frac{1}{2}\int_{-\infty}^\infty\frac{\sin(\pi x)}{\cosh(\pi t) - \cos(\pi x)} \log\abs[0]{ F(it)} \d t
\end{equation*}
				
\noindent and in a similar manner
		
\begin{equation*}
	\frac{1}{2\pi} \int_0^\pi \frac{\sin(\pi x)}{1 + \cos(\pi x)\sin(\varphi)} \log \abs[0]{ F(h(e^{i\varphi}))} \d\varphi = \frac{1}{2}\int_{-\infty}^\infty\frac{\sin(\pi x)}{\cosh(\pi t) + \cos(\pi x)} \log\abs[0]{ F(1 + it) } \d t
\end{equation*}

\noindent holds since

\begin{gather*}
	\begin{aligned}
		\sin(\Psi(t))\cosh(\pi t) &= \sin( -i\log( -\tanh (\pi t) + i \sech(\pi t) )) \cosh(\pi t)\\
		&= \frac{1}{2i} \sbr[3]{ -\tanh(\pi t) + i\sech(\pi t) - \frac{1}{-\tanh(\pi t) + i\sech(\pi t) }}\cosh(\pi t)\\
		&= \frac{1}{2i} \sbr[3]{ \frac{-\cosh(\pi t) + \tanh(\pi t) \sinh(\pi t) - 2i\tanh(\pi t) - \sech(\pi t)}{-\tanh(\pi t) + i\sech(\pi t)}}\\
		&= \frac{1}{2i} \sbr[3]{ \frac{- \cosh^2(\pi t) + \sinh^2(\pi t) - 2i \sinh(\pi t) - 1}{i - \sinh(\pi t)}}\\
		&= \frac{1 + i\sinh(\pi t)}{1 + i\sinh(\pi t)} = 1
	\end{aligned}
\end{gather*}

Thus the case $y = 0$ is prooven.\\
The case \underline{$y \neq 0$.} follows easily from the previous one. Fix $y \neq 0$ and define $G(z) := F(z + iy)$ for $z \in \overline{S}$. Then $G$ is a holomorphic function in $S$ and continuous on $\overline{S}$ as a composition of continuous and holomorphic functions. Moreover, the hypothesis on $F$ yields

		\begin{equation}
			\log \abs[0]{ G(z) } = \log \abs[0]{ F(z + iy) } \leq Ae^{\tau_0 \abs[0]{ \Im z + y}} \leq Ae^{\tau_0 \abs[0]{ \Im z }}e^{\tau_0 \abs[0]{ y }}
		\end{equation}

		for all $z \in \overline{S}$. The previous case yields for $G$ with $A$ replaced by $Ae^{\tau_0\abs[0]{ y }}$

		\begin{equation}
			\abs[0]{ G(x) } \leq \exp\del[3]{ \frac{\sin(\pi x)}{2} \int_{-\infty}^\infty \sbr{ \frac{\log \abs[0]{ G(it)}}{\cosh(\pi t) - \cos(\pi x)} + \frac{\log \abs[0]{ G(1 + it)}}{\cosh(\pi t) + \cos(\pi x)} } \d t}
		\end{equation}

		Now, observing $G(x) = F(x + iy)$, $G(it) = F(it + iy)$ and $G(1 + it) = F(1 + it + iy)$ yields the desired result.
\end{proof}

\begin{remark}
	Exercise \textbf{1.3.8.} \textup{\cite[48]{grafakos:fourier:2014}} shows that the name extension is an appropriate choice. For $0 < x < 1$ consider

	\begin{equation*}
		\begin{aligned}
			\frac{\sin( \pi x )}{2} \int_{-\infty}^\infty \frac{1}{\cosh(\pi t ) + \cos( \pi x )} \d t &= \frac{\sin( \pi x )}{2} \int_{-\infty}^\infty \frac{1}{\frac{1}{2}( e^{\pi t} + e^{-\pi t} ) + \cos( \pi x )} \d t\\
			&= \frac{\sin( \pi x )}{\pi} \int_{0}^\infty \frac{1}{s^2 + 2\cos( \pi x )s + 1} \d s\\
			&= \frac{\sin( \pi x )}{\pi} \int_{0}^\infty \frac{1}{( s + \cos( \pi x ) )^2 + \sin^2( \pi x )} \d s\\
			&= \frac{1}{\pi\sin( \pi x )} \int_{0}^\infty \frac{1}{\del{ \frac{s + \cos( \pi x )}{\sin( \pi x )}}^2 + 1} \d s\\
			&= \frac{1}{\pi} \int_{\cot( \pi x )}^\infty \frac{1}{u^2 + 1} \d u\\
			&= \frac{1}{\pi}\sbr{ \frac{\pi}{2} - \arctan( \cot( \pi x )) }\\
			&= x
		\end{aligned}
	\end{equation*}

	\noindent and in the same manner

	\begin{equation*}
		\frac{\sin( \pi x )}{2} \int_{-\infty}^\infty \frac{1}{\cosh( \pi t ) - \cos( \pi x )} \d t = 1 - x	
	\end{equation*}

	Assume that $F$ is holomorphic in $S$, continuous and bounded on $\overline{S}$ with $\abs[0]{F( z )} \leq B_0$ when $\Re z = 0$ and $\abs[0]{F( z )} \leq B_1$ when $\Re z = 1$ for some $0 < B_0, B_1 < \infty$. If $\abs[0]{F( z )} \leq M$ for $0 < M < \infty$, $F$ satisfies the hypothesis of lemma \ref{lem:EHTL} with $A := \log( M )$ and $\tau_0 = 0$. Therefore 
	
	\begin{equation*}
		\begin{aligned}
			\abs[0]{ F(z) } &\leq \exp\del[3]{ \frac{\sin(\pi x)}{2} \int_{-\infty}^\infty \sbr{ \frac{\log \abs[0]{ F(it + iy)}}{\cosh(\pi t) - \cos(\pi x)} + \frac{\log \abs[0]{ F(1 + it + iy)}}{\cosh(\pi t) + \cos(\pi x)} } \d t}\\
			&\leq \exp\del[3]{ \frac{\sin(\pi x)}{2} \int_{-\infty}^\infty \sbr{ \frac{\log B_0}{\cosh(\pi t) - \cos(\pi x)} + \frac{\log B_1}{\cosh(\pi t) + \cos(\pi x)} } \d t}\\
		&= \exp( x\log B_0 + ( 1 - x)\log B_1 )\\
		&= B_0^xB_1^{1 - x}
		\end{aligned}
	\end{equation*}

	\noindent whenever $z := x + iy \in S$. Hence lemma \ref{lem:EHTL} reduces to lemma \ref{lem:HTL}.
\end{remark}

\subsection{Stein-Weiss Interpolation Theorem of Analytic Families of Operators}
Because of the complex nature of its proof, the Riesz-Thorin theorem \ref{thm:Riesz_Thorin} can be extended to appropriate families of linear operators $(T_z)_{z \in \Omega}$ depending on a parameter $z \in \Omega \subseteq \mathbb{C}$. This result is due to Elias M. Stein and Guido Weiss. First we need to establish some common terminology.  

\vspace{2mm}

\begin{mdframed}
	\begin{definition}\emph{(Analytic family, admissible growth)}
		Let $\del{X,\mu}$, $\del{Y,\nu}$ be two $\sigma$-finite measure spaces and $( T_z )_{z \in \overline{S}}$, where $T_z$ is defined $\Sigma_X$ and taking values in the space of all measurable functions on $Y$ such that

		\begin{equation}
			\int_Y \abs[0]{ T_z(\chi_A)\chi_B} \d\nu
		\end{equation}

		\noindent whenever $\mu(A),\nu(B) < \infty$. The family $( T_z )_{z \in \overline{S}}$ is said to be \emph{analytic} if for all $f \in \Sigma_X$, $g \in \Sigma_Y$ we have that

		\begin{equation}
			z \mapsto \int_Y T_z(f)g\d\nu
		\end{equation}

		\noindent is analytic on $S$ and continuous on $\overline{S}$. Further, an analytic family $( T_z )_{z \in \overline{S}}$ is called of \emph{admissible growth}, if there is a constant $\tau_0 \in \intco{0,\pi}$, such that for all $f \in \Sigma_X$, $g \in \Sigma_Y$ a constant $C(f,g)$ exists with

			\begin{equation}
				\log\abs[3]{ \int_Y T_z(f) g \d\nu} \leq C(f,g)e^{\tau_0\abs[0]{ \Im z}}
			\end{equation}

			\noindent for all $z \in \overline{S}$.
	\end{definition}
\end{mdframed}

\vspace{2mm}

Now we are able to formulate the theorem.

\vspace{2mm}

\begin{mdframed}
	\begin{theorem}\emph{(Stein-Weiss interpolation theorem of Analytic Families of Operators)}
		Let $( T_z )_{z \in \overline{S}}$ be an analytic family of admissible growth, $1 \leq p_0,p_1,q_0,q_1 \leq \infty$ and suppose that $M_0$, $M_1$ are positive functions on the real line such that for some $\tau_1 \in \intco{0,\pi}$

			\begin{equation}
				\sup_{-\infty < y <\infty} e^{-\tau_1 \abs[0]{ y }} \log M_0(y) < \infty \qquad \text{and} \qquad \sup_{-\infty < y < \infty} e^{-\tau_1 \abs[0]{ y }} \log M_1(y) < \infty.
				\label{eq:growth_conditions}
			\end{equation}

			Fix $0 < \theta < 1$ and define

			\begin{equation}
				\frac{1}{p} := \frac{1 - \theta}{p_0} + \frac{\theta}{p_1} \qquad \text{and} \qquad \frac{1}{q} := \frac{1 - \theta}{q_0} + \frac{\theta}{q_1}.
			\end{equation}

			Further suppose that for all $f \in \Sigma_X$ and $y \in \mathbb{R}$ we have

			\begin{equation}
				\norm[0]{T_{iy}(f)}_{L^{q_0}} \leq M_0(y)\norm[0]{f}_{L^{p_0}} \qquad \text{and} \qquad \norm[0]{T_{1 + iy}(f)}_{L^{q_1}} \leq M_1(y)\norm[0]{f}_{L^{p_1}}.
			\end{equation}

			Then for all $f \in \Sigma_X$ we have

			\begin{equation*}
				\norm[0]{T_\theta(f)}_{L^q} \leq M(\theta)\norm[0]{f}_{L^p}
			\end{equation*}

			\noindent where for $0 < x < 1$

			\begin{equation*}
				M(x) = \exp\del[3]{ \frac{\sin(\pi x)}{2} \int_{-\infty}^\infty \sbr[3]{ \frac{\log M_0(t)}{\cosh(\pi t) - \cos(\pi x)} + \frac{\log M_1(t)}{\cosh(\pi t) + \cos(\pi x)}} \d t }.
			\end{equation*}
	\end{theorem}
\end{mdframed}

\begin{proof}
	Fix $0 < \theta < 1$ and $f \in \Sigma_X$, $g \in \Sigma_Y$ with $\norm[0]{f}_{L^p} = \norm[0]{g}_{L^{q'}} = 1$. Define $f_z$, $g_z$ as in (\ref{eq:def_fzgz}) and for $z \in \overline{S}$

	\begin{equation*}
		F(z) := \int_Y T_z(f_z)g_z \d\nu	
	\end{equation*}

	Since the family $(T_z)_{z \in \overline{S}}$ is of admissible growth we have that there exist constants $c(\chi_{A_j},\chi_{B_k})$ for any $j = 1,\dots,n$ and $k = 1,\dots,m$ such that 

	\begin{equation*}
		\log\abs[3]{\int_{B_k}T_z\del[0]{\chi_{A_j}}\d \nu} \leq c\del[0]{\chi_{A_j},\chi_{B_k}}e^{\tau_0\abs{\Im z}}
	\end{equation*}

	For shortness we will denote these constants simply by $c(A_j,B_k)$ and get

	\begin{gather*}
		\begin{aligned}
			\log \abs[0]{ F(z) } &= \log \abs[4]{ \sum_{j = 1}^n\sum_{k = 1}^m a^{P(z)}_j b_j^{Q(z)} e^{i\alpha_j} e^{i\beta_k} \int_YT_z(\chi_{A_j})(y)\chi_{B_k}(y)\d\nu(y)}\\
			&\leq \log \sbr[4]{ \sum_{j = 1}^n\sum_{k = 1}^m \max\cbr[0]{1,a_j^{p/p_0 + p/p_1}}\max\cbr[0]{1,b_k^{q'/q'_0 + q'/q'_1}} \abs[3]{\int_{B_k} T_z(\chi_{A_j}) \d\nu}}\\
			&\leq \log\sbr[4]{ \sum_{j = 1}^n\sum_{k = 1}^m \del[0]{ 1 + a_j}^{p/p_0 + p/p_1} \del[0]{1 + b_k}^{q'/q'_0 + q'/q'_1} e^{c(A_j,B_k)e^{\tau_0 \abs[0]{ \Im z}}} }\\
			&\leq  \log\sbr[4]{ \sum_{j = 1}^n\sum_{k = 1}^m e^{\log\del[0]{\del[0]{ 1 + a_j}^{p/p_0 + p/p_1}\del[0]{ 1 + b_k}^{q'/q'_0 + q'/q'_1}} + c(A_j,B_k)e^{\tau_0 \abs[0]{ \Im z}}} }\\
			&\leq \log\del[1]{ mn e^{\sum_{j = 1}^n\sum_{k = 1}^m\log\del[0]{\del[0]{ 1 + a_j}^{p/p_0 + p/p_1} \del[0]{1 + b_k}^{q'/q'_0 + q'/q'_1}} + c(A_j,B_k)e^{\tau_0 \abs[0]{ \Im z}}} }\\
			&= \log(mn) + \sum_{j = 1}^n\sum_{k = 1}^m\log\del[0]{\del[0]{1 + a_j}^{p/p_0 + p/p_1}\del[0]{1 +  b_k}^{q'/q'_0 + q'/q'_1}} + c(A_j,B_k)e^{\tau_0 \abs[0]{ \Im z}}
		\end{aligned}
	\end{gather*}

	\noindent since $\tau_0 \in \intco{0,\pi}$ and thus $e^{\tau_0 \abs[0]{ \Im z}} \geq 1$, $F$ satisfies the hypotheses of the extension of Hadamard's three lines lemma \ref{lem:EHTL} with 

		\begin{equation*}
			A =  \log( mn ) + \sum_{j = 1}^n\sum_{k = 1}^m\del[2]{ \frac{p}{p_0} + \frac{p}{p_1}}\log(1 + a_j) + \del[2]{ \frac{q'}{q'_0} + \frac{q'}{q'_1} } \log( 1 + b_k) + c(A_j,B_k)
	\end{equation*}

	The same calculations as in the proof of the Riesz-Thorin interpolation theorem \ref{thm:Riesz_Thorin} yield for $y \in \mathbb{R}$

	\begin{equation*}
		\norm[0]{ f_{iy}}_{L^{p_0}} = \norm[0]{ f}^{p/p_0}_{L^p} = 1 = \norm[0]{ g}_{L^{q'}}^{q'/q'_0} = \norm[0]{ g_{iy}}_{L^{q_0'}}
	\end{equation*}

	\noindent and

	\begin{equation*}
		\norm[0]{ f_{1 + iy}}_{L^{p_1}} = \norm[0]{ f}^{p/p_1}_{L^p} = 1 = \norm[0]{ g}_{L^{q'}}^{q'/q'_1} = \norm[0]{ g_{1 + iy}}_{L^{q_1'}}
	\end{equation*}

	Further

	\begin{equation*}
		\norm[0]{ F(iy)} \leq \norm[0]{ T_{iy}(f_{iy})}_{L^{q_0}} \norm[0]{ g_{iy}}_{L^{q'_0}} \leq M_0(y) \norm[0]{f_{iy}}_{L^{p_0}}\norm[0]{ g_{iy}}_{L^{q_0'}} = M_0(y)
	\end{equation*}

	\noindent and

	\begin{equation*}
		\norm[0]{ F(1 + iy)} \leq \norm[0]{ T_{1 + iy}(f_{1 + iy})}_{L^{q_1}} \norm[0]{ g_{1 + iy}}_{L^{q'_1}} \leq M_1(y) \norm[0]{f_{1 + iy}}_{L^{p_1}}\norm[0]{ g_{1 + iy}}_{L^{q_1'}} = M_1(y)
	\end{equation*}

	\noindent by H\"older's inequality and the hypotheses on the analytic family $(T_z)_{z \in \overline{S}}$. Therefore the extension of Hadamard's three lines lemma \ref{lem:EHTL} yields

	\begin{equation*}
		\abs[0]{ F(x)} \leq \exp\del[3]{ \frac{\sin(\pi x)}{2} \int_{-\infty}^\infty \sbr[3]{ \frac{\log M_0(t)}{\cosh(\pi t) - \cos(\pi x)} + \frac{\log M_1(t)}{\cosh(\pi t) + \cos(\pi x)}} \d t } = M(x)
	\end{equation*}

	\noindent for every $0 < x < 1$. Furthermore observe that

	\begin{equation*}
		F(\theta) = \int_Y T_\theta(f)g\d\nu
	\end{equation*}

	\noindent and thus by \cite[189]{folland:real_analysis:1999} 

	\begin{gather*}
		\begin{aligned}
			M_q(T_\theta(f)) &= \sup\cbr[3]{ \abs[3]{ \int_Y T_\theta(f) g \d \nu} : g \in \Sigma_Y, \norm[0]{g}_{L^{q'}}}\\
			&= \sup\cbr[0]{ \abs[0]{ F(\theta)} : g \in \Sigma_Y, \norm[0]{ g }_{L^{q'}}}\\
			&\leq M(\theta)
		\end{aligned}
	\end{gather*}

	Since $M(\theta)$ is an absolutely convergent integral (this is immediate by the growth conditions (\ref{eq:growth_conditions})) for any $0 < \theta < 1$, $M_q(T_\theta(f)) < \infty$ and thus $M_q(T_\theta(f)) = \norm[0]{ T_\theta(f)}_{L^q}$. The general statement follows by replacing $f$ with $f/\norm[0]{ f}_{L^p}$ when $\norm[0]{f}_{L^p} \neq 0$. The theorem is trivially true when $\norm[0]{ f}_{L^p} = 0$.
\end{proof}
